\documentclass[russian]{article}
\usepackage[T2A,T1]{fontenc}
\usepackage[utf8]{inputenc}
\usepackage[a4paper]{geometry}
\geometry{verbose,tmargin=2cm,bmargin=2cm,lmargin=2cm,rmargin=2cm}
\usepackage{babel}

\title{Лабораторная по производным}

\begin{document}

\maketitle

\section{Постановка задачи}

Ваша программа должна уметь брать производные. Во входном файле \texttt{deriv.in} содержится некоторое количество строк. Каждая строка представляет собой описание функции, и на каждую функцию вы должны вывести в выходной файл \texttt{deriv.out} производную этой функции.

\section{Формат входных данных}
Требуется поддерживать следующие операции (в порядке возрастания их приоритета):
\begin{enumerate}
 \item Сложение и вычитание (обозначаются \texttt{+} и \texttt{-} соответственно).
 \item По желанию -- унарный минус.
 \item Умножение и деление  (обозначаются \texttt{*} и \texttt{/} соответственно).
 \item Возведение в степень (обозначается \texttt{**}).
 \item Применение \textit{натурального} логарифма, тригонометрических и обратных тригонометрических функций (\texttt{ln}, \texttt{sin}, \texttt{cos}, \texttt{tg}, \texttt{ctg}, \texttt{arcsin}, \texttt{arctg}).
\end{enumerate}

Пример записи функции $\frac{x^2 + \sqrt[3]{x^4 + 1}}{\ln (x^{x + \cos x} + 1)}$:

\texttt{(x ** 2 + (x ** 4 + 1) ** (1 / 3)) / ln(x ** (x + cos(x)) + 1)} 

\section{Формат выходных данных}
В выходой файл должна быть выведена производная функции, записанная по тем же правилам, что и основная функция. Она может быть не упрощена, то есть конструкции вида \texttt{(x * 0 - 1 * 1) / (x) ** 2} считаются валидными.

\section{Формы сдачи задания}

\textit{Первый способ (рекомендуемый)} 

Вы можете просто прислать мне в контакт или телеграм ссылку на свой репозиторий, в котором хранится код вашей программы. Я просто cклонирую ваш репозиторий и сам протестирую вашу программу. Если вы до этого не работали с системами контроля версий и слабо представляете, что такое репозиторий, почтиайте про то, что такое git и github.com. 

Важное замечание: в репозитории должен храниться файл readme, который описывает, как надо запускать вашу программу. Пожалуйста, пишите его так, как будто хотите, чтобы ваша бабушка запустила эту программу. Я в душе не помню, как запускать программы на паскале. С другими языками я справлюсь, но там тоже могут быть неочевидные нюансы, о которых вам проще написать в readme.
Учтите, пожалуйста, что я не могу тестить проги на винде, поэтому программы на C++ я буду компилировать с помощью gcc версии 6.1.1. Это не должно вызвать особых накладок в контексте наших лаб, но теоретически может повлиять на компилируемость и поведение вашей программы. С остальными языками таких проблем возникнуть не должно, как мне кажется.

\textit{Второй способ}

Будут проведены одна-две очные сдачи. Точная дата будет сообщена позже, но ориентируйтесь на последнюю неделю декабря. На очных сдачах я также буду давать вашей программе тестовые файлы, а потом тестить ее вывод.

\section{Сроки сдачи задания}

Первый способ сдачи действителен до конца этого года (да, даже в 23:59 31 декабря можно прислать). Второй способ -- до последней очной сдачи, что логично.

\end{document}
