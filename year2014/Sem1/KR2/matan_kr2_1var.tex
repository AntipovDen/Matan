%% LyX 2.1.0 created this file.  For more info, see http://www.lyx.org/.
%% Do not edit unless you really know what you are doing.
\documentclass[russian]{article}
\usepackage[T2A,T1]{fontenc}
\usepackage[utf8]{inputenc}
\usepackage[a4paper]{geometry}
\geometry{verbose,tmargin=1cm,bmargin=0cm,lmargin=0cm,rmargin=0cm}

\makeatletter

%%%%%%%%%%%%%%%%%%%%%%%%%%%%%% LyX specific LaTeX commands.
\DeclareRobustCommand{\cyrtext}{%
  \fontencoding{T2A}\selectfont\def\encodingdefault{T2A}}
\DeclareRobustCommand{\textcyr}[1]{\leavevmode{\cyrtext #1}}
\AtBeginDocument{\DeclareFontEncoding{T2A}{}{}}


\makeatother

\usepackage{babel}
\begin{document}

\begin{tabular}{cc}
\begin{tabular}{l}
Вариант Andrea \tabularnewline
Продифференцируйте f(x):\tabularnewline
$f(x) = x^\frac{2}{\ln x} - 2x^{\log_x e} e^{1+\ln x} + e^{1+\frac{2}{\log_x e}}$\tabularnewline
\noalign{\vskip4mm}

Посчитайте предел, пользуясь правилом Лопиталя:\tabularnewline
$\lim\limits_{x \to +\infty} (\pi - 2\arctan\sqrt{x})\sqrt{x}$\tabularnewline
\noalign{\vskip4mm}

Разложите по формуле Тейлора с остатком $o((x - 1)^n)$:\tabularnewline
$f(x) = \ln \sqrt[4]{\frac{x - 2}{5 - x}}$\tabularnewline
\noalign{\vskip4mm}

Посчитайте предел, пользуясь формулой Тейлора:\tabularnewline
$\lim\limits_{x \to 0} \left(\frac{x \sin x}{2 \cosh - 2}\right)^\frac{1}{\sin^2 x}$\tabularnewline
\noalign{\vskip4mm}

\end{tabular}& %
\begin{tabular}{l}
Вариант Beth \tabularnewline
Продифференцируйте f(x) 50 раз:\tabularnewline
$f(x) =  (x^2 - 1)(4 \sin^3 x + \sin 3x)$\tabularnewline
\noalign{\vskip4mm}

Посчитайте предел, пользуясь правилом Лопиталя:\tabularnewline
$\lim\limits_{x \to +\infty} (\pi - 2\arctan\sqrt{x})\sqrt{x}$\tabularnewline
\noalign{\vskip4mm}

Разложите по формуле Тейлора с остатком $o((x-1)^{2n})$:\tabularnewline
$f(x) = \frac{x^2-2x+1}{\sqrt[3]{x(2 - x)}}$\tabularnewline
\noalign{\vskip4mm}

Посчитайте предел, пользуясь формулой Тейлора:\tabularnewline
$\lim\limits_{x \to 0} (\sqrt{1 + 2 \tan x} + \ln(1 - x))^\frac{1}{x^2}$\tabularnewline
\noalign{\vskip4mm}

\end{tabular}\tabularnewline
\noalign{\vskip4mm}
\begin{tabular}{l}
Вариант Bob \tabularnewline
Продифференцируйте f(x) 50 раз:\tabularnewline
$f(x) =  (x^2 - 1)(4 \sin^3 x + \sin 3x)$\tabularnewline
\noalign{\vskip4mm}

Посчитайте предел, пользуясь правилом Лопиталя:\tabularnewline
$\lim\limits_{x \to +\infty} (\pi - 2\arctan\sqrt{x})\sqrt{x}$\tabularnewline
\noalign{\vskip4mm}

Разложите по формуле Тейлора с остатком $o((x-1)^{2n+1})$:\tabularnewline
$f(x) = (3x^2 - 6x + 4)e^{2x^2-4x+5}$\tabularnewline
\noalign{\vskip4mm}

Посчитайте предел, пользуясь формулой Тейлора:\tabularnewline
$\lim\limits_{x \to 0} \left(\frac{x \sin x}{2 \cosh - 2}\right)^\frac{1}{\sin^2 x}$\tabularnewline
\noalign{\vskip4mm}

\end{tabular}& %
\begin{tabular}{l}
Вариант Carl Grimes \tabularnewline
Продифференцируйте f(x):\tabularnewline
$f(x) = x^\frac{2}{\ln x} - 2x^{\log_x e} e^{1+\ln x} + e^{1+\frac{2}{\log_x e}}$\tabularnewline
\noalign{\vskip4mm}

Посчитайте предел, пользуясь правилом Лопиталя:\tabularnewline
$\lim\limits_{x \to +\infty} (\pi - 2\arctan\sqrt{x})\sqrt{x}$\tabularnewline
\noalign{\vskip4mm}

Разложите по формуле Тейлора с остатком $o((x-1)^{2n})$:\tabularnewline
$f(x) = \frac{x^2-2x+1}{\sqrt[3]{x(2 - x)}}$\tabularnewline
\noalign{\vskip4mm}

Посчитайте предел, пользуясь формулой Тейлора:\tabularnewline
$\lim\limits_{x \to 0} (\sqrt{1 + 2 \tan x} + \ln(1 - x))^\frac{1}{x^2}$\tabularnewline
\noalign{\vskip4mm}

\end{tabular}\tabularnewline
\noalign{\vskip4mm}
\begin{tabular}{l}
Вариант Carol \tabularnewline
Продифференцируйте f(x):\tabularnewline
$f(x) = x^\frac{2}{\ln x} - 2x^{\log_x e} e^{1+\ln x} + e^{1+\frac{2}{\log_x e}}$\tabularnewline
\noalign{\vskip4mm}

Посчитайте предел, пользуясь правилом Лопиталя:\tabularnewline
$\lim\limits_{x \to +\infty} (\pi - 2\arctan\sqrt{x})\sqrt{x}$\tabularnewline
\noalign{\vskip4mm}

Разложите по формуле Тейлора с остатком $o((x-1)^{2n})$:\tabularnewline
$f(x) = \frac{x^2-2x+1}{\sqrt[3]{x(2 - x)}}$\tabularnewline
\noalign{\vskip4mm}

Посчитайте предел, пользуясь формулой Тейлора:\tabularnewline
$\lim\limits_{x \to 0} \left(\frac{x \sin x}{2 \cosh - 2}\right)^\frac{1}{\sin^2 x}$\tabularnewline
\noalign{\vskip4mm}

\end{tabular}& %
\begin{tabular}{l}
Вариант Daryl \tabularnewline
Продифференцируйте f(x):\tabularnewline
$f(x) = x^\frac{2}{\ln x} - 2x^{\log_x e} e^{1+\ln x} + e^{1+\frac{2}{\log_x e}}$\tabularnewline
\noalign{\vskip4mm}

Посчитайте предел, пользуясь правилом Лопиталя:\tabularnewline
$\lim\limits_{x \to 0} \sin x \ln \cot x$\tabularnewline
\noalign{\vskip4mm}

Разложите по формуле Тейлора с остатком $o((x - 1)^n)$:\tabularnewline
$f(x) = \ln \sqrt[4]{\frac{x - 2}{5 - x}}$\tabularnewline
\noalign{\vskip4mm}

Посчитайте предел, пользуясь формулой Тейлора:\tabularnewline
$\lim\limits_{x \to 0} \left(\frac{x \sin x}{2 \cosh - 2}\right)^\frac{1}{\sin^2 x}$\tabularnewline
\noalign{\vskip4mm}

\end{tabular}\tabularnewline
\noalign{\vskip4mm}
\begin{tabular}{l}
Вариант Glenn \tabularnewline
Продифференцируйте f(x):\tabularnewline
$f(x) = x^\frac{2}{\ln x} - 2x^{\log_x e} e^{1+\ln x} + e^{1+\frac{2}{\log_x e}}$\tabularnewline
\noalign{\vskip4mm}

Посчитайте предел, пользуясь правилом Лопиталя:\tabularnewline
$\lim\limits_{x \to 0} \sin x \ln \cot x$\tabularnewline
\noalign{\vskip4mm}

Разложите по формуле Тейлора с остатком $o((x-1)^{2n+1})$:\tabularnewline
$f(x) = (3x^2 - 6x + 4)e^{2x^2-4x+5}$\tabularnewline
\noalign{\vskip4mm}

Посчитайте предел, пользуясь формулой Тейлора:\tabularnewline
$\lim\limits_{x \to 0} \frac{\ln (1 + x) + \frac{1}{2}\sinh (x^2) - x}{\sqrt{1 + \tan x} - \sqrt{1 + \sin x}}$\tabularnewline
\noalign{\vskip4mm}

\end{tabular}& %
\begin{tabular}{l}
Вариант Hershel \tabularnewline
Продифференцируйте f(x):\tabularnewline
$f(x) = x^\frac{2}{\ln x} - 2x^{\log_x e} e^{1+\ln x} + e^{1+\frac{2}{\log_x e}}$\tabularnewline
\noalign{\vskip4mm}

Посчитайте предел, пользуясь правилом Лопиталя:\tabularnewline
$\lim\limits_{x \to +\infty} (\pi - 2\arctan\sqrt{x})\sqrt{x}$\tabularnewline
\noalign{\vskip4mm}

Разложите по формуле Тейлора с остатком $o((x-1)^{2n})$:\tabularnewline
$f(x) = \frac{x^2-2x+1}{\sqrt[3]{x(2 - x)}}$\tabularnewline
\noalign{\vskip4mm}

Посчитайте предел, пользуясь формулой Тейлора:\tabularnewline
$\lim\limits_{x \to 0} \left(\frac{x \sin x}{2 \cosh - 2}\right)^\frac{1}{\sin^2 x}$\tabularnewline
\noalign{\vskip4mm}

\end{tabular}\tabularnewline
\noalign{\vskip4mm}
\end{tabular}


\begin{tabular}{cc}
\begin{tabular}{l}
Вариант Lori Grimes \tabularnewline
Продифференцируйте f(x) 50 раз:\tabularnewline
$f(x) =  (x^2 - 1)(4 \sin^3 x + \sin 3x)$\tabularnewline
\noalign{\vskip4mm}

Посчитайте предел, пользуясь правилом Лопиталя:\tabularnewline
$\lim\limits_{x \to +\infty} (\pi - 2\arctan\sqrt{x})\sqrt{x}$\tabularnewline
\noalign{\vskip4mm}

Разложите по формуле Тейлора с остатком $o((x-1)^{2n})$:\tabularnewline
$f(x) = \frac{x^2-2x+1}{\sqrt[3]{x(2 - x)}}$\tabularnewline
\noalign{\vskip4mm}

Посчитайте предел, пользуясь формулой Тейлора:\tabularnewline
$\lim\limits_{x \to 0} \frac{\ln (1 + x) + \frac{1}{2}\sinh (x^2) - x}{\sqrt{1 + \tan x} - \sqrt{1 + \sin x}}$\tabularnewline
\noalign{\vskip4mm}

\end{tabular}& %
\begin{tabular}{l}
Вариант Meggie \tabularnewline
Продифференцируйте f(x):\tabularnewline
$f(x) = x^\frac{2}{\ln x} - 2x^{\log_x e} e^{1+\ln x} + e^{1+\frac{2}{\log_x e}}$\tabularnewline
\noalign{\vskip4mm}

Посчитайте предел, пользуясь правилом Лопиталя:\tabularnewline
$\lim\limits_{x \to 0} \sin x \ln \cot x$\tabularnewline
\noalign{\vskip4mm}

Разложите по формуле Тейлора с остатком $o((x-1)^{2n+1})$:\tabularnewline
$f(x) = (3x^2 - 6x + 4)e^{2x^2-4x+5}$\tabularnewline
\noalign{\vskip4mm}

Посчитайте предел, пользуясь формулой Тейлора:\tabularnewline
$\lim\limits_{x \to 0} (\sqrt{1 + 2 \tan x} + \ln(1 - x))^\frac{1}{x^2}$\tabularnewline
\noalign{\vskip4mm}

\end{tabular}\tabularnewline
\noalign{\vskip4mm}
\begin{tabular}{l}
Вариант Merle \tabularnewline
Продифференцируйте f(x) 50 раз:\tabularnewline
$f(x) =  (x^2 - 1)(4 \sin^3 x + \sin 3x)$\tabularnewline
\noalign{\vskip4mm}

Посчитайте предел, пользуясь правилом Лопиталя:\tabularnewline
$\lim\limits_{x \to +\infty} (\pi - 2\arctan\sqrt{x})\sqrt{x}$\tabularnewline
\noalign{\vskip4mm}

Разложите по формуле Тейлора с остатком $o((x-1)^{2n})$:\tabularnewline
$f(x) = \frac{x^2-2x+1}{\sqrt[3]{x(2 - x)}}$\tabularnewline
\noalign{\vskip4mm}

Посчитайте предел, пользуясь формулой Тейлора:\tabularnewline
$\lim\limits_{x \to 0} (\sqrt{1 + 2 \tan x} + \ln(1 - x))^\frac{1}{x^2}$\tabularnewline
\noalign{\vskip4mm}

\end{tabular}& %
\begin{tabular}{l}
Вариант Michonne \tabularnewline
Продифференцируйте f(x):\tabularnewline
$f(x) = x^\frac{2}{\ln x} - 2x^{\log_x e} e^{1+\ln x} + e^{1+\frac{2}{\log_x e}}$\tabularnewline
\noalign{\vskip4mm}

Посчитайте предел, пользуясь правилом Лопиталя:\tabularnewline
$\lim\limits_{x \to +\infty} (\pi - 2\arctan\sqrt{x})\sqrt{x}$\tabularnewline
\noalign{\vskip4mm}

Разложите по формуле Тейлора с остатком $o((x-1)^{2n})$:\tabularnewline
$f(x) = \frac{x^2-2x+1}{\sqrt[3]{x(2 - x)}}$\tabularnewline
\noalign{\vskip4mm}

Посчитайте предел, пользуясь формулой Тейлора:\tabularnewline
$\lim\limits_{x \to 0} \left(\frac{x \sin x}{2 \cosh - 2}\right)^\frac{1}{\sin^2 x}$\tabularnewline
\noalign{\vskip4mm}

\end{tabular}\tabularnewline
\noalign{\vskip4mm}
\begin{tabular}{l}
Вариант Rick Grimes \tabularnewline
Продифференцируйте f(x) 50 раз:\tabularnewline
$f(x) =  (x^2 - 1)(4 \sin^3 x + \sin 3x)$\tabularnewline
\noalign{\vskip4mm}

Посчитайте предел, пользуясь правилом Лопиталя:\tabularnewline
$\lim\limits_{x \to +\infty} (\pi - 2\arctan\sqrt{x})\sqrt{x}$\tabularnewline
\noalign{\vskip4mm}

Разложите по формуле Тейлора с остатком $o((x - 1)^n)$:\tabularnewline
$f(x) = \ln \sqrt[4]{\frac{x - 2}{5 - x}}$\tabularnewline
\noalign{\vskip4mm}

Посчитайте предел, пользуясь формулой Тейлора:\tabularnewline
$\lim\limits_{x \to 0} \left(\frac{x \sin x}{2 \cosh - 2}\right)^\frac{1}{\sin^2 x}$\tabularnewline
\noalign{\vskip4mm}

\end{tabular}& %
\begin{tabular}{l}
Вариант Sasha \tabularnewline
Продифференцируйте f(x) 50 раз:\tabularnewline
$f(x) =  (x^2 - 1)(4 \sin^3 x + \sin 3x)$\tabularnewline
\noalign{\vskip4mm}

Посчитайте предел, пользуясь правилом Лопиталя:\tabularnewline
$\lim\limits_{x \to +\infty} (\pi - 2\arctan\sqrt{x})\sqrt{x}$\tabularnewline
\noalign{\vskip4mm}

Разложите по формуле Тейлора с остатком $o((x - 1)^n)$:\tabularnewline
$f(x) = \ln \sqrt[4]{\frac{x - 2}{5 - x}}$\tabularnewline
\noalign{\vskip4mm}

Посчитайте предел, пользуясь формулой Тейлора:\tabularnewline
$\lim\limits_{x \to 0} (\sqrt{1 + 2 \tan x} + \ln(1 - x))^\frac{1}{x^2}$\tabularnewline
\noalign{\vskip4mm}

\end{tabular}\tabularnewline
\noalign{\vskip4mm}
\begin{tabular}{l}
Вариант Shane \tabularnewline
Продифференцируйте f(x):\tabularnewline
$f(x) = x^\frac{2}{\ln x} - 2x^{\log_x e} e^{1+\ln x} + e^{1+\frac{2}{\log_x e}}$\tabularnewline
\noalign{\vskip4mm}

Посчитайте предел, пользуясь правилом Лопиталя:\tabularnewline
$\lim\limits_{x \to +\infty} (\pi - 2\arctan\sqrt{x})\sqrt{x}$\tabularnewline
\noalign{\vskip4mm}

Разложите по формуле Тейлора с остатком $o((x-1)^{2n+1})$:\tabularnewline
$f(x) = (3x^2 - 6x + 4)e^{2x^2-4x+5}$\tabularnewline
\noalign{\vskip4mm}

Посчитайте предел, пользуясь формулой Тейлора:\tabularnewline
$\lim\limits_{x \to 0} \left(\frac{x \sin x}{2 \cosh - 2}\right)^\frac{1}{\sin^2 x}$\tabularnewline
\noalign{\vskip4mm}

\end{tabular}& %
\begin{tabular}{l}
Вариант Tara \tabularnewline
Продифференцируйте f(x):\tabularnewline
$f(x) = x^\frac{2}{\ln x} - 2x^{\log_x e} e^{1+\ln x} + e^{1+\frac{2}{\log_x e}}$\tabularnewline
\noalign{\vskip4mm}

Посчитайте предел, пользуясь правилом Лопиталя:\tabularnewline
$\lim\limits_{x \to +\infty} (\pi - 2\arctan\sqrt{x})\sqrt{x}$\tabularnewline
\noalign{\vskip4mm}

Разложите по формуле Тейлора с остатком $o((x-1)^{2n+1})$:\tabularnewline
$f(x) = (3x^2 - 6x + 4)e^{2x^2-4x+5}$\tabularnewline
\noalign{\vskip4mm}

Посчитайте предел, пользуясь формулой Тейлора:\tabularnewline
$\lim\limits_{x \to 0} \left(\frac{x \sin x}{2 \cosh - 2}\right)^\frac{1}{\sin^2 x}$\tabularnewline
\noalign{\vskip4mm}

\end{tabular}\tabularnewline
\noalign{\vskip4mm}
\end{tabular}
\begin{tabular}{cc}
\begin{tabular}{l}
Вариант The Governor \tabularnewline
Продифференцируйте f(x):\tabularnewline
$f(x) = x^\frac{2}{\ln x} - 2x^{\log_x e} e^{1+\ln x} + e^{1+\frac{2}{\log_x e}}$\tabularnewline
\noalign{\vskip4mm}

Посчитайте предел, пользуясь правилом Лопиталя:\tabularnewline
$\lim\limits_{x \to 0} \sin x \ln \cot x$\tabularnewline
\noalign{\vskip4mm}

Разложите по формуле Тейлора с остатком $o((x-1)^{2n+1})$:\tabularnewline
$f(x) = (3x^2 - 6x + 4)e^{2x^2-4x+5}$\tabularnewline
\noalign{\vskip4mm}

Посчитайте предел, пользуясь формулой Тейлора:\tabularnewline
$\lim\limits_{x \to 0} \left(\frac{x \sin x}{2 \cosh - 2}\right)^\frac{1}{\sin^2 x}$\tabularnewline
\noalign{\vskip4mm}

\end{tabular}& %
\begin{tabular}{l}
Вариант Tyreese \tabularnewline
Продифференцируйте f(x):\tabularnewline
$f(x) = x^\frac{2}{\ln x} - 2x^{\log_x e} e^{1+\ln x} + e^{1+\frac{2}{\log_x e}}$\tabularnewline
\noalign{\vskip4mm}

Посчитайте предел, пользуясь правилом Лопиталя:\tabularnewline
$\lim\limits_{x \to 0} \sin x \ln \cot x$\tabularnewline
\noalign{\vskip4mm}

Разложите по формуле Тейлора с остатком $o((x-1)^{2n})$:\tabularnewline
$f(x) = \frac{x^2-2x+1}{\sqrt[3]{x(2 - x)}}$\tabularnewline
\noalign{\vskip4mm}

Посчитайте предел, пользуясь формулой Тейлора:\tabularnewline
$\lim\limits_{x \to 0} \left(\frac{x \sin x}{2 \cosh - 2}\right)^\frac{1}{\sin^2 x}$\tabularnewline
\noalign{\vskip4mm}

\end{tabular}\tabularnewline
\noalign{\vskip4mm}
\end{tabular}


\end{document}