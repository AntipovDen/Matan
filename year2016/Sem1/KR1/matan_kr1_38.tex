
\documentclass[russian]{article}
\usepackage[T2A,T1]{fontenc}
\usepackage[utf8]{inputenc}
\usepackage[a4paper]{geometry}
\geometry{verbose,tmargin=1cm,bmargin=0cm,lmargin=0cm,rmargin=0cm}
\usepackage{amsmath}

\makeatletter

\DeclareRobustCommand{\cyrtext}{%
  \fontencoding{T2A}\selectfont\def\encodingdefault{T2A}}
\DeclareRobustCommand{\textcyr}[1]{\leavevmode{\cyrtext #1}}
\AtBeginDocument{\DeclareFontEncoding{T2A}{}{}}


\makeatother

\usepackage[russian]{babel}
\begin{document}
 
\begin{tabular}{cc}
\begin{tabular}{l}
Вариант Анетерон \tabularnewline
Постройте график в декартовых координатах:\tabularnewline
$y = \log_3\frac{x + 2}{x}$\tabularnewline
\noalign{\vskip4mm}

Докажите предел по определению:\tabularnewline
$\lim\limits_{n \to +\infty} \left(\frac{n + 1}{2n}\right)^n = 0$\tabularnewline
\noalign{\vskip4mm}

Докажите, что последовательность сходится:\tabularnewline
$x_1 = 1, x_n = \sqrt[3]{6 + x_{n - 1}}$\tabularnewline
\noalign{\vskip4mm}

Вычислите предел функции:\tabularnewline
$\lim\limits_{x \to 0} \left(\frac{\ch 2x}{\ch x}\right)^\frac{1}{x^2}$\tabularnewline
\noalign{\vskip4mm}

\end{tabular}& %
\begin{tabular}{l}
Вариант Анубарак \tabularnewline
Постройте график в декартовых координатах:\tabularnewline
$y = \log_3\frac{x + 2}{x}$\tabularnewline
\noalign{\vskip4mm}

Докажите предел по определению:\tabularnewline
$\lim\limits_{n \to +\infty} \left(\frac{n + 1}{2n}\right)^n = 0$\tabularnewline
\noalign{\vskip4mm}

Докажите, что последовательность сходится:\tabularnewline
$x_n = \sum_{k = 1}^n \frac{(-1)^{k - 1}}{k(k + 1)}$\tabularnewline
\noalign{\vskip4mm}

Вычислите предел функции:\tabularnewline
$\lim\limits_{x \to \pi} \frac{1 - \cos x \cos 2x \cos 3x}{1 + \cos x}$\tabularnewline
\noalign{\vskip4mm}

\end{tabular}\tabularnewline
\noalign{\vskip4mm}
\begin{tabular}{l}
Вариант Артес \tabularnewline
Постройте график в декартовых координатах:\tabularnewline
$y = \sin x + \sqrt{3} \cos x$\tabularnewline
\noalign{\vskip4mm}

Докажите предел по определению:\tabularnewline
$\lim\limits_{n \to +\infty} \left(8 - \frac{1}{n^2}\right)^{-\frac{1}{3}} = \frac{1}{2}$\tabularnewline
\noalign{\vskip4mm}

Докажите, что последовательность сходится:\tabularnewline
$x_n = \sum_{k = 1}^n \frac{(-1)^{k - 1}}{k(k + 1)}$\tabularnewline
\noalign{\vskip4mm}

Вычислите предел функции:\tabularnewline
$\lim\limits_{x \to 0} \left(\frac{\ch 2x}{\ch x}\right)^\frac{1}{x^2}$\tabularnewline
\noalign{\vskip4mm}

\end{tabular}& %
\begin{tabular}{l}
Вариант Архимонд \tabularnewline
Постройте график в декартовых координатах:\tabularnewline
$y = \log_3\frac{x + 2}{x}$\tabularnewline
\noalign{\vskip4mm}

Докажите предел по определению:\tabularnewline
$\lim\limits_{n \to +\infty} \frac{\sqrt[3]{n^2 + n}}{n + 2} = 0$\tabularnewline
\noalign{\vskip4mm}

Докажите, что последовательность сходится:\tabularnewline
$x_n = \sum_{k = 1}^n \frac{(-1)^{k - 1}}{k(k + 1)}$\tabularnewline
\noalign{\vskip4mm}

Вычислите предел функции:\tabularnewline
$\lim\limits_{x \to \pi} \frac{1 - \cos x \cos 2x \cos 3x}{1 + \cos x}$\tabularnewline
\noalign{\vskip4mm}

\end{tabular}\tabularnewline
\noalign{\vskip4mm}
\begin{tabular}{l}
Вариант Бальназар \tabularnewline
Постройте график в декартовых координатах:\tabularnewline
$y = \log_3\frac{x + 2}{x}$\tabularnewline
\noalign{\vskip4mm}

Докажите предел по определению:\tabularnewline
$\lim\limits_{n \to +\infty} \frac{\sqrt[3]{n^2 + n}}{n + 2} = 0$\tabularnewline
\noalign{\vskip4mm}

Докажите, что последовательность сходится:\tabularnewline
$x_n = \sum_{k = 1}^n \frac{(-1)^{k - 1}}{k(k + 1)}$\tabularnewline
\noalign{\vskip4mm}

Вычислите предел функции:\tabularnewline
$\lim\limits_{x \to 0} \frac{x\tg 3x}{\sqrt{1 + \sin^2 2x} - \sqrt{1 + \sin^2 x}}$\tabularnewline
\noalign{\vskip4mm}

\end{tabular}& %
\begin{tabular}{l}
Вариант Гром Задира \tabularnewline
Постройте график в полярных координатах:\tabularnewline
$r = \frac{1}{1 - \sin\phi}$\tabularnewline
\noalign{\vskip4mm}

Докажите предел по определению:\tabularnewline
$\lim\limits_{n \to +\infty} \frac{\sqrt[3]{n^2 + n}}{n + 2} = 0$\tabularnewline
\noalign{\vskip4mm}

Докажите, что последовательность сходится:\tabularnewline
$x_1 = 1, x_n = \sqrt[3]{6 + x_{n - 1}}$\tabularnewline
\noalign{\vskip4mm}

Вычислите предел функции:\tabularnewline
$\lim\limits_{x \to \pi} \frac{1 - \cos x \cos 2x \cos 3x}{1 + \cos x}$\tabularnewline
\noalign{\vskip4mm}

\end{tabular}\tabularnewline
\noalign{\vskip4mm}
\begin{tabular}{l}
Вариант Дальвенгир \tabularnewline
Постройте график в декартовых координатах:\tabularnewline
$y = \sin x + \sqrt{3} \cos x$\tabularnewline
\noalign{\vskip4mm}

Докажите предел по определению:\tabularnewline
$\lim\limits_{n \to +\infty} \left(\frac{n + 1}{2n}\right)^n = 0$\tabularnewline
\noalign{\vskip4mm}

Докажите, что последовательность расходится:\tabularnewline
$x_n = \left[\frac{n^2 + 1}{3}\right] - \frac{n^2}{3}$\tabularnewline
\noalign{\vskip4mm}

Вычислите предел функции:\tabularnewline
$\lim\limits_{x \to 0} \left(\frac{\ch 2x}{\ch x}\right)^\frac{1}{x^2}$\tabularnewline
\noalign{\vskip4mm}

\end{tabular}& %
\begin{tabular}{l}
Вариант Детерок \tabularnewline
Постройте график в полярных координатах:\tabularnewline
$r = \frac{1}{1 - \sin\phi}$\tabularnewline
\noalign{\vskip4mm}

Докажите предел по определению:\tabularnewline
$\lim\limits_{n \to +\infty} \frac{\sqrt[3]{n^2 + n}}{n + 2} = 0$\tabularnewline
\noalign{\vskip4mm}

Докажите, что последовательность сходится:\tabularnewline
$x_1 = 1, x_n = \sqrt[3]{6 + x_{n - 1}}$\tabularnewline
\noalign{\vskip4mm}

Вычислите предел функции:\tabularnewline
$\lim\limits_{x \to \pi} \frac{1 - \cos x \cos 2x \cos 3x}{1 + \cos x}$\tabularnewline
\noalign{\vskip4mm}

\end{tabular}\tabularnewline
\noalign{\vskip4mm}
\end{tabular}

\begin{tabular}{cc}
\begin{tabular}{l}
Вариант Дэлин Праудмур \tabularnewline
Постройте график в декартовых координатах:\tabularnewline
$y = \log_3\frac{x + 2}{x}$\tabularnewline
\noalign{\vskip4mm}

Докажите предел по определению:\tabularnewline
$\lim\limits_{n \to +\infty} \left(8 - \frac{1}{n^2}\right)^{-\frac{1}{3}} = \frac{1}{2}$\tabularnewline
\noalign{\vskip4mm}

Докажите, что последовательность расходится:\tabularnewline
$x_n = \left[\frac{n^2 + 1}{3}\right] - \frac{n^2}{3}$\tabularnewline
\noalign{\vskip4mm}

Вычислите предел функции:\tabularnewline
$\lim\limits_{x \to \pi} \frac{1 - \cos x \cos 2x \cos 3x}{1 + \cos x}$\tabularnewline
\noalign{\vskip4mm}

\end{tabular}& %
\begin{tabular}{l}
Вариант Иллидан \tabularnewline
Постройте график в декартовых координатах:\tabularnewline
$y = \sin x + \sqrt{3} \cos x$\tabularnewline
\noalign{\vskip4mm}

Докажите предел по определению:\tabularnewline
$\lim\limits_{n \to +\infty} \left(\frac{n + 1}{2n}\right)^n = 0$\tabularnewline
\noalign{\vskip4mm}

Докажите, что последовательность сходится:\tabularnewline
$x_1 = 1, x_n = \sqrt[3]{6 + x_{n - 1}}$\tabularnewline
\noalign{\vskip4mm}

Вычислите предел функции:\tabularnewline
$\lim\limits_{x \to 0} \frac{x\tg 3x}{\sqrt{1 + \sin^2 2x} - \sqrt{1 + \sin^2 x}}$\tabularnewline
\noalign{\vskip4mm}

\end{tabular}\tabularnewline
\noalign{\vskip4mm}
\begin{tabular}{l}
Вариант Кел'Тузед \tabularnewline
Постройте график в декартовых координатах:\tabularnewline
$y = \log_3\frac{x + 2}{x}$\tabularnewline
\noalign{\vskip4mm}

Докажите предел по определению:\tabularnewline
$\lim\limits_{n \to +\infty} \left(8 - \frac{1}{n^2}\right)^{-\frac{1}{3}} = \frac{1}{2}$\tabularnewline
\noalign{\vskip4mm}

Докажите, что последовательность расходится:\tabularnewline
$x_n = \left[\frac{n^2 + 1}{3}\right] - \frac{n^2}{3}$\tabularnewline
\noalign{\vskip4mm}

Вычислите предел функции:\tabularnewline
$\lim\limits_{x \to \pi} \frac{1 - \cos x \cos 2x \cos 3x}{1 + \cos x}$\tabularnewline
\noalign{\vskip4mm}

\end{tabular}& %
\begin{tabular}{l}
Вариант Кель \tabularnewline
Постройте график в декартовых координатах:\tabularnewline
$y = \log_3\frac{x + 2}{x}$\tabularnewline
\noalign{\vskip4mm}

Докажите предел по определению:\tabularnewline
$\lim\limits_{n \to +\infty} \frac{\sqrt[3]{n^2 + n}}{n + 2} = 0$\tabularnewline
\noalign{\vskip4mm}

Докажите, что последовательность расходится:\tabularnewline
$x_n = \left[\frac{n^2 + 1}{3}\right] - \frac{n^2}{3}$\tabularnewline
\noalign{\vskip4mm}

Вычислите предел функции:\tabularnewline
$\lim\limits_{x \to \pi} \frac{1 - \cos x \cos 2x \cos 3x}{1 + \cos x}$\tabularnewline
\noalign{\vskip4mm}

\end{tabular}\tabularnewline
\noalign{\vskip4mm}
\begin{tabular}{l}
Вариант Кил'Джеден \tabularnewline
Постройте график в декартовых координатах:\tabularnewline
$y = \log_3\frac{x + 2}{x}$\tabularnewline
\noalign{\vskip4mm}

Докажите предел по определению:\tabularnewline
$\lim\limits_{n \to +\infty} \left(8 - \frac{1}{n^2}\right)^{-\frac{1}{3}} = \frac{1}{2}$\tabularnewline
\noalign{\vskip4mm}

Докажите, что последовательность сходится:\tabularnewline
$x_n = \sum_{k = 1}^n \frac{(-1)^{k - 1}}{k(k + 1)}$\tabularnewline
\noalign{\vskip4mm}

Вычислите предел функции:\tabularnewline
$\lim\limits_{x \to 0} \frac{x\tg 3x}{\sqrt{1 + \sin^2 2x} - \sqrt{1 + \sin^2 x}}$\tabularnewline
\noalign{\vskip4mm}

\end{tabular}& %
\begin{tabular}{l}
Вариант Кэрн Кровавый Рог \tabularnewline
Постройте график в полярных координатах:\tabularnewline
$r = \frac{1}{1 - \sin\phi}$\tabularnewline
\noalign{\vskip4mm}

Докажите предел по определению:\tabularnewline
$\lim\limits_{n \to +\infty} \frac{\sqrt[3]{n^2 + n}}{n + 2} = 0$\tabularnewline
\noalign{\vskip4mm}

Докажите, что последовательность сходится:\tabularnewline
$x_n = \sum_{k = 1}^n \frac{(-1)^{k - 1}}{k(k + 1)}$\tabularnewline
\noalign{\vskip4mm}

Вычислите предел функции:\tabularnewline
$\lim\limits_{x \to 0} \frac{x\tg 3x}{\sqrt{1 + \sin^2 2x} - \sqrt{1 + \sin^2 x}}$\tabularnewline
\noalign{\vskip4mm}

\end{tabular}\tabularnewline
\noalign{\vskip4mm}
\begin{tabular}{l}
Вариант Маннорох \tabularnewline
Постройте график в декартовых координатах:\tabularnewline
$y = \log_3\frac{x + 2}{x}$\tabularnewline
\noalign{\vskip4mm}

Докажите предел по определению:\tabularnewline
$\lim\limits_{n \to +\infty} \frac{\sqrt[3]{n^2 + n}}{n + 2} = 0$\tabularnewline
\noalign{\vskip4mm}

Докажите, что последовательность сходится:\tabularnewline
$x_n = \sum_{k = 1}^n \frac{(-1)^{k - 1}}{k(k + 1)}$\tabularnewline
\noalign{\vskip4mm}

Вычислите предел функции:\tabularnewline
$\lim\limits_{x \to \pi} \frac{1 - \cos x \cos 2x \cos 3x}{1 + \cos x}$\tabularnewline
\noalign{\vskip4mm}

\end{tabular}& %
\begin{tabular}{l}
Вариант Мурадин \tabularnewline
Постройте график в декартовых координатах:\tabularnewline
$y = \sin x + \sqrt{3} \cos x$\tabularnewline
\noalign{\vskip4mm}

Докажите предел по определению:\tabularnewline
$\lim\limits_{n \to +\infty} \frac{\sqrt[3]{n^2 + n}}{n + 2} = 0$\tabularnewline
\noalign{\vskip4mm}

Докажите, что последовательность сходится:\tabularnewline
$x_1 = 1, x_n = \sqrt[3]{6 + x_{n - 1}}$\tabularnewline
\noalign{\vskip4mm}

Вычислите предел функции:\tabularnewline
$\lim\limits_{x \to \pi} \frac{1 - \cos x \cos 2x \cos 3x}{1 + \cos x}$\tabularnewline
\noalign{\vskip4mm}

\end{tabular}\tabularnewline
\noalign{\vskip4mm}
\end{tabular}

\begin{tabular}{cc}
\begin{tabular}{l}
Вариант Нер'Зул \tabularnewline
Постройте график в декартовых координатах:\tabularnewline
$y = \log_3\frac{x + 2}{x}$\tabularnewline
\noalign{\vskip4mm}

Докажите предел по определению:\tabularnewline
$\lim\limits_{n \to +\infty} \frac{\sqrt[3]{n^2 + n}}{n + 2} = 0$\tabularnewline
\noalign{\vskip4mm}

Докажите, что последовательность расходится:\tabularnewline
$x_n = \left[\frac{n^2 + 1}{3}\right] - \frac{n^2}{3}$\tabularnewline
\noalign{\vskip4mm}

Вычислите предел функции:\tabularnewline
$\lim\limits_{x \to 0} \left(\frac{\ch 2x}{\ch x}\right)^\frac{1}{x^2}$\tabularnewline
\noalign{\vskip4mm}

\end{tabular}& %
\begin{tabular}{l}
Вариант Рексар \tabularnewline
Постройте график в декартовых координатах:\tabularnewline
$y = \sin x + \sqrt{3} \cos x$\tabularnewline
\noalign{\vskip4mm}

Докажите предел по определению:\tabularnewline
$\lim\limits_{n \to +\infty} \left(8 - \frac{1}{n^2}\right)^{-\frac{1}{3}} = \frac{1}{2}$\tabularnewline
\noalign{\vskip4mm}

Докажите, что последовательность сходится:\tabularnewline
$x_n = \sum_{k = 1}^n \frac{(-1)^{k - 1}}{k(k + 1)}$\tabularnewline
\noalign{\vskip4mm}

Вычислите предел функции:\tabularnewline
$\lim\limits_{x \to 0} \frac{x\tg 3x}{\sqrt{1 + \sin^2 2x} - \sqrt{1 + \sin^2 x}}$\tabularnewline
\noalign{\vskip4mm}

\end{tabular}\tabularnewline
\noalign{\vskip4mm}
\begin{tabular}{l}
Вариант Сапфирон \tabularnewline
Постройте график в полярных координатах:\tabularnewline
$r = \frac{1}{1 - \sin\phi}$\tabularnewline
\noalign{\vskip4mm}

Докажите предел по определению:\tabularnewline
$\lim\limits_{n \to +\infty} \left(8 - \frac{1}{n^2}\right)^{-\frac{1}{3}} = \frac{1}{2}$\tabularnewline
\noalign{\vskip4mm}

Докажите, что последовательность сходится:\tabularnewline
$x_n = \sum_{k = 1}^n \frac{(-1)^{k - 1}}{k(k + 1)}$\tabularnewline
\noalign{\vskip4mm}

Вычислите предел функции:\tabularnewline
$\lim\limits_{x \to 0} \left(\frac{\ch 2x}{\ch x}\right)^\frac{1}{x^2}$\tabularnewline
\noalign{\vskip4mm}

\end{tabular}& %
\begin{tabular}{l}
Вариант Саргерас \tabularnewline
Постройте график в декартовых координатах:\tabularnewline
$y = \log_3\frac{x + 2}{x}$\tabularnewline
\noalign{\vskip4mm}

Докажите предел по определению:\tabularnewline
$\lim\limits_{n \to +\infty} \left(\frac{n + 1}{2n}\right)^n = 0$\tabularnewline
\noalign{\vskip4mm}

Докажите, что последовательность расходится:\tabularnewline
$x_n = \left[\frac{n^2 + 1}{3}\right] - \frac{n^2}{3}$\tabularnewline
\noalign{\vskip4mm}

Вычислите предел функции:\tabularnewline
$\lim\limits_{x \to \pi} \frac{1 - \cos x \cos 2x \cos 3x}{1 + \cos x}$\tabularnewline
\noalign{\vskip4mm}

\end{tabular}\tabularnewline
\noalign{\vskip4mm}
\begin{tabular}{l}
Вариант Сильвана \tabularnewline
Постройте график в декартовых координатах:\tabularnewline
$y = \log_3\frac{x + 2}{x}$\tabularnewline
\noalign{\vskip4mm}

Докажите предел по определению:\tabularnewline
$\lim\limits_{n \to +\infty} \frac{\sqrt[3]{n^2 + n}}{n + 2} = 0$\tabularnewline
\noalign{\vskip4mm}

Докажите, что последовательность расходится:\tabularnewline
$x_n = \left[\frac{n^2 + 1}{3}\right] - \frac{n^2}{3}$\tabularnewline
\noalign{\vskip4mm}

Вычислите предел функции:\tabularnewline
$\lim\limits_{x \to \pi} \frac{1 - \cos x \cos 2x \cos 3x}{1 + \cos x}$\tabularnewline
\noalign{\vskip4mm}

\end{tabular}& %
\begin{tabular}{l}
Вариант Тикондрус \tabularnewline
Постройте график в полярных координатах:\tabularnewline
$r = \frac{1}{1 - \sin\phi}$\tabularnewline
\noalign{\vskip4mm}

Докажите предел по определению:\tabularnewline
$\lim\limits_{n \to +\infty} \left(\frac{n + 1}{2n}\right)^n = 0$\tabularnewline
\noalign{\vskip4mm}

Докажите, что последовательность расходится:\tabularnewline
$x_n = \left[\frac{n^2 + 1}{3}\right] - \frac{n^2}{3}$\tabularnewline
\noalign{\vskip4mm}

Вычислите предел функции:\tabularnewline
$\lim\limits_{x \to \pi} \frac{1 - \cos x \cos 2x \cos 3x}{1 + \cos x}$\tabularnewline
\noalign{\vskip4mm}

\end{tabular}\tabularnewline
\noalign{\vskip4mm}
\begin{tabular}{l}
Вариант Тралл \tabularnewline
Постройте график в декартовых координатах:\tabularnewline
$y = \log_3\frac{x + 2}{x}$\tabularnewline
\noalign{\vskip4mm}

Докажите предел по определению:\tabularnewline
$\lim\limits_{n \to +\infty} \left(\frac{n + 1}{2n}\right)^n = 0$\tabularnewline
\noalign{\vskip4mm}

Докажите, что последовательность сходится:\tabularnewline
$x_n = \sum_{k = 1}^n \frac{(-1)^{k - 1}}{k(k + 1)}$\tabularnewline
\noalign{\vskip4mm}

Вычислите предел функции:\tabularnewline
$\lim\limits_{x \to 0} \left(\frac{\ch 2x}{\ch x}\right)^\frac{1}{x^2}$\tabularnewline
\noalign{\vskip4mm}

\end{tabular}& %
\begin{tabular}{l}
Вариант Фарион \tabularnewline
Постройте график в полярных координатах:\tabularnewline
$r = \frac{1}{1 - \sin\phi}$\tabularnewline
\noalign{\vskip4mm}

Докажите предел по определению:\tabularnewline
$\lim\limits_{n \to +\infty} \left(8 - \frac{1}{n^2}\right)^{-\frac{1}{3}} = \frac{1}{2}$\tabularnewline
\noalign{\vskip4mm}

Докажите, что последовательность сходится:\tabularnewline
$x_n = \sum_{k = 1}^n \frac{(-1)^{k - 1}}{k(k + 1)}$\tabularnewline
\noalign{\vskip4mm}

Вычислите предел функции:\tabularnewline
$\lim\limits_{x \to 0} \left(\frac{\ch 2x}{\ch x}\right)^\frac{1}{x^2}$\tabularnewline
\noalign{\vskip4mm}

\end{tabular}\tabularnewline
\noalign{\vskip4mm}
\end{tabular}



 
\end{document}
