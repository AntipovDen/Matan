
\documentclass[russian]{article}
\usepackage[T2A,T1]{fontenc}
\usepackage[utf8]{inputenc}
\usepackage[a4paper]{geometry}
\geometry{verbose,tmargin=1cm,bmargin=0cm,lmargin=0cm,rmargin=0cm}
\usepackage{amsmath}

\makeatletter

\DeclareRobustCommand{\cyrtext}{%
  \fontencoding{T2A}\selectfont\def\encodingdefault{T2A}}
\DeclareRobustCommand{\textcyr}[1]{\leavevmode{\cyrtext #1}}
\AtBeginDocument{\DeclareFontEncoding{T2A}{}{}}


\makeatother

\usepackage[russian]{babel}
\begin{document}
 
\begin{tabular}{cc}
\begin{tabular}{l}
Вариант Кэсси Кейдж \tabularnewline
Постройте график в декартовых координатах:\tabularnewline
$y = \sqrt{1 + x^2}$\tabularnewline
\noalign{\vskip4mm}

Докажите предел по определению:\tabularnewline
$\lim\limits_{n \to +\infty} \frac{n + 1}{\sqrt{n^2 + 1}} = 1$\tabularnewline
\noalign{\vskip4mm}

Докажите, что последовательность сходится:\tabularnewline
$x_n = \sum_{k = 1}^n \frac{\sin k \alpha}{2^k}$\tabularnewline
\noalign{\vskip4mm}

Вычислите предел функции:\tabularnewline
$\lim\limits_{x \to 0} \frac{\ln\cos 5x}{\ln\cos 4x}$\tabularnewline
\noalign{\vskip4mm}

\end{tabular}& %
\begin{tabular}{l}
Вариант Такеда Такахаши \tabularnewline
Постройте график в полярных координатах:\tabularnewline
$r = 8\sin(\phi - \frac{\pi}{3})$\tabularnewline
\noalign{\vskip4mm}

Докажите предел по определению:\tabularnewline
$\lim\limits_{n \to +\infty} (\sqrt[3]{n^3 + 2n} - n) = 0$\tabularnewline
\noalign{\vskip4mm}

Докажите, что последовательность расходится:\tabularnewline
$x_n = \frac{n \cos\pi n - 1}{2n}$\tabularnewline
\noalign{\vskip4mm}

Вычислите предел функции:\tabularnewline
$\lim\limits_{x \to 0} \frac{\sqrt[3]{x + 8} - 2}{\sqrt{1 + 2x} - 1}$\tabularnewline
\noalign{\vskip4mm}

\end{tabular}\tabularnewline
\noalign{\vskip4mm}
\begin{tabular}{l}
Вариант Джэки Бриггс \tabularnewline
Постройте график в полярных координатах:\tabularnewline
$r = 8\sin(\phi - \frac{\pi}{3})$\tabularnewline
\noalign{\vskip4mm}

Докажите предел по определению:\tabularnewline
$\lim\limits_{n \to +\infty} (\sqrt[3]{n^3 + 2n} - n) = 0$\tabularnewline
\noalign{\vskip4mm}

Докажите, что последовательность расходится:\tabularnewline
$x_n = \frac{n \cos\pi n - 1}{2n}$\tabularnewline
\noalign{\vskip4mm}

Вычислите предел функции:\tabularnewline
$\lim\limits_{x \to 0} (\sqrt{1 + x} - x)^\frac{1}{x}$\tabularnewline
\noalign{\vskip4mm}

\end{tabular}& %
\begin{tabular}{l}
Вариант Кунг Джин \tabularnewline
Постройте график в декартовых координатах:\tabularnewline
$y = 2^\frac{x + 1}{x}$\tabularnewline
\noalign{\vskip4mm}

Докажите предел по определению:\tabularnewline
$\lim\limits_{n \to +\infty} \frac{n + 1}{\sqrt{n^2 + 1}} = 1$\tabularnewline
\noalign{\vskip4mm}

Докажите, что последовательность расходится:\tabularnewline
$x_n = \frac{n \cos\pi n - 1}{2n}$\tabularnewline
\noalign{\vskip4mm}

Вычислите предел функции:\tabularnewline
$\lim\limits_{x \to 0} \frac{\sqrt[3]{x + 8} - 2}{\sqrt{1 + 2x} - 1}$\tabularnewline
\noalign{\vskip4mm}

\end{tabular}\tabularnewline
\noalign{\vskip4mm}
\begin{tabular}{l}
Вариант Эррон Блэк \tabularnewline
Постройте график в декартовых координатах:\tabularnewline
$y = \sqrt{1 + x^2}$\tabularnewline
\noalign{\vskip4mm}

Докажите предел по определению:\tabularnewline
$\lim\limits_{n \to +\infty} (\sqrt[3]{n^3 + 2n} - n) = 0$\tabularnewline
\noalign{\vskip4mm}

Докажите, что последовательность расходится:\tabularnewline
$x_n = \frac{n \cos\pi n - 1}{2n}$\tabularnewline
\noalign{\vskip4mm}

Вычислите предел функции:\tabularnewline
$\lim\limits_{x \to 0} (\sqrt{1 + x} - x)^\frac{1}{x}$\tabularnewline
\noalign{\vskip4mm}

\end{tabular}& %
\begin{tabular}{l}
Вариант Коталь Кан \tabularnewline
Постройте график в полярных координатах:\tabularnewline
$r = 8\sin(\phi - \frac{\pi}{3})$\tabularnewline
\noalign{\vskip4mm}

Докажите предел по определению:\tabularnewline
$\lim\limits_{n \to +\infty} (\sqrt{n^2 + n} - n) = \frac{1}{2}$\tabularnewline
\noalign{\vskip4mm}

Докажите, что последовательность сходится:\tabularnewline
$x_n = 0.77..7 \text{($n$ семерок)}$\tabularnewline
\noalign{\vskip4mm}

Вычислите предел функции:\tabularnewline
$\lim\limits_{x \to 0} \frac{\ln\cos 5x}{\ln\cos 4x}$\tabularnewline
\noalign{\vskip4mm}

\end{tabular}\tabularnewline
\noalign{\vskip4mm}
\begin{tabular}{l}
Вариант Ди'Вора \tabularnewline
Постройте график в декартовых координатах:\tabularnewline
$y = \sqrt{1 + x^2}$\tabularnewline
\noalign{\vskip4mm}

Докажите предел по определению:\tabularnewline
$\lim\limits_{n \to +\infty} (\sqrt[3]{n^3 + 2n} - n) = 0$\tabularnewline
\noalign{\vskip4mm}

Докажите, что последовательность расходится:\tabularnewline
$x_n = \frac{n \cos\pi n - 1}{2n}$\tabularnewline
\noalign{\vskip4mm}

Вычислите предел функции:\tabularnewline
$\lim\limits_{x \to 0} \frac{\ln\cos 5x}{\ln\cos 4x}$\tabularnewline
\noalign{\vskip4mm}

\end{tabular}& %
\begin{tabular}{l}
Вариант Ферра и Торр \tabularnewline
Постройте график в полярных координатах:\tabularnewline
$r = 8\sin(\phi - \frac{\pi}{3})$\tabularnewline
\noalign{\vskip4mm}

Докажите предел по определению:\tabularnewline
$\lim\limits_{n \to +\infty} (\sqrt[3]{n^3 + 2n} - n) = 0$\tabularnewline
\noalign{\vskip4mm}

Докажите, что последовательность сходится:\tabularnewline
$x_n = 0.77..7 \text{($n$ семерок)}$\tabularnewline
\noalign{\vskip4mm}

Вычислите предел функции:\tabularnewline
$\lim\limits_{x \to 0} (\sqrt{1 + x} - x)^\frac{1}{x}$\tabularnewline
\noalign{\vskip4mm}

\end{tabular}\tabularnewline
\noalign{\vskip4mm}
\end{tabular}

\begin{tabular}{cc}
\begin{tabular}{l}
Вариант Кунг Лао \tabularnewline
Постройте график в декартовых координатах:\tabularnewline
$y = 2^\frac{x + 1}{x}$\tabularnewline
\noalign{\vskip4mm}

Докажите предел по определению:\tabularnewline
$\lim\limits_{n \to +\infty} (\sqrt{n^2 + n} - n) = \frac{1}{2}$\tabularnewline
\noalign{\vskip4mm}

Докажите, что последовательность сходится:\tabularnewline
$x_n = 0.77..7 \text{($n$ семерок)}$\tabularnewline
\noalign{\vskip4mm}

Вычислите предел функции:\tabularnewline
$\lim\limits_{x \to 0} \frac{\sqrt[3]{x + 8} - 2}{\sqrt{1 + 2x} - 1}$\tabularnewline
\noalign{\vskip4mm}

\end{tabular}& %
\begin{tabular}{l}
Вариант Джакс \tabularnewline
Постройте график в полярных координатах:\tabularnewline
$r = 8\sin(\phi - \frac{\pi}{3})$\tabularnewline
\noalign{\vskip4mm}

Докажите предел по определению:\tabularnewline
$\lim\limits_{n \to +\infty} \frac{n + 1}{\sqrt{n^2 + 1}} = 1$\tabularnewline
\noalign{\vskip4mm}

Докажите, что последовательность сходится:\tabularnewline
$x_n = \sum_{k = 1}^n \frac{\sin k \alpha}{2^k}$\tabularnewline
\noalign{\vskip4mm}

Вычислите предел функции:\tabularnewline
$\lim\limits_{x \to 0} \frac{\ln\cos 5x}{\ln\cos 4x}$\tabularnewline
\noalign{\vskip4mm}

\end{tabular}\tabularnewline
\noalign{\vskip4mm}
\begin{tabular}{l}
Вариант Соня Блейд \tabularnewline
Постройте график в декартовых координатах:\tabularnewline
$y = \sqrt{1 + x^2}$\tabularnewline
\noalign{\vskip4mm}

Докажите предел по определению:\tabularnewline
$\lim\limits_{n \to +\infty} (\sqrt{n^2 + n} - n) = \frac{1}{2}$\tabularnewline
\noalign{\vskip4mm}

Докажите, что последовательность сходится:\tabularnewline
$x_n = \sum_{k = 1}^n \frac{\sin k \alpha}{2^k}$\tabularnewline
\noalign{\vskip4mm}

Вычислите предел функции:\tabularnewline
$\lim\limits_{x \to 0} \frac{\ln\cos 5x}{\ln\cos 4x}$\tabularnewline
\noalign{\vskip4mm}

\end{tabular}& %
\begin{tabular}{l}
Вариант Кенши \tabularnewline
Постройте график в декартовых координатах:\tabularnewline
$y = \sqrt{1 + x^2}$\tabularnewline
\noalign{\vskip4mm}

Докажите предел по определению:\tabularnewline
$\lim\limits_{n \to +\infty} \frac{n + 1}{\sqrt{n^2 + 1}} = 1$\tabularnewline
\noalign{\vskip4mm}

Докажите, что последовательность сходится:\tabularnewline
$x_n = \sum_{k = 1}^n \frac{\sin k \alpha}{2^k}$\tabularnewline
\noalign{\vskip4mm}

Вычислите предел функции:\tabularnewline
$\lim\limits_{x \to 0} (\sqrt{1 + x} - x)^\frac{1}{x}$\tabularnewline
\noalign{\vskip4mm}

\end{tabular}\tabularnewline
\noalign{\vskip4mm}
\begin{tabular}{l}
Вариант Китана \tabularnewline
Постройте график в полярных координатах:\tabularnewline
$r = 8\sin(\phi - \frac{\pi}{3})$\tabularnewline
\noalign{\vskip4mm}

Докажите предел по определению:\tabularnewline
$\lim\limits_{n \to +\infty} \frac{n + 1}{\sqrt{n^2 + 1}} = 1$\tabularnewline
\noalign{\vskip4mm}

Докажите, что последовательность расходится:\tabularnewline
$x_n = \frac{n \cos\pi n - 1}{2n}$\tabularnewline
\noalign{\vskip4mm}

Вычислите предел функции:\tabularnewline
$\lim\limits_{x \to 0} (\sqrt{1 + x} - x)^\frac{1}{x}$\tabularnewline
\noalign{\vskip4mm}

\end{tabular}& %
\begin{tabular}{l}
Вариант Скорпион \tabularnewline
Постройте график в декартовых координатах:\tabularnewline
$y = \sqrt{1 + x^2}$\tabularnewline
\noalign{\vskip4mm}

Докажите предел по определению:\tabularnewline
$\lim\limits_{n \to +\infty} \frac{n + 1}{\sqrt{n^2 + 1}} = 1$\tabularnewline
\noalign{\vskip4mm}

Докажите, что последовательность сходится:\tabularnewline
$x_n = 0.77..7 \text{($n$ семерок)}$\tabularnewline
\noalign{\vskip4mm}

Вычислите предел функции:\tabularnewline
$\lim\limits_{x \to 0} \frac{\sqrt[3]{x + 8} - 2}{\sqrt{1 + 2x} - 1}$\tabularnewline
\noalign{\vskip4mm}

\end{tabular}\tabularnewline
\noalign{\vskip4mm}
\begin{tabular}{l}
Вариант Саб-Зиро \tabularnewline
Постройте график в декартовых координатах:\tabularnewline
$y = \sqrt{1 + x^2}$\tabularnewline
\noalign{\vskip4mm}

Докажите предел по определению:\tabularnewline
$\lim\limits_{n \to +\infty} \frac{n + 1}{\sqrt{n^2 + 1}} = 1$\tabularnewline
\noalign{\vskip4mm}

Докажите, что последовательность сходится:\tabularnewline
$x_n = 0.77..7 \text{($n$ семерок)}$\tabularnewline
\noalign{\vskip4mm}

Вычислите предел функции:\tabularnewline
$\lim\limits_{x \to 0} (\sqrt{1 + x} - x)^\frac{1}{x}$\tabularnewline
\noalign{\vskip4mm}

\end{tabular}& %
\begin{tabular}{l}
Вариант Милина \tabularnewline
Постройте график в декартовых координатах:\tabularnewline
$y = \sqrt{1 + x^2}$\tabularnewline
\noalign{\vskip4mm}

Докажите предел по определению:\tabularnewline
$\lim\limits_{n \to +\infty} (\sqrt{n^2 + n} - n) = \frac{1}{2}$\tabularnewline
\noalign{\vskip4mm}

Докажите, что последовательность расходится:\tabularnewline
$x_n = \frac{n \cos\pi n - 1}{2n}$\tabularnewline
\noalign{\vskip4mm}

Вычислите предел функции:\tabularnewline
$\lim\limits_{x \to 0} \frac{\sqrt[3]{x + 8} - 2}{\sqrt{1 + 2x} - 1}$\tabularnewline
\noalign{\vskip4mm}

\end{tabular}\tabularnewline
\noalign{\vskip4mm}
\end{tabular}

\begin{tabular}{cc}
\begin{tabular}{l}
Вариант Кано \tabularnewline
Постройте график в декартовых координатах:\tabularnewline
$y = 2^\frac{x + 1}{x}$\tabularnewline
\noalign{\vskip4mm}

Докажите предел по определению:\tabularnewline
$\lim\limits_{n \to +\infty} \frac{n + 1}{\sqrt{n^2 + 1}} = 1$\tabularnewline
\noalign{\vskip4mm}

Докажите, что последовательность расходится:\tabularnewline
$x_n = \frac{n \cos\pi n - 1}{2n}$\tabularnewline
\noalign{\vskip4mm}

Вычислите предел функции:\tabularnewline
$\lim\limits_{x \to 0} \frac{\sqrt[3]{x + 8} - 2}{\sqrt{1 + 2x} - 1}$\tabularnewline
\noalign{\vskip4mm}

\end{tabular}& %
\begin{tabular}{l}
Вариант Джонни Кейдж \tabularnewline
Постройте график в декартовых координатах:\tabularnewline
$y = \sqrt{1 + x^2}$\tabularnewline
\noalign{\vskip4mm}

Докажите предел по определению:\tabularnewline
$\lim\limits_{n \to +\infty} (\sqrt[3]{n^3 + 2n} - n) = 0$\tabularnewline
\noalign{\vskip4mm}

Докажите, что последовательность сходится:\tabularnewline
$x_n = \sum_{k = 1}^n \frac{\sin k \alpha}{2^k}$\tabularnewline
\noalign{\vskip4mm}

Вычислите предел функции:\tabularnewline
$\lim\limits_{x \to 0} (\sqrt{1 + x} - x)^\frac{1}{x}$\tabularnewline
\noalign{\vskip4mm}

\end{tabular}\tabularnewline
\noalign{\vskip4mm}
\begin{tabular}{l}
Вариант Лю Кенг \tabularnewline
Постройте график в декартовых координатах:\tabularnewline
$y = \sqrt{1 + x^2}$\tabularnewline
\noalign{\vskip4mm}

Докажите предел по определению:\tabularnewline
$\lim\limits_{n \to +\infty} \frac{n + 1}{\sqrt{n^2 + 1}} = 1$\tabularnewline
\noalign{\vskip4mm}

Докажите, что последовательность сходится:\tabularnewline
$x_n = \sum_{k = 1}^n \frac{\sin k \alpha}{2^k}$\tabularnewline
\noalign{\vskip4mm}

Вычислите предел функции:\tabularnewline
$\lim\limits_{x \to 0} \frac{\ln\cos 5x}{\ln\cos 4x}$\tabularnewline
\noalign{\vskip4mm}

\end{tabular}& %
\begin{tabular}{l}
Вариант Ермак \tabularnewline
Постройте график в декартовых координатах:\tabularnewline
$y = \sqrt{1 + x^2}$\tabularnewline
\noalign{\vskip4mm}

Докажите предел по определению:\tabularnewline
$\lim\limits_{n \to +\infty} (\sqrt{n^2 + n} - n) = \frac{1}{2}$\tabularnewline
\noalign{\vskip4mm}

Докажите, что последовательность сходится:\tabularnewline
$x_n = 0.77..7 \text{($n$ семерок)}$\tabularnewline
\noalign{\vskip4mm}

Вычислите предел функции:\tabularnewline
$\lim\limits_{x \to 0} \frac{\ln\cos 5x}{\ln\cos 4x}$\tabularnewline
\noalign{\vskip4mm}

\end{tabular}\tabularnewline
\noalign{\vskip4mm}
\begin{tabular}{l}
Вариант Рептилия \tabularnewline
Постройте график в декартовых координатах:\tabularnewline
$y = 2^\frac{x + 1}{x}$\tabularnewline
\noalign{\vskip4mm}

Докажите предел по определению:\tabularnewline
$\lim\limits_{n \to +\infty} \frac{n + 1}{\sqrt{n^2 + 1}} = 1$\tabularnewline
\noalign{\vskip4mm}

Докажите, что последовательность расходится:\tabularnewline
$x_n = \frac{n \cos\pi n - 1}{2n}$\tabularnewline
\noalign{\vskip4mm}

Вычислите предел функции:\tabularnewline
$\lim\limits_{x \to 0} \frac{\ln\cos 5x}{\ln\cos 4x}$\tabularnewline
\noalign{\vskip4mm}

\end{tabular}& %
\begin{tabular}{l}
Вариант Рейден \tabularnewline
Постройте график в декартовых координатах:\tabularnewline
$y = \sqrt{1 + x^2}$\tabularnewline
\noalign{\vskip4mm}

Докажите предел по определению:\tabularnewline
$\lim\limits_{n \to +\infty} (\sqrt[3]{n^3 + 2n} - n) = 0$\tabularnewline
\noalign{\vskip4mm}

Докажите, что последовательность расходится:\tabularnewline
$x_n = \frac{n \cos\pi n - 1}{2n}$\tabularnewline
\noalign{\vskip4mm}

Вычислите предел функции:\tabularnewline
$\lim\limits_{x \to 0} (\sqrt{1 + x} - x)^\frac{1}{x}$\tabularnewline
\noalign{\vskip4mm}

\end{tabular}\tabularnewline
\noalign{\vskip4mm}
\begin{tabular}{l}
Вариант Куан Чи \tabularnewline
Постройте график в декартовых координатах:\tabularnewline
$y = 2^\frac{x + 1}{x}$\tabularnewline
\noalign{\vskip4mm}

Докажите предел по определению:\tabularnewline
$\lim\limits_{n \to +\infty} \frac{n + 1}{\sqrt{n^2 + 1}} = 1$\tabularnewline
\noalign{\vskip4mm}

Докажите, что последовательность расходится:\tabularnewline
$x_n = \frac{n \cos\pi n - 1}{2n}$\tabularnewline
\noalign{\vskip4mm}

Вычислите предел функции:\tabularnewline
$\lim\limits_{x \to 0} \frac{\ln\cos 5x}{\ln\cos 4x}$\tabularnewline
\noalign{\vskip4mm}

\end{tabular}& %
\begin{tabular}{l}
Вариант  \tabularnewline
Постройте график в полярных координатах:\tabularnewline
$r = 8\sin(\phi - \frac{\pi}{3})$\tabularnewline
\noalign{\vskip4mm}

Докажите предел по определению:\tabularnewline
$\lim\limits_{n \to +\infty} (\sqrt{n^2 + n} - n) = \frac{1}{2}$\tabularnewline
\noalign{\vskip4mm}

Докажите, что последовательность расходится:\tabularnewline
$x_n = \frac{n \cos\pi n - 1}{2n}$\tabularnewline
\noalign{\vskip4mm}

Вычислите предел функции:\tabularnewline
$\lim\limits_{x \to 0} \frac{\sqrt[3]{x + 8} - 2}{\sqrt{1 + 2x} - 1}$\tabularnewline
\noalign{\vskip4mm}

\end{tabular}\tabularnewline
\noalign{\vskip4mm}
\end{tabular}


 
\end{document}
