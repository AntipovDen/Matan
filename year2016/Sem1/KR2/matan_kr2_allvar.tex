\documentclass[russian]{article}
\usepackage[T2A,T1]{fontenc}
\usepackage[utf8]{inputenc}
\usepackage[a4paper]{geometry}
\geometry{verbose,tmargin=1cm,bmargin=0cm,lmargin=0cm,rmargin=0cm}

\makeatletter

%%%%%%%%%%%%%%%%%%%%%%%%%%%%%% LyX specific LaTeX commands.
\DeclareRobustCommand{\cyrtext}{%
  \fontencoding{T2A}\selectfont\def\encodingdefault{T2A}}
\DeclareRobustCommand{\textcyr}[1]{\leavevmode{\cyrtext #1}}
\AtBeginDocument{\DeclareFontEncoding{T2A}{}{}}


\makeatother

\usepackage[russian]{babel}
\begin{document}
 
 %36-37
 Посчитайте производную:
 $y = \arcsin \frac{\sin 2017 \sin x}{1 - \cos 2017 \cos x}$
 
 Посчитайте производную:
 $y = \arctg e^\frac{x}{2} - \ln \sqrt{\frac{e^x}{e^x + 1}}$
 
 Посчитайте производную:
 $y = |\sin x|^{\cos x}$
 
 Посчитайте 2017-ю производную:
 $y = x\ln\frac{3 + x}{3 - x}$
 
 Посчитайте 30-ю производную:
 $(x^2 + x) \cos^2 x$
 
 
 Разложите в ряд Тейлора до $o(x^5)$:
 $y = \frac{1}{1 - \ln^2(1 + x)}$
 
 Разложите в ряд Тейлора до $o(x^n)$:
 $y = \ln\sqrt[3]{\frac{2 + x^2}{x^4 - 3x^2 + 2}}$
 
 Разложите в ряд Тейлора до $o((x - 3)^n)$:
 $y = \ln\sqrt[4]{\frac{x - 2}{5 - x}}$
 
 Разложите в ряд Тейлора до $o(x^n)$:
 $y = \cos^4 x + \sin^4 x$
 
 
 Посчитайте предел с помощью рядов Тейлора:
 $\lim\limits_{x \to 0} \frac{\sqrt{1 + \sin x} - \frac{1}{2} \tg x + \frac{x^2}{8} - 1}{e^x - \sqrt{1 + 2x} - x^2}$
 
 Посчитайте предел с помощью рядов Тейлора:
 $\lim\limits_{x \to 0} \frac{\sqrt{2x + \cos 2x} - e^{\tg x} + 2x^2}{2 \sin x - 2 \ln(1 + x) - x^2}$
 
 Посчитайте предел с помощью рядов Тейлора:
 $\lim\limits_{x \to 0} \left(\frac{\ln(\sqrt{1 + x^2} + x)}{x}\right)^\frac{1}{x^2}$
 
 Посчитайте предел с помощью рядов Тейлора:
 $\lim\limits_{x \to 0} \left( e^{\frac{1}{3}\sin x} + \sqrt[3]{1 - \tg x} - 1 \right)^\frac{1}{\ln(1 + x^2)}$
 
 
 Посчитайте предел с помощью правила Лопиталя:
 $\lim\limits_{x \to 0} \sin x \ln \ctg x$
 
 Посчитайте предел с помощью правила Лопиталя:
 $\lim\limits_{x \to 0} (\cos x)^\frac{1}{x^2}$
 
 Посчитайте предел с помощью правила Лопиталя:
 $\lim\limits_{x \to 0} \frac{(x + 1) \ln (1 + x) - x}{e^x - x - 1}$
 
 Посчитайте предел с помощью правила Лопиталя:
 $\lim\limits_{x \to 1} \frac{x^{10} - 10 x + 9}{(x - 1)^2}$
 
 
 %38-39
 Посчитайте производную:
 $y = \frac{5x + 2}{x^2 + x + 1} + \ln\sqrt[3]{\frac{(x - 1)^2}{x^2 + x + 1}} + \frac{8}{\sqrt{3}}\arctg\frac{2x + 1}{\sqrt{3}}$
 
 Посчитайте производную:
 $y = \frac{3 - \sin x}{2} \sqrt{\cos^2 x - 2 \sin x} + 2 \arcsin \frac{1 + \sin x}{\sqrt{2}}$
 
 Посчитайте производную:
 $y = \left( \ln(\sqrt{x^2 + 4} - \sqrt{x^2 - 4}) \right)^{\tg(x^2 + \ln 2x)}$
 
 Посчитайте 666-ю производную:
 $y = \frac{1 + x^2}{1 - x^2}$
 
 Посчитайте 4-ю производную:
 $y = \ln(1 + x^2)$
 
 
 Разложите в ряд Тейлора до $o(x^5)$:
 $y = (1 + x)^{\sin x}$
 
 Разложите в ряд Тейлора до $o(x^n)$:
 $y = \ln(x + \sqrt{x^2 + 1})$
 
 Разложите в ряд Тейлора до $o(x^n)$:
 $y = \ln\sqrt{\frac{e - x^3}{1 - ex^3}}$
 
 Разложите в ряд Тейлора до $o(x^4)$:
 $y = \frac{x}{e^x - 1}$
 
 
 Посчитайте предел с помощью рядов Тейлора:
 $\lim\limits_{x \to 0} \frac{e^{\cos x} - e\sqrt[3]{1 - 4x^2}}{\frac{\arcsin 2x}{x} - 2\cos (x^2)}$
 
 Посчитайте предел с помощью рядов Тейлора:
 $\lim\limits_{x \to 0} \frac{\sqrt{1 - 2x} - e^{-x} + x^2\sqrt[3]{1 + x}}{\sin^2 x - \ln \ch^2 x}$
 
 Посчитайте предел с помощью рядов Тейлора:
 $\lim\limits_{x \to 0} \left( \frac{1}{e} (1 + x)^\frac{1}{x} + \frac{2x}{4 + 5x} \right)^{\ctg^2 x}$
 
 Посчитайте предел с помощью рядов Тейлора:
 $\lim\limits_{x \to 0} \left( \frac{\sin (2x + x^3) - \sh(x + 2x^3)}{x} \right)^\frac{1}{2\ln(1 + x^2) - \ln^2(1 + x)}$
 
 
 
 
 Посчитайте предел с помощью правила Лопиталя:
 $\lim\limits_{x \to 0} \frac{2\tg 3x - 6 \tg x}{3 \arctg x - \arctg 3x}$
 
 Посчитайте предел с помощью правила Лопиталя:
 $\lim\limits_{x \to -1} \frac{2x^4 + 3x^3 - 4x^2 - 9x - 4}{3x^4 + 5x^3 +3x^2 + 3x + 2}$
 
 Посчитайте предел с помощью правила Лопиталя:
 $\lim\limits_{x \to 0} \frac{e^{\sin x} - e^x}{\sin x - x}$
 
 Посчитайте предел с помощью правила Лопиталя:
 $\lim\limits_{x \to +\infty} \frac{\sqrt[3]{x} \ln\ln x}{\sqrt[3]{2x + 3}\sqrt{\ln x}}$ 

 
\end{document}
