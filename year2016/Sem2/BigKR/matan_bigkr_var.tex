\documentclass[russian]{article}
\usepackage[T2A,T1]{fontenc}
\usepackage[utf8]{inputenc}
\usepackage[a4paper]{geometry}
\geometry{verbose,tmargin=1cm,bmargin=0cm,lmargin=0cm,rmargin=0cm}
\usepackage{amsmath}
\usepackage{amsfonts}

\makeatletter

\DeclareRobustCommand{\cyrtext}{%
  \fontencoding{T2A}\selectfont\def\encodingdefault{T2A}}
\DeclareRobustCommand{\textcyr}[1]{\leavevmode{\cyrtext #1}}
\AtBeginDocument{\DeclareFontEncoding{T2A}{}{}}

\newcommand{\dx}{\text{d}x}
\makeatother

\usepackage[russian]{babel}
\begin{document}

\begin{tabular}{cc}
\begin{tabular}{l}
Вариант Фанатик \tabularnewline
1) Найдите интеграл:\tabularnewline
$\int \frac{x \ln x \dx}{\sqrt{1 + x^2}}$\tabularnewline
\noalign{\vskip4mm}

2) Найдите интеграл:\tabularnewline
$\int \frac{\dx}{(x^2 + 1)^2}$\tabularnewline
\noalign{\vskip4mm}

3) Вычислите площадь фигуры, ограниченной кривыми:\tabularnewline
$r = 2 |\tg \phi|; r = \frac{1}{\cos \phi}$\tabularnewline
\noalign{\vskip4mm}

4) Вычислите длину кривой:\tabularnewline
$r = \cos^3 \frac{\phi}{3}$\tabularnewline
\noalign{\vskip4mm}

5) Определите сходимость интеграла:\tabularnewline
$\int\limits_{0}^{1} \frac{\sin(\arcsin + x^3) - x}{\sqrt{\sin^7 x}} \dx$\tabularnewline
\noalign{\vskip4mm}

6) Определите сходимость интеграла:\tabularnewline
$\int\limits_{1}^{+\infty} \frac{\ln{x}\dx}{x \sqrt{x^2 - 1}}$\tabularnewline
\noalign{\vskip4mm}

7) Вычислите сумму ряда:\tabularnewline
$\sum\limits_{n = 1}^{+\infty} \left(\sin \frac{1}{n(n + 1)} \right) / \left(\cos\left(\frac{1}{n(n + 1)}\right) + \cos\left(\frac{2n + 1}{n(n + 1)}\right)\right)$\tabularnewline
\noalign{\vskip4mm}

8) Определите сходимость ряда:\tabularnewline
$\sum\limits_{n=1}^{+\infty} \frac{(2n)!!}{n!} \arctan\frac{1}{3^n}$\tabularnewline
\noalign{\vskip4mm}

\end{tabular}& %
\begin{tabular}{l}
Вариант Чемпион \tabularnewline
1) Найдите интеграл:\tabularnewline
$\int \frac{\dx}{\sin x ( 1 + \cos x)}$\tabularnewline
\noalign{\vskip4mm}

2) Найдите интеграл:\tabularnewline
$\int \frac{\dx}{(x^2 + 1) \sqrt{x^2 + 9}}$\tabularnewline
\noalign{\vskip4mm}

3) Вычислите площадь фигуры, ограниченной кривыми:\tabularnewline
$r = 2 |\tg \phi|; r = \frac{1}{\cos \phi}$\tabularnewline
\noalign{\vskip4mm}

4) Вычислите длину кривой:\tabularnewline
$r = \cos^3 \frac{\phi}{3}$\tabularnewline
\noalign{\vskip4mm}

5) Определите сходимость интеграла:\tabularnewline
$\int\limits_{0}^{1} \frac{\sin(\arcsin + x^3) - x}{\sqrt{\sin^7 x}} \dx$\tabularnewline
\noalign{\vskip4mm}

6) Определите сходимость интеграла:\tabularnewline
$\int\limits_{0}^{+\infty} \sin^3 (x^2 + 2x) \dx$\tabularnewline
\noalign{\vskip4mm}

7) Вычислите сумму ряда:\tabularnewline
$\sum\limits_{n = 1}^{+\infty} \left(2n\sin\frac{1}{2n(n + 1)}\cos\frac{2n + 1}{2n(n + 1)} - \sin\frac{1}{n + 1}\right)$\tabularnewline
\noalign{\vskip4mm}

8) Определите сходимость ряда:\tabularnewline
$\sum\limits_{n=1}^{+\infty} \frac{(2n)!!}{n!} \arctan\frac{1}{3^n}$\tabularnewline
\noalign{\vskip4mm}

\end{tabular}\tabularnewline
\noalign{\vskip4mm}
\begin{tabular}{l}
Вариант Архангел \tabularnewline
1) Найдите интеграл:\tabularnewline
$\int \frac{\dx}{\sin x ( 1 + \cos x)}$\tabularnewline
\noalign{\vskip4mm}

2) Найдите интеграл:\tabularnewline
$\int \frac{\dx}{(x^2 + 1)^2}$\tabularnewline
\noalign{\vskip4mm}

3) Вычислите площадь фигуры, ограниченной кривыми:\tabularnewline
$y = \frac{10}{x^2 + 4}; y = \frac{x^2 + 5x + 4}{x^2 + 4}$\tabularnewline
\noalign{\vskip4mm}

4) Найдите объем фигуры, полученной вращением вокруг\tabularnewline
оси OY плоской фигуры, ограниченной кривыми:\tabularnewline
$y = e^{x^2}; y = 0; x = 0; x = 1$\tabularnewline
\noalign{\vskip4mm}

5) Определите сходимость интеграла:\tabularnewline
$\int\limits_{0}^{1} \frac{\sin(\arcsin + x^3) - x}{\sqrt{\sin^7 x}} \dx$\tabularnewline
\noalign{\vskip4mm}

6) Определите сходимость интеграла:\tabularnewline
$\int\limits_{0}^{+\infty} \sin^3 (x^2 + 2x) \dx$\tabularnewline
\noalign{\vskip4mm}

7) Вычислите сумму ряда:\tabularnewline
$\sum\limits_{n = 1}^{+\infty} \left(\sin \frac{1}{n(n + 1)} \right) / \left(\cos\left(\frac{1}{n(n + 1)}\right) + \cos\left(\frac{2n + 1}{n(n + 1)}\right)\right)$\tabularnewline
\noalign{\vskip4mm}

8) Определите сходимость ряда:\tabularnewline
$\sum\limits_{n=1}^{+\infty} \frac{(2n)!!}{n!} \arctan\frac{1}{3^n}$\tabularnewline
\noalign{\vskip4mm}

\end{tabular}& %
\begin{tabular}{l}
Вариант Дендроид солдат \tabularnewline
1) Найдите интеграл:\tabularnewline
$\int \frac{\dx}{\sin x ( 1 + \cos x)}$\tabularnewline
\noalign{\vskip4mm}

2) Найдите интеграл:\tabularnewline
$\int \frac{\dx}{(x^2 + 1) \sqrt{x^2 + 9}}$\tabularnewline
\noalign{\vskip4mm}

3) Вычислите площадь фигуры, ограниченной кривыми:\tabularnewline
$r = 2 |\tg \phi|; r = \frac{1}{\cos \phi}$\tabularnewline
\noalign{\vskip4mm}

4) Найдите объем фигуры, полученной вращением вокруг\tabularnewline
оси OY плоской фигуры, ограниченной кривыми:\tabularnewline
$y = e^{x^2}; y = 0; x = 0; x = 1$\tabularnewline
\noalign{\vskip4mm}

5) Определите сходимость интеграла:\tabularnewline
$\int\limits_{0}^{1} \frac{\sin(\arcsin + x^3) - x}{\sqrt{\sin^7 x}} \dx$\tabularnewline
\noalign{\vskip4mm}

6) Определите сходимость интеграла:\tabularnewline
$\int\limits_{0}^{+\infty} \sin^3 (x^2 + 2x) \dx$\tabularnewline
\noalign{\vskip4mm}

7) Вычислите сумму ряда:\tabularnewline
$\sum\limits_{n = 1}^{+\infty} \left(\sin \frac{1}{n(n + 1)} \right) / \left(\cos\left(\frac{1}{n(n + 1)}\right) + \cos\left(\frac{2n + 1}{n(n + 1)}\right)\right)$\tabularnewline
\noalign{\vskip4mm}

8) Определите сходимость ряда:\tabularnewline
$\sum\limits_{n=1}^{+\infty} \frac{(2n)!!}{n!} \arctan\frac{1}{3^n}$\tabularnewline
\noalign{\vskip4mm}

\end{tabular}\tabularnewline
\noalign{\vskip4mm}
\end{tabular}

\begin{tabular}{cc}
\begin{tabular}{l}
Вариант Боевой единорог \tabularnewline
1) Найдите интеграл:\tabularnewline
$\int \frac{x \ln x \dx}{\sqrt{1 + x^2}}$\tabularnewline
\noalign{\vskip4mm}

2) Найдите интеграл:\tabularnewline
$\int \frac{\dx}{(x^2 + 1) \sqrt{x^2 + 9}}$\tabularnewline
\noalign{\vskip4mm}

3) Вычислите площадь фигуры, ограниченной кривыми:\tabularnewline
$r = 2 |\tg \phi|; r = \frac{1}{\cos \phi}$\tabularnewline
\noalign{\vskip4mm}

4) Вычислите длину кривой:\tabularnewline
$r = \cos^3 \frac{\phi}{3}$\tabularnewline
\noalign{\vskip4mm}

5) Определите сходимость интеграла:\tabularnewline
$\int\limits_{0}^{1} \frac{\sin(\arcsin + x^3) - x}{\sqrt{\sin^7 x}} \dx$\tabularnewline
\noalign{\vskip4mm}

6) Определите сходимость интеграла:\tabularnewline
$\int\limits_{0}^{+\infty} \sin^3 (x^2 + 2x) \dx$\tabularnewline
\noalign{\vskip4mm}

7) Вычислите сумму ряда:\tabularnewline
$\sum\limits_{n = 1}^{+\infty} \left(2n\sin\frac{1}{2n(n + 1)}\cos\frac{2n + 1}{2n(n + 1)} - \sin\frac{1}{n + 1}\right)$\tabularnewline
\noalign{\vskip4mm}

8) Определите сходимость ряда:\tabularnewline
$\sum\limits_{n=1}^{+\infty} \frac{\sin{n}}{n + \sin{n}}$\tabularnewline
\noalign{\vskip4mm}

\end{tabular}& %
\begin{tabular}{l}
Вариант Золотой дракон \tabularnewline
1) Найдите интеграл:\tabularnewline
$\int \frac{\dx}{\sin x ( 1 + \cos x)}$\tabularnewline
\noalign{\vskip4mm}

2) Найдите интеграл:\tabularnewline
$\int \frac{\dx}{(x^2 + 1) \sqrt{x^2 + 9}}$\tabularnewline
\noalign{\vskip4mm}

3) Вычислите площадь фигуры, ограниченной кривыми:\tabularnewline
$y = \frac{10}{x^2 + 4}; y = \frac{x^2 + 5x + 4}{x^2 + 4}$\tabularnewline
\noalign{\vskip4mm}

4) Найдите объем фигуры, полученной вращением вокруг\tabularnewline
оси OY плоской фигуры, ограниченной кривыми:\tabularnewline
$y = e^{x^2}; y = 0; x = 0; x = 1$\tabularnewline
\noalign{\vskip4mm}

5) Определите сходимость интеграла:\tabularnewline
$\int\limits_{0}^{2} \frac{\sqrt{x}\dx}{e^{\sin{x}} - 1}$\tabularnewline
\noalign{\vskip4mm}

6) Определите сходимость интеграла:\tabularnewline
$\int\limits_{1}^{+\infty} \frac{\ln{x}\dx}{x \sqrt{x^2 - 1}}$\tabularnewline
\noalign{\vskip4mm}

7) Вычислите сумму ряда:\tabularnewline
$\sum\limits_{n = 1}^{+\infty} \left(\sin \frac{1}{n(n + 1)} \right) / \left(\cos\left(\frac{1}{n(n + 1)}\right) + \cos\left(\frac{2n + 1}{n(n + 1)}\right)\right)$\tabularnewline
\noalign{\vskip4mm}

8) Определите сходимость ряда:\tabularnewline
$\sum\limits_{n=1}^{+\infty} \frac{(2n)!!}{n!} \arctan\frac{1}{3^n}$\tabularnewline
\noalign{\vskip4mm}

\end{tabular}\tabularnewline
\noalign{\vskip4mm}
\begin{tabular}{l}
Вариант Мастер джин \tabularnewline
1) Найдите интеграл:\tabularnewline
$\int \frac{x \ln x \dx}{\sqrt{1 + x^2}}$\tabularnewline
\noalign{\vskip4mm}

2) Найдите интеграл:\tabularnewline
$\int \frac{\dx}{(x^2 + 1) \sqrt{x^2 + 9}}$\tabularnewline
\noalign{\vskip4mm}

3) Вычислите площадь фигуры, ограниченной кривыми:\tabularnewline
$r = 2 |\tg \phi|; r = \frac{1}{\cos \phi}$\tabularnewline
\noalign{\vskip4mm}

4) Вычислите длину кривой:\tabularnewline
$r = \cos^3 \frac{\phi}{3}$\tabularnewline
\noalign{\vskip4mm}

5) Определите сходимость интеграла:\tabularnewline
$\int\limits_{0}^{1} \frac{\sin(\arcsin + x^3) - x}{\sqrt{\sin^7 x}} \dx$\tabularnewline
\noalign{\vskip4mm}

6) Определите сходимость интеграла:\tabularnewline
$\int\limits_{1}^{+\infty} \frac{\ln{x}\dx}{x \sqrt{x^2 - 1}}$\tabularnewline
\noalign{\vskip4mm}

7) Вычислите сумму ряда:\tabularnewline
$\sum\limits_{n = 1}^{+\infty} \left(\sin \frac{1}{n(n + 1)} \right) / \left(\cos\left(\frac{1}{n(n + 1)}\right) + \cos\left(\frac{2n + 1}{n(n + 1)}\right)\right)$\tabularnewline
\noalign{\vskip4mm}

8) Определите сходимость ряда:\tabularnewline
$\sum\limits_{n=1}^{+\infty} \frac{(2n)!!}{n!} \arctan\frac{1}{3^n}$\tabularnewline
\noalign{\vskip4mm}

\end{tabular}& %
\begin{tabular}{l}
Вариант Королева Нага \tabularnewline
1) Найдите интеграл:\tabularnewline
$\int \frac{\dx}{\sin x ( 1 + \cos x)}$\tabularnewline
\noalign{\vskip4mm}

2) Найдите интеграл:\tabularnewline
$\int \frac{\dx}{(x^2 + 1)^2}$\tabularnewline
\noalign{\vskip4mm}

3) Вычислите площадь фигуры, ограниченной кривыми:\tabularnewline
$r = 2 |\tg \phi|; r = \frac{1}{\cos \phi}$\tabularnewline
\noalign{\vskip4mm}

4) Найдите объем фигуры, полученной вращением вокруг\tabularnewline
оси OY плоской фигуры, ограниченной кривыми:\tabularnewline
$y = e^{x^2}; y = 0; x = 0; x = 1$\tabularnewline
\noalign{\vskip4mm}

5) Определите сходимость интеграла:\tabularnewline
$\int\limits_{0}^{2} \frac{\sqrt{x}\dx}{e^{\sin{x}} - 1}$\tabularnewline
\noalign{\vskip4mm}

6) Определите сходимость интеграла:\tabularnewline
$\int\limits_{0}^{+\infty} \sin^3 (x^2 + 2x) \dx$\tabularnewline
\noalign{\vskip4mm}

7) Вычислите сумму ряда:\tabularnewline
$\sum\limits_{n = 1}^{+\infty} \left(\sin \frac{1}{n(n + 1)} \right) / \left(\cos\left(\frac{1}{n(n + 1)}\right) + \cos\left(\frac{2n + 1}{n(n + 1)}\right)\right)$\tabularnewline
\noalign{\vskip4mm}

8) Определите сходимость ряда:\tabularnewline
$\sum\limits_{n=1}^{+\infty} \frac{(2n)!!}{n!} \arctan\frac{1}{3^n}$\tabularnewline
\noalign{\vskip4mm}

\end{tabular}\tabularnewline
\noalign{\vskip4mm}
\end{tabular}

\begin{tabular}{cc}
\begin{tabular}{l}
Вариант Титан \tabularnewline
1) Найдите интеграл:\tabularnewline
$\int \frac{x \ln x \dx}{\sqrt{1 + x^2}}$\tabularnewline
\noalign{\vskip4mm}

2) Найдите интеграл:\tabularnewline
$\int \frac{\dx}{(x^2 + 1) \sqrt{x^2 + 9}}$\tabularnewline
\noalign{\vskip4mm}

3) Вычислите площадь фигуры, ограниченной кривыми:\tabularnewline
$r = 2 |\tg \phi|; r = \frac{1}{\cos \phi}$\tabularnewline
\noalign{\vskip4mm}

4) Найдите объем фигуры, полученной вращением вокруг\tabularnewline
оси OY плоской фигуры, ограниченной кривыми:\tabularnewline
$y = e^{x^2}; y = 0; x = 0; x = 1$\tabularnewline
\noalign{\vskip4mm}

5) Определите сходимость интеграла:\tabularnewline
$\int\limits_{0}^{2} \frac{\sqrt{x}\dx}{e^{\sin{x}} - 1}$\tabularnewline
\noalign{\vskip4mm}

6) Определите сходимость интеграла:\tabularnewline
$\int\limits_{1}^{+\infty} \frac{\ln{x}\dx}{x \sqrt{x^2 - 1}}$\tabularnewline
\noalign{\vskip4mm}

7) Вычислите сумму ряда:\tabularnewline
$\sum\limits_{n = 1}^{+\infty} \left(\sin \frac{1}{n(n + 1)} \right) / \left(\cos\left(\frac{1}{n(n + 1)}\right) + \cos\left(\frac{2n + 1}{n(n + 1)}\right)\right)$\tabularnewline
\noalign{\vskip4mm}

8) Определите сходимость ряда:\tabularnewline
$\sum\limits_{n=1}^{+\infty} \frac{(2n)!!}{n!} \arctan\frac{1}{3^n}$\tabularnewline
\noalign{\vskip4mm}

\end{tabular}& %
\begin{tabular}{l}
Вариант Архи-Лич \tabularnewline
1) Найдите интеграл:\tabularnewline
$\int \frac{\dx}{\sin x ( 1 + \cos x)}$\tabularnewline
\noalign{\vskip4mm}

2) Найдите интеграл:\tabularnewline
$\int \frac{\dx}{(x^2 + 1) \sqrt{x^2 + 9}}$\tabularnewline
\noalign{\vskip4mm}

3) Вычислите площадь фигуры, ограниченной кривыми:\tabularnewline
$y = \frac{10}{x^2 + 4}; y = \frac{x^2 + 5x + 4}{x^2 + 4}$\tabularnewline
\noalign{\vskip4mm}

4) Найдите объем фигуры, полученной вращением вокруг\tabularnewline
оси OY плоской фигуры, ограниченной кривыми:\tabularnewline
$y = e^{x^2}; y = 0; x = 0; x = 1$\tabularnewline
\noalign{\vskip4mm}

5) Определите сходимость интеграла:\tabularnewline
$\int\limits_{0}^{2} \frac{\sqrt{x}\dx}{e^{\sin{x}} - 1}$\tabularnewline
\noalign{\vskip4mm}

6) Определите сходимость интеграла:\tabularnewline
$\int\limits_{1}^{+\infty} \frac{\ln{x}\dx}{x \sqrt{x^2 - 1}}$\tabularnewline
\noalign{\vskip4mm}

7) Вычислите сумму ряда:\tabularnewline
$\sum\limits_{n = 1}^{+\infty} \left(\sin \frac{1}{n(n + 1)} \right) / \left(\cos\left(\frac{1}{n(n + 1)}\right) + \cos\left(\frac{2n + 1}{n(n + 1)}\right)\right)$\tabularnewline
\noalign{\vskip4mm}

8) Определите сходимость ряда:\tabularnewline
$\sum\limits_{n=1}^{+\infty} \frac{\sin{n}}{n + \sin{n}}$\tabularnewline
\noalign{\vskip4mm}

\end{tabular}\tabularnewline
\noalign{\vskip4mm}
\begin{tabular}{l}
Вариант Рыцарь смерти \tabularnewline
1) Найдите интеграл:\tabularnewline
$\int \frac{\dx}{\sin x ( 1 + \cos x)}$\tabularnewline
\noalign{\vskip4mm}

2) Найдите интеграл:\tabularnewline
$\int \frac{\dx}{(x^2 + 1) \sqrt{x^2 + 9}}$\tabularnewline
\noalign{\vskip4mm}

3) Вычислите площадь фигуры, ограниченной кривыми:\tabularnewline
$y = \frac{10}{x^2 + 4}; y = \frac{x^2 + 5x + 4}{x^2 + 4}$\tabularnewline
\noalign{\vskip4mm}

4) Вычислите длину кривой:\tabularnewline
$r = \cos^3 \frac{\phi}{3}$\tabularnewline
\noalign{\vskip4mm}

5) Определите сходимость интеграла:\tabularnewline
$\int\limits_{0}^{2} \frac{\sqrt{x}\dx}{e^{\sin{x}} - 1}$\tabularnewline
\noalign{\vskip4mm}

6) Определите сходимость интеграла:\tabularnewline
$\int\limits_{0}^{+\infty} \sin^3 (x^2 + 2x) \dx$\tabularnewline
\noalign{\vskip4mm}

7) Вычислите сумму ряда:\tabularnewline
$\sum\limits_{n = 1}^{+\infty} \left(\sin \frac{1}{n(n + 1)} \right) / \left(\cos\left(\frac{1}{n(n + 1)}\right) + \cos\left(\frac{2n + 1}{n(n + 1)}\right)\right)$\tabularnewline
\noalign{\vskip4mm}

8) Определите сходимость ряда:\tabularnewline
$\sum\limits_{n=1}^{+\infty} \frac{(2n)!!}{n!} \arctan\frac{1}{3^n}$\tabularnewline
\noalign{\vskip4mm}

\end{tabular}& %
\begin{tabular}{l}
Вариант Дракон-Призрак \tabularnewline
1) Найдите интеграл:\tabularnewline
$\int \frac{\dx}{\sin x ( 1 + \cos x)}$\tabularnewline
\noalign{\vskip4mm}

2) Найдите интеграл:\tabularnewline
$\int \frac{\dx}{(x^2 + 1) \sqrt{x^2 + 9}}$\tabularnewline
\noalign{\vskip4mm}

3) Вычислите площадь фигуры, ограниченной кривыми:\tabularnewline
$r = 2 |\tg \phi|; r = \frac{1}{\cos \phi}$\tabularnewline
\noalign{\vskip4mm}

4) Вычислите длину кривой:\tabularnewline
$r = \cos^3 \frac{\phi}{3}$\tabularnewline
\noalign{\vskip4mm}

5) Определите сходимость интеграла:\tabularnewline
$\int\limits_{0}^{2} \frac{\sqrt{x}\dx}{e^{\sin{x}} - 1}$\tabularnewline
\noalign{\vskip4mm}

6) Определите сходимость интеграла:\tabularnewline
$\int\limits_{1}^{+\infty} \frac{\ln{x}\dx}{x \sqrt{x^2 - 1}}$\tabularnewline
\noalign{\vskip4mm}

7) Вычислите сумму ряда:\tabularnewline
$\sum\limits_{n = 1}^{+\infty} \left(\sin \frac{1}{n(n + 1)} \right) / \left(\cos\left(\frac{1}{n(n + 1)}\right) + \cos\left(\frac{2n + 1}{n(n + 1)}\right)\right)$\tabularnewline
\noalign{\vskip4mm}

8) Определите сходимость ряда:\tabularnewline
$\sum\limits_{n=1}^{+\infty} \frac{(2n)!!}{n!} \arctan\frac{1}{3^n}$\tabularnewline
\noalign{\vskip4mm}

\end{tabular}\tabularnewline
\noalign{\vskip4mm}
\end{tabular}

\begin{tabular}{cc}
\begin{tabular}{l}
Вариант Король - минотавр \tabularnewline
1) Найдите интеграл:\tabularnewline
$\int \frac{x \ln x \dx}{\sqrt{1 + x^2}}$\tabularnewline
\noalign{\vskip4mm}

2) Найдите интеграл:\tabularnewline
$\int \frac{\dx}{(x^2 + 1)^2}$\tabularnewline
\noalign{\vskip4mm}

3) Вычислите площадь фигуры, ограниченной кривыми:\tabularnewline
$r = 2 |\tg \phi|; r = \frac{1}{\cos \phi}$\tabularnewline
\noalign{\vskip4mm}

4) Вычислите длину кривой:\tabularnewline
$r = \cos^3 \frac{\phi}{3}$\tabularnewline
\noalign{\vskip4mm}

5) Определите сходимость интеграла:\tabularnewline
$\int\limits_{0}^{1} \frac{\sin(\arcsin + x^3) - x}{\sqrt{\sin^7 x}} \dx$\tabularnewline
\noalign{\vskip4mm}

6) Определите сходимость интеграла:\tabularnewline
$\int\limits_{0}^{+\infty} \sin^3 (x^2 + 2x) \dx$\tabularnewline
\noalign{\vskip4mm}

7) Вычислите сумму ряда:\tabularnewline
$\sum\limits_{n = 1}^{+\infty} \left(\sin \frac{1}{n(n + 1)} \right) / \left(\cos\left(\frac{1}{n(n + 1)}\right) + \cos\left(\frac{2n + 1}{n(n + 1)}\right)\right)$\tabularnewline
\noalign{\vskip4mm}

8) Определите сходимость ряда:\tabularnewline
$\sum\limits_{n=1}^{+\infty} \frac{\sin{n}}{n + \sin{n}}$\tabularnewline
\noalign{\vskip4mm}

\end{tabular}& %
\begin{tabular}{l}
Вариант Скорпикора \tabularnewline
1) Найдите интеграл:\tabularnewline
$\int \frac{\dx}{\sin x ( 1 + \cos x)}$\tabularnewline
\noalign{\vskip4mm}

2) Найдите интеграл:\tabularnewline
$\int \frac{\dx}{(x^2 + 1) \sqrt{x^2 + 9}}$\tabularnewline
\noalign{\vskip4mm}

3) Вычислите площадь фигуры, ограниченной кривыми:\tabularnewline
$r = 2 |\tg \phi|; r = \frac{1}{\cos \phi}$\tabularnewline
\noalign{\vskip4mm}

4) Найдите объем фигуры, полученной вращением вокруг\tabularnewline
оси OY плоской фигуры, ограниченной кривыми:\tabularnewline
$y = e^{x^2}; y = 0; x = 0; x = 1$\tabularnewline
\noalign{\vskip4mm}

5) Определите сходимость интеграла:\tabularnewline
$\int\limits_{0}^{2} \frac{\sqrt{x}\dx}{e^{\sin{x}} - 1}$\tabularnewline
\noalign{\vskip4mm}

6) Определите сходимость интеграла:\tabularnewline
$\int\limits_{0}^{+\infty} \sin^3 (x^2 + 2x) \dx$\tabularnewline
\noalign{\vskip4mm}

7) Вычислите сумму ряда:\tabularnewline
$\sum\limits_{n = 1}^{+\infty} \left(\sin \frac{1}{n(n + 1)} \right) / \left(\cos\left(\frac{1}{n(n + 1)}\right) + \cos\left(\frac{2n + 1}{n(n + 1)}\right)\right)$\tabularnewline
\noalign{\vskip4mm}

8) Определите сходимость ряда:\tabularnewline
$\sum\limits_{n=1}^{+\infty} \frac{(2n)!!}{n!} \arctan\frac{1}{3^n}$\tabularnewline
\noalign{\vskip4mm}

\end{tabular}\tabularnewline
\noalign{\vskip4mm}
\begin{tabular}{l}
Вариант Черный дракон \tabularnewline
1) Найдите интеграл:\tabularnewline
$\int \frac{x \ln x \dx}{\sqrt{1 + x^2}}$\tabularnewline
\noalign{\vskip4mm}

2) Найдите интеграл:\tabularnewline
$\int \frac{\dx}{(x^2 + 1) \sqrt{x^2 + 9}}$\tabularnewline
\noalign{\vskip4mm}

3) Вычислите площадь фигуры, ограниченной кривыми:\tabularnewline
$r = 2 |\tg \phi|; r = \frac{1}{\cos \phi}$\tabularnewline
\noalign{\vskip4mm}

4) Найдите объем фигуры, полученной вращением вокруг\tabularnewline
оси OY плоской фигуры, ограниченной кривыми:\tabularnewline
$y = e^{x^2}; y = 0; x = 0; x = 1$\tabularnewline
\noalign{\vskip4mm}

5) Определите сходимость интеграла:\tabularnewline
$\int\limits_{0}^{2} \frac{\sqrt{x}\dx}{e^{\sin{x}} - 1}$\tabularnewline
\noalign{\vskip4mm}

6) Определите сходимость интеграла:\tabularnewline
$\int\limits_{0}^{+\infty} \sin^3 (x^2 + 2x) \dx$\tabularnewline
\noalign{\vskip4mm}

7) Вычислите сумму ряда:\tabularnewline
$\sum\limits_{n = 1}^{+\infty} \left(\sin \frac{1}{n(n + 1)} \right) / \left(\cos\left(\frac{1}{n(n + 1)}\right) + \cos\left(\frac{2n + 1}{n(n + 1)}\right)\right)$\tabularnewline
\noalign{\vskip4mm}

8) Определите сходимость ряда:\tabularnewline
$\sum\limits_{n=1}^{+\infty} \frac{\sin{n}}{n + \sin{n}}$\tabularnewline
\noalign{\vskip4mm}

\end{tabular}& %
\begin{tabular}{l}
Вариант Адское отродье \tabularnewline
1) Найдите интеграл:\tabularnewline
$\int \frac{\dx}{\sin x ( 1 + \cos x)}$\tabularnewline
\noalign{\vskip4mm}

2) Найдите интеграл:\tabularnewline
$\int \frac{\dx}{(x^2 + 1)^2}$\tabularnewline
\noalign{\vskip4mm}

3) Вычислите площадь фигуры, ограниченной кривыми:\tabularnewline
$y = \frac{10}{x^2 + 4}; y = \frac{x^2 + 5x + 4}{x^2 + 4}$\tabularnewline
\noalign{\vskip4mm}

4) Вычислите длину кривой:\tabularnewline
$r = \cos^3 \frac{\phi}{3}$\tabularnewline
\noalign{\vskip4mm}

5) Определите сходимость интеграла:\tabularnewline
$\int\limits_{0}^{2} \frac{\sqrt{x}\dx}{e^{\sin{x}} - 1}$\tabularnewline
\noalign{\vskip4mm}

6) Определите сходимость интеграла:\tabularnewline
$\int\limits_{0}^{+\infty} \sin^3 (x^2 + 2x) \dx$\tabularnewline
\noalign{\vskip4mm}

7) Вычислите сумму ряда:\tabularnewline
$\sum\limits_{n = 1}^{+\infty} \left(\sin \frac{1}{n(n + 1)} \right) / \left(\cos\left(\frac{1}{n(n + 1)}\right) + \cos\left(\frac{2n + 1}{n(n + 1)}\right)\right)$\tabularnewline
\noalign{\vskip4mm}

8) Определите сходимость ряда:\tabularnewline
$\sum\limits_{n=1}^{+\infty} \frac{\sin{n}}{n + \sin{n}}$\tabularnewline
\noalign{\vskip4mm}

\end{tabular}\tabularnewline
\noalign{\vskip4mm}
\end{tabular}

\begin{tabular}{cc}
\begin{tabular}{l}
Вариант Султан Эфрит \tabularnewline
1) Найдите интеграл:\tabularnewline
$\int \frac{\dx}{\sin x ( 1 + \cos x)}$\tabularnewline
\noalign{\vskip4mm}

2) Найдите интеграл:\tabularnewline
$\int \frac{\dx}{(x^2 + 1)^2}$\tabularnewline
\noalign{\vskip4mm}

3) Вычислите площадь фигуры, ограниченной кривыми:\tabularnewline
$r = 2 |\tg \phi|; r = \frac{1}{\cos \phi}$\tabularnewline
\noalign{\vskip4mm}

4) Вычислите длину кривой:\tabularnewline
$r = \cos^3 \frac{\phi}{3}$\tabularnewline
\noalign{\vskip4mm}

5) Определите сходимость интеграла:\tabularnewline
$\int\limits_{0}^{2} \frac{\sqrt{x}\dx}{e^{\sin{x}} - 1}$\tabularnewline
\noalign{\vskip4mm}

6) Определите сходимость интеграла:\tabularnewline
$\int\limits_{1}^{+\infty} \frac{\ln{x}\dx}{x \sqrt{x^2 - 1}}$\tabularnewline
\noalign{\vskip4mm}

7) Вычислите сумму ряда:\tabularnewline
$\sum\limits_{n = 1}^{+\infty} \left(\sin \frac{1}{n(n + 1)} \right) / \left(\cos\left(\frac{1}{n(n + 1)}\right) + \cos\left(\frac{2n + 1}{n(n + 1)}\right)\right)$\tabularnewline
\noalign{\vskip4mm}

8) Определите сходимость ряда:\tabularnewline
$\sum\limits_{n=1}^{+\infty} \frac{(2n)!!}{n!} \arctan\frac{1}{3^n}$\tabularnewline
\noalign{\vskip4mm}

\end{tabular}& %
\begin{tabular}{l}
Вариант Архидьявол \tabularnewline
1) Найдите интеграл:\tabularnewline
$\int \frac{x \ln x \dx}{\sqrt{1 + x^2}}$\tabularnewline
\noalign{\vskip4mm}

2) Найдите интеграл:\tabularnewline
$\int \frac{\dx}{(x^2 + 1) \sqrt{x^2 + 9}}$\tabularnewline
\noalign{\vskip4mm}

3) Вычислите площадь фигуры, ограниченной кривыми:\tabularnewline
$y = \frac{10}{x^2 + 4}; y = \frac{x^2 + 5x + 4}{x^2 + 4}$\tabularnewline
\noalign{\vskip4mm}

4) Вычислите длину кривой:\tabularnewline
$r = \cos^3 \frac{\phi}{3}$\tabularnewline
\noalign{\vskip4mm}

5) Определите сходимость интеграла:\tabularnewline
$\int\limits_{0}^{1} \frac{\sin(\arcsin + x^3) - x}{\sqrt{\sin^7 x}} \dx$\tabularnewline
\noalign{\vskip4mm}

6) Определите сходимость интеграла:\tabularnewline
$\int\limits_{0}^{+\infty} \sin^3 (x^2 + 2x) \dx$\tabularnewline
\noalign{\vskip4mm}

7) Вычислите сумму ряда:\tabularnewline
$\sum\limits_{n = 1}^{+\infty} \left(2n\sin\frac{1}{2n(n + 1)}\cos\frac{2n + 1}{2n(n + 1)} - \sin\frac{1}{n + 1}\right)$\tabularnewline
\noalign{\vskip4mm}

8) Определите сходимость ряда:\tabularnewline
$\sum\limits_{n=1}^{+\infty} \frac{(2n)!!}{n!} \arctan\frac{1}{3^n}$\tabularnewline
\noalign{\vskip4mm}

\end{tabular}\tabularnewline
\noalign{\vskip4mm}
\begin{tabular}{l}
Вариант Могучая горгона \tabularnewline
1) Найдите интеграл:\tabularnewline
$\int \frac{\dx}{\sin x ( 1 + \cos x)}$\tabularnewline
\noalign{\vskip4mm}

2) Найдите интеграл:\tabularnewline
$\int \frac{\dx}{(x^2 + 1) \sqrt{x^2 + 9}}$\tabularnewline
\noalign{\vskip4mm}

3) Вычислите площадь фигуры, ограниченной кривыми:\tabularnewline
$r = 2 |\tg \phi|; r = \frac{1}{\cos \phi}$\tabularnewline
\noalign{\vskip4mm}

4) Найдите объем фигуры, полученной вращением вокруг\tabularnewline
оси OY плоской фигуры, ограниченной кривыми:\tabularnewline
$y = e^{x^2}; y = 0; x = 0; x = 1$\tabularnewline
\noalign{\vskip4mm}

5) Определите сходимость интеграла:\tabularnewline
$\int\limits_{0}^{2} \frac{\sqrt{x}\dx}{e^{\sin{x}} - 1}$\tabularnewline
\noalign{\vskip4mm}

6) Определите сходимость интеграла:\tabularnewline
$\int\limits_{1}^{+\infty} \frac{\ln{x}\dx}{x \sqrt{x^2 - 1}}$\tabularnewline
\noalign{\vskip4mm}

7) Вычислите сумму ряда:\tabularnewline
$\sum\limits_{n = 1}^{+\infty} \left(2n\sin\frac{1}{2n(n + 1)}\cos\frac{2n + 1}{2n(n + 1)} - \sin\frac{1}{n + 1}\right)$\tabularnewline
\noalign{\vskip4mm}

8) Определите сходимость ряда:\tabularnewline
$\sum\limits_{n=1}^{+\infty} \frac{(2n)!!}{n!} \arctan\frac{1}{3^n}$\tabularnewline
\noalign{\vskip4mm}

\end{tabular}& %
\begin{tabular}{l}
Вариант Виверна - монарх \tabularnewline
1) Найдите интеграл:\tabularnewline
$\int \frac{x \ln x \dx}{\sqrt{1 + x^2}}$\tabularnewline
\noalign{\vskip4mm}

2) Найдите интеграл:\tabularnewline
$\int \frac{\dx}{(x^2 + 1) \sqrt{x^2 + 9}}$\tabularnewline
\noalign{\vskip4mm}

3) Вычислите площадь фигуры, ограниченной кривыми:\tabularnewline
$r = 2 |\tg \phi|; r = \frac{1}{\cos \phi}$\tabularnewline
\noalign{\vskip4mm}

4) Вычислите длину кривой:\tabularnewline
$r = \cos^3 \frac{\phi}{3}$\tabularnewline
\noalign{\vskip4mm}

5) Определите сходимость интеграла:\tabularnewline
$\int\limits_{0}^{2} \frac{\sqrt{x}\dx}{e^{\sin{x}} - 1}$\tabularnewline
\noalign{\vskip4mm}

6) Определите сходимость интеграла:\tabularnewline
$\int\limits_{1}^{+\infty} \frac{\ln{x}\dx}{x \sqrt{x^2 - 1}}$\tabularnewline
\noalign{\vskip4mm}

7) Вычислите сумму ряда:\tabularnewline
$\sum\limits_{n = 1}^{+\infty} \left(\sin \frac{1}{n(n + 1)} \right) / \left(\cos\left(\frac{1}{n(n + 1)}\right) + \cos\left(\frac{2n + 1}{n(n + 1)}\right)\right)$\tabularnewline
\noalign{\vskip4mm}

8) Определите сходимость ряда:\tabularnewline
$\sum\limits_{n=1}^{+\infty} \frac{(2n)!!}{n!} \arctan\frac{1}{3^n}$\tabularnewline
\noalign{\vskip4mm}

\end{tabular}\tabularnewline
\noalign{\vskip4mm}
\end{tabular}

\begin{tabular}{cc}
\begin{tabular}{l}
Вариант Гидра хаоса \tabularnewline
1) Найдите интеграл:\tabularnewline
$\int \frac{x \ln x \dx}{\sqrt{1 + x^2}}$\tabularnewline
\noalign{\vskip4mm}

2) Найдите интеграл:\tabularnewline
$\int \frac{\dx}{(x^2 + 1) \sqrt{x^2 + 9}}$\tabularnewline
\noalign{\vskip4mm}

3) Вычислите площадь фигуры, ограниченной кривыми:\tabularnewline
$y = \frac{10}{x^2 + 4}; y = \frac{x^2 + 5x + 4}{x^2 + 4}$\tabularnewline
\noalign{\vskip4mm}

4) Найдите объем фигуры, полученной вращением вокруг\tabularnewline
оси OY плоской фигуры, ограниченной кривыми:\tabularnewline
$y = e^{x^2}; y = 0; x = 0; x = 1$\tabularnewline
\noalign{\vskip4mm}

5) Определите сходимость интеграла:\tabularnewline
$\int\limits_{0}^{2} \frac{\sqrt{x}\dx}{e^{\sin{x}} - 1}$\tabularnewline
\noalign{\vskip4mm}

6) Определите сходимость интеграла:\tabularnewline
$\int\limits_{1}^{+\infty} \frac{\ln{x}\dx}{x \sqrt{x^2 - 1}}$\tabularnewline
\noalign{\vskip4mm}

7) Вычислите сумму ряда:\tabularnewline
$\sum\limits_{n = 1}^{+\infty} \left(2n\sin\frac{1}{2n(n + 1)}\cos\frac{2n + 1}{2n(n + 1)} - \sin\frac{1}{n + 1}\right)$\tabularnewline
\noalign{\vskip4mm}

8) Определите сходимость ряда:\tabularnewline
$\sum\limits_{n=1}^{+\infty} \frac{\sin{n}}{n + \sin{n}}$\tabularnewline
\noalign{\vskip4mm}

\end{tabular}& %
\begin{tabular}{l}
Вариант Птица грома \tabularnewline
1) Найдите интеграл:\tabularnewline
$\int \frac{x \ln x \dx}{\sqrt{1 + x^2}}$\tabularnewline
\noalign{\vskip4mm}

2) Найдите интеграл:\tabularnewline
$\int \frac{\dx}{(x^2 + 1)^2}$\tabularnewline
\noalign{\vskip4mm}

3) Вычислите площадь фигуры, ограниченной кривыми:\tabularnewline
$r = 2 |\tg \phi|; r = \frac{1}{\cos \phi}$\tabularnewline
\noalign{\vskip4mm}

4) Найдите объем фигуры, полученной вращением вокруг\tabularnewline
оси OY плоской фигуры, ограниченной кривыми:\tabularnewline
$y = e^{x^2}; y = 0; x = 0; x = 1$\tabularnewline
\noalign{\vskip4mm}

5) Определите сходимость интеграла:\tabularnewline
$\int\limits_{0}^{2} \frac{\sqrt{x}\dx}{e^{\sin{x}} - 1}$\tabularnewline
\noalign{\vskip4mm}

6) Определите сходимость интеграла:\tabularnewline
$\int\limits_{0}^{+\infty} \sin^3 (x^2 + 2x) \dx$\tabularnewline
\noalign{\vskip4mm}

7) Вычислите сумму ряда:\tabularnewline
$\sum\limits_{n = 1}^{+\infty} \left(\sin \frac{1}{n(n + 1)} \right) / \left(\cos\left(\frac{1}{n(n + 1)}\right) + \cos\left(\frac{2n + 1}{n(n + 1)}\right)\right)$\tabularnewline
\noalign{\vskip4mm}

8) Определите сходимость ряда:\tabularnewline
$\sum\limits_{n=1}^{+\infty} \frac{(2n)!!}{n!} \arctan\frac{1}{3^n}$\tabularnewline
\noalign{\vskip4mm}

\end{tabular}\tabularnewline
\noalign{\vskip4mm}
\begin{tabular}{l}
Вариант Король циклопов \tabularnewline
1) Найдите интеграл:\tabularnewline
$\int \frac{x \ln x \dx}{\sqrt{1 + x^2}}$\tabularnewline
\noalign{\vskip4mm}

2) Найдите интеграл:\tabularnewline
$\int \frac{\dx}{(x^2 + 1) \sqrt{x^2 + 9}}$\tabularnewline
\noalign{\vskip4mm}

3) Вычислите площадь фигуры, ограниченной кривыми:\tabularnewline
$y = \frac{10}{x^2 + 4}; y = \frac{x^2 + 5x + 4}{x^2 + 4}$\tabularnewline
\noalign{\vskip4mm}

4) Найдите объем фигуры, полученной вращением вокруг\tabularnewline
оси OY плоской фигуры, ограниченной кривыми:\tabularnewline
$y = e^{x^2}; y = 0; x = 0; x = 1$\tabularnewline
\noalign{\vskip4mm}

5) Определите сходимость интеграла:\tabularnewline
$\int\limits_{0}^{1} \frac{\sin(\arcsin + x^3) - x}{\sqrt{\sin^7 x}} \dx$\tabularnewline
\noalign{\vskip4mm}

6) Определите сходимость интеграла:\tabularnewline
$\int\limits_{0}^{+\infty} \sin^3 (x^2 + 2x) \dx$\tabularnewline
\noalign{\vskip4mm}

7) Вычислите сумму ряда:\tabularnewline
$\sum\limits_{n = 1}^{+\infty} \left(2n\sin\frac{1}{2n(n + 1)}\cos\frac{2n + 1}{2n(n + 1)} - \sin\frac{1}{n + 1}\right)$\tabularnewline
\noalign{\vskip4mm}

8) Определите сходимость ряда:\tabularnewline
$\sum\limits_{n=1}^{+\infty} \frac{(2n)!!}{n!} \arctan\frac{1}{3^n}$\tabularnewline
\noalign{\vskip4mm}

\end{tabular}& %
\begin{tabular}{l}
Вариант Древнее чудище \tabularnewline
1) Найдите интеграл:\tabularnewline
$\int \frac{x \ln x \dx}{\sqrt{1 + x^2}}$\tabularnewline
\noalign{\vskip4mm}

2) Найдите интеграл:\tabularnewline
$\int \frac{\dx}{(x^2 + 1)^2}$\tabularnewline
\noalign{\vskip4mm}

3) Вычислите площадь фигуры, ограниченной кривыми:\tabularnewline
$r = 2 |\tg \phi|; r = \frac{1}{\cos \phi}$\tabularnewline
\noalign{\vskip4mm}

4) Найдите объем фигуры, полученной вращением вокруг\tabularnewline
оси OY плоской фигуры, ограниченной кривыми:\tabularnewline
$y = e^{x^2}; y = 0; x = 0; x = 1$\tabularnewline
\noalign{\vskip4mm}

5) Определите сходимость интеграла:\tabularnewline
$\int\limits_{0}^{1} \frac{\sin(\arcsin + x^3) - x}{\sqrt{\sin^7 x}} \dx$\tabularnewline
\noalign{\vskip4mm}

6) Определите сходимость интеграла:\tabularnewline
$\int\limits_{1}^{+\infty} \frac{\ln{x}\dx}{x \sqrt{x^2 - 1}}$\tabularnewline
\noalign{\vskip4mm}

7) Вычислите сумму ряда:\tabularnewline
$\sum\limits_{n = 1}^{+\infty} \left(2n\sin\frac{1}{2n(n + 1)}\cos\frac{2n + 1}{2n(n + 1)} - \sin\frac{1}{n + 1}\right)$\tabularnewline
\noalign{\vskip4mm}

8) Определите сходимость ряда:\tabularnewline
$\sum\limits_{n=1}^{+\infty} \frac{(2n)!!}{n!} \arctan\frac{1}{3^n}$\tabularnewline
\noalign{\vskip4mm}

\end{tabular}\tabularnewline
\noalign{\vskip4mm}
\end{tabular}

\begin{tabular}{cc}
\begin{tabular}{l}
Вариант Элементаль магмы \tabularnewline
1) Найдите интеграл:\tabularnewline
$\int \frac{x \ln x \dx}{\sqrt{1 + x^2}}$\tabularnewline
\noalign{\vskip4mm}

2) Найдите интеграл:\tabularnewline
$\int \frac{\dx}{(x^2 + 1) \sqrt{x^2 + 9}}$\tabularnewline
\noalign{\vskip4mm}

3) Вычислите площадь фигуры, ограниченной кривыми:\tabularnewline
$y = \frac{10}{x^2 + 4}; y = \frac{x^2 + 5x + 4}{x^2 + 4}$\tabularnewline
\noalign{\vskip4mm}

4) Найдите объем фигуры, полученной вращением вокруг\tabularnewline
оси OY плоской фигуры, ограниченной кривыми:\tabularnewline
$y = e^{x^2}; y = 0; x = 0; x = 1$\tabularnewline
\noalign{\vskip4mm}

5) Определите сходимость интеграла:\tabularnewline
$\int\limits_{0}^{1} \frac{\sin(\arcsin + x^3) - x}{\sqrt{\sin^7 x}} \dx$\tabularnewline
\noalign{\vskip4mm}

6) Определите сходимость интеграла:\tabularnewline
$\int\limits_{0}^{+\infty} \sin^3 (x^2 + 2x) \dx$\tabularnewline
\noalign{\vskip4mm}

7) Вычислите сумму ряда:\tabularnewline
$\sum\limits_{n = 1}^{+\infty} \left(2n\sin\frac{1}{2n(n + 1)}\cos\frac{2n + 1}{2n(n + 1)} - \sin\frac{1}{n + 1}\right)$\tabularnewline
\noalign{\vskip4mm}

8) Определите сходимость ряда:\tabularnewline
$\sum\limits_{n=1}^{+\infty} \frac{\sin{n}}{n + \sin{n}}$\tabularnewline
\noalign{\vskip4mm}

\end{tabular}& %
\begin{tabular}{l}
Вариант Магический элементаль \tabularnewline
1) Найдите интеграл:\tabularnewline
$\int \frac{x \ln x \dx}{\sqrt{1 + x^2}}$\tabularnewline
\noalign{\vskip4mm}

2) Найдите интеграл:\tabularnewline
$\int \frac{\dx}{(x^2 + 1) \sqrt{x^2 + 9}}$\tabularnewline
\noalign{\vskip4mm}

3) Вычислите площадь фигуры, ограниченной кривыми:\tabularnewline
$r = 2 |\tg \phi|; r = \frac{1}{\cos \phi}$\tabularnewline
\noalign{\vskip4mm}

4) Найдите объем фигуры, полученной вращением вокруг\tabularnewline
оси OY плоской фигуры, ограниченной кривыми:\tabularnewline
$y = e^{x^2}; y = 0; x = 0; x = 1$\tabularnewline
\noalign{\vskip4mm}

5) Определите сходимость интеграла:\tabularnewline
$\int\limits_{0}^{2} \frac{\sqrt{x}\dx}{e^{\sin{x}} - 1}$\tabularnewline
\noalign{\vskip4mm}

6) Определите сходимость интеграла:\tabularnewline
$\int\limits_{0}^{+\infty} \sin^3 (x^2 + 2x) \dx$\tabularnewline
\noalign{\vskip4mm}

7) Вычислите сумму ряда:\tabularnewline
$\sum\limits_{n = 1}^{+\infty} \left(2n\sin\frac{1}{2n(n + 1)}\cos\frac{2n + 1}{2n(n + 1)} - \sin\frac{1}{n + 1}\right)$\tabularnewline
\noalign{\vskip4mm}

8) Определите сходимость ряда:\tabularnewline
$\sum\limits_{n=1}^{+\infty} \frac{\sin{n}}{n + \sin{n}}$\tabularnewline
\noalign{\vskip4mm}

\end{tabular}\tabularnewline
\noalign{\vskip4mm}
\begin{tabular}{l}
Вариант Феникс \tabularnewline
1) Найдите интеграл:\tabularnewline
$\int \frac{x \ln x \dx}{\sqrt{1 + x^2}}$\tabularnewline
\noalign{\vskip4mm}

2) Найдите интеграл:\tabularnewline
$\int \frac{\dx}{(x^2 + 1)^2}$\tabularnewline
\noalign{\vskip4mm}

3) Вычислите площадь фигуры, ограниченной кривыми:\tabularnewline
$r = 2 |\tg \phi|; r = \frac{1}{\cos \phi}$\tabularnewline
\noalign{\vskip4mm}

4) Найдите объем фигуры, полученной вращением вокруг\tabularnewline
оси OY плоской фигуры, ограниченной кривыми:\tabularnewline
$y = e^{x^2}; y = 0; x = 0; x = 1$\tabularnewline
\noalign{\vskip4mm}

5) Определите сходимость интеграла:\tabularnewline
$\int\limits_{0}^{2} \frac{\sqrt{x}\dx}{e^{\sin{x}} - 1}$\tabularnewline
\noalign{\vskip4mm}

6) Определите сходимость интеграла:\tabularnewline
$\int\limits_{0}^{+\infty} \sin^3 (x^2 + 2x) \dx$\tabularnewline
\noalign{\vskip4mm}

7) Вычислите сумму ряда:\tabularnewline
$\sum\limits_{n = 1}^{+\infty} \left(2n\sin\frac{1}{2n(n + 1)}\cos\frac{2n + 1}{2n(n + 1)} - \sin\frac{1}{n + 1}\right)$\tabularnewline
\noalign{\vskip4mm}

8) Определите сходимость ряда:\tabularnewline
$\sum\limits_{n=1}^{+\infty} \frac{(2n)!!}{n!} \arctan\frac{1}{3^n}$\tabularnewline
\noalign{\vskip4mm}

\end{tabular}& %
\begin{tabular}{l}
Вариант Лазурный дракон \tabularnewline
1) Найдите интеграл:\tabularnewline
$\int \frac{\dx}{\sin x ( 1 + \cos x)}$\tabularnewline
\noalign{\vskip4mm}

2) Найдите интеграл:\tabularnewline
$\int \frac{\dx}{(x^2 + 1) \sqrt{x^2 + 9}}$\tabularnewline
\noalign{\vskip4mm}

3) Вычислите площадь фигуры, ограниченной кривыми:\tabularnewline
$y = \frac{10}{x^2 + 4}; y = \frac{x^2 + 5x + 4}{x^2 + 4}$\tabularnewline
\noalign{\vskip4mm}

4) Найдите объем фигуры, полученной вращением вокруг\tabularnewline
оси OY плоской фигуры, ограниченной кривыми:\tabularnewline
$y = e^{x^2}; y = 0; x = 0; x = 1$\tabularnewline
\noalign{\vskip4mm}

5) Определите сходимость интеграла:\tabularnewline
$\int\limits_{0}^{1} \frac{\sin(\arcsin + x^3) - x}{\sqrt{\sin^7 x}} \dx$\tabularnewline
\noalign{\vskip4mm}

6) Определите сходимость интеграла:\tabularnewline
$\int\limits_{0}^{+\infty} \sin^3 (x^2 + 2x) \dx$\tabularnewline
\noalign{\vskip4mm}

7) Вычислите сумму ряда:\tabularnewline
$\sum\limits_{n = 1}^{+\infty} \left(\sin \frac{1}{n(n + 1)} \right) / \left(\cos\left(\frac{1}{n(n + 1)}\right) + \cos\left(\frac{2n + 1}{n(n + 1)}\right)\right)$\tabularnewline
\noalign{\vskip4mm}

8) Определите сходимость ряда:\tabularnewline
$\sum\limits_{n=1}^{+\infty} \frac{(2n)!!}{n!} \arctan\frac{1}{3^n}$\tabularnewline
\noalign{\vskip4mm}

\end{tabular}\tabularnewline
\noalign{\vskip4mm}
\end{tabular}


\end{document}
