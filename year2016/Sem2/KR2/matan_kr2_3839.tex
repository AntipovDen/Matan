\documentclass[russian,landscape]{article}
\usepackage[T2A,T1]{fontenc}
\usepackage[utf8]{inputenc}
\usepackage[a4paper]{geometry}
\geometry{verbose,tmargin=1cm,bmargin=0cm,lmargin=0cm,rmargin=0cm}
\usepackage{amsmath}
\usepackage{amsfonts}

\makeatletter

\DeclareRobustCommand{\cyrtext}{%
  \fontencoding{T2A}\selectfont\def\encodingdefault{T2A}}
\DeclareRobustCommand{\textcyr}[1]{\leavevmode{\cyrtext #1}}
\AtBeginDocument{\DeclareFontEncoding{T2A}{}{}}

\newcommand{\dx}{\text{d}x}
\makeatother

\usepackage[russian]{babel}
\begin{document}

\begin{tabular}{cc}
\begin{tabular}{l}
Вариант Повелитель гигантов \tabularnewline
Вычислите интеграл:\tabularnewline
$\int \left(\frac{\ln x}{x} \right)^3 \dx$\tabularnewline
\noalign{\vskip4mm}

Вычислите интеграл:\tabularnewline
$\int \frac{(x + 1)dx}{(2x^2 + 1) \sqrt{x^2 + 1}}$\tabularnewline
\noalign{\vskip4mm}

Вычислите площадь фигуры, ограниченной кривыми:\tabularnewline
$x^2 +y^2 = 2 \text{ и } y^2 = 2x - 1$\tabularnewline
\noalign{\vskip4mm}

Вычислите объем тела вращения плоской фигуры, ограниченной данными кривыми, вокруг оси Oy:\tabularnewline
$y = e^x + 6, y = e^{2x}, x = 0$ \tabularnewline
\noalign{\vskip4mm}

\end{tabular}& %
\begin{tabular}{l}
Вариант Два Драконьих всадника \tabularnewline
Вычислите интеграл:\tabularnewline
$\int \frac{\cos^3 x}{\sqrt[5]{\sin x}}\dx$\tabularnewline
\noalign{\vskip4mm}

Вычислите интеграл:\tabularnewline
$\int \frac{x^4 - 2x^2 + 2}{(x^2 - 2x + 2)^2}\dx$\tabularnewline
\noalign{\vskip4mm}

Вычислите площадь фигуры, ограниченной кривыми:\tabularnewline
$x^2 +y^2 = 2 \text{ и } y^2 = 2x - 1$\tabularnewline
\noalign{\vskip4mm}

Вычислите длину петли кривой:\tabularnewline
$r = \frac{1}{\sin^2 (\phi / 3)}, \phi \in [0; 3\pi]$\tabularnewline
\noalign{\vskip4mm}

\end{tabular}\tabularnewline
\noalign{\vskip4mm}
\begin{tabular}{l}
Вариант Защитник трона и Смотритель трона \tabularnewline
Выразите через $Li(x)$:\tabularnewline
$\int \frac{x^{100} \dx}{\ln x}$\tabularnewline
\noalign{\vskip4mm}

Вычислите интеграл:\tabularnewline
$\int \frac{x^4 - 2x^2 + 2}{(x^2 - 2x + 2)^2}\dx$\tabularnewline
\noalign{\vskip4mm}

Вычислите площадь фигуры, ограниченной кривыми:\tabularnewline
$r = 2\sqrt{\phi\arccos(\phi^2 - 1)}, \text{ }, \phi = 1 \text{ и } \phi = \sqrt{\frac{3}{2}}$\tabularnewline
\noalign{\vskip4mm}

Вычислите длину кривой:\tabularnewline
$x = (1 - \cos 2t), y = \sin t - \frac{\sin 3t}{3}, t \in [0; \frac{\pi}{2}]$\tabularnewline
\noalign{\vskip4mm}

\end{tabular}& %
\begin{tabular}{l}
Вариант Нашандра \tabularnewline
Вычислите интеграл:\tabularnewline
$\int \left(\frac{\ln x}{x} \right)^3 \dx$\tabularnewline
\noalign{\vskip4mm}

Вычислите интеграл:\tabularnewline
$\int \frac{\dx}{(x - 1)^3 \sqrt{x^2 - 2x - 1}}$\tabularnewline
\noalign{\vskip4mm}

Вычислите площадь фигуры, ограниченной кривыми:\tabularnewline
$x^2 +y^2 = 2 \text{ и } y^2 = 2x - 1$\tabularnewline
\noalign{\vskip4mm}

Вычислите длину петли кривой:\tabularnewline
$r = \frac{1}{\sin^2 (\phi / 3)}, \phi \in [0; 3\pi]$\tabularnewline
\noalign{\vskip4mm}

\end{tabular}\tabularnewline
\noalign{\vskip4mm}
\begin{tabular}{l}
Вариант Преследователь \tabularnewline
Вычислите интеграл:\tabularnewline
$\int \frac{\dx}{3 \ch x + 5 \sh x + 3}$\tabularnewline
\noalign{\vskip4mm}

Вычислите интеграл:\tabularnewline
$\int \frac{(x + 1)dx}{(2x^2 + 1) \sqrt{x^2 + 1}}$\tabularnewline
\noalign{\vskip4mm}

Вычислите площадь фигуры, ограниченной кривыми:\tabularnewline
$x = \cos^3 t, y = \sin^3 t$\tabularnewline
\noalign{\vskip4mm}

Вычислите длину петли кривой:\tabularnewline
$r = \frac{1}{\sin^2 (\phi / 3)}, \phi \in [0; 3\pi]$\tabularnewline
\noalign{\vskip4mm}

\end{tabular}& %
\begin{tabular}{l}
Вариант Древний Драконоборец \tabularnewline
Вычислите интеграл:\tabularnewline
$\int \cos x \ln (1 + \sin^2 x) \dx$\tabularnewline
\noalign{\vskip4mm}

Вычислите интеграл:\tabularnewline
$\int \frac{\dx}{x^3 \sqrt[3]{2 - x^3}}$\tabularnewline
\noalign{\vskip4mm}

Вычислите площадь фигуры, ограниченной кривыми:\tabularnewline
$r = 2\sqrt{\phi\arccos(\phi^2 - 1)}, \text{ }, \phi = 1 \text{ и } \phi = \sqrt{\frac{3}{2}}$\tabularnewline
\noalign{\vskip4mm}

Вычислите объем тела вращения данной кривой вокруг оси Ox:\tabularnewline
$x = \cos^3 t, y = \sin^3 t, t\in[0; 2\pi]$\tabularnewline
\noalign{\vskip4mm}

\end{tabular}\tabularnewline
\noalign{\vskip4mm}
\end{tabular}

\begin{tabular}{cc}
\begin{tabular}{l}
Вариант Гибкий часовой \tabularnewline
Вычислите интеграл:\tabularnewline
$\int \cos x \ln (1 + \sin^2 x) \dx$\tabularnewline
\noalign{\vskip4mm}

Вычислите интеграл:\tabularnewline
$\int \frac{\dx}{(x - 1)^3 \sqrt{x^2 - 2x - 1}}$\tabularnewline
\noalign{\vskip4mm}

Вычислите площадь фигуры, ограниченной кривыми:\tabularnewline
$r = 2\sqrt{\phi\arccos(\phi^2 - 1)}, \text{ }, \phi = 1 \text{ и } \phi = \sqrt{\frac{3}{2}}$\tabularnewline
\noalign{\vskip4mm}

Вычислите длину петли кривой:\tabularnewline
$r = \frac{1}{\sin^2 (\phi / 3)}, \phi \in [0; 3\pi]$\tabularnewline
\noalign{\vskip4mm}

\end{tabular}& %
\begin{tabular}{l}
Вариант Горгульи с башни \tabularnewline
Вычислите интеграл:\tabularnewline
$\int \frac{\dx}{3 \ch x + 5 \sh x + 3}$\tabularnewline
\noalign{\vskip4mm}

Вычислите интеграл:\tabularnewline
$\int \frac{\dx}{x^3 \sqrt[3]{2 - x^3}}$\tabularnewline
\noalign{\vskip4mm}

Вычислите площадь фигуры, ограниченной кривыми:\tabularnewline
$x^2 +y^2 = 2 \text{ и } y^2 = 2x - 1$\tabularnewline
\noalign{\vskip4mm}

Вычислите объем тела вращения данной кривой вокруг оси Ox:\tabularnewline
$x = \cos^3 t, y = \sin^3 t, t\in[0; 2\pi]$\tabularnewline
\noalign{\vskip4mm}

\end{tabular}\tabularnewline
\noalign{\vskip4mm}
\begin{tabular}{l}
Вариант Колесница палача \tabularnewline
Вычислите интеграл:\tabularnewline
$\int \frac{\dx}{3 \ch x + 5 \sh x + 3}$\tabularnewline
\noalign{\vskip4mm}

Вычислите интеграл:\tabularnewline
$\int \frac{(x + 1)dx}{(2x^2 + 1) \sqrt{x^2 + 1}}$\tabularnewline
\noalign{\vskip4mm}

Вычислите площадь фигуры, ограниченной кривыми:\tabularnewline
$x = \cos^3 t, y = \sin^3 t$\tabularnewline
\noalign{\vskip4mm}

Вычислите объем тела вращения плоской фигуры, ограниченной данными кривыми, вокруг оси Oy:\tabularnewline
$y = e^x + 6, y = e^{2x}, x = 0$ \tabularnewline
\noalign{\vskip4mm}

\end{tabular}& %
\begin{tabular}{l}
Вариант Прячущийся во тьме \tabularnewline
Вычислите интеграл:\tabularnewline
$\int \cos \ln x \dx$\tabularnewline
\noalign{\vskip4mm}

Вычислите интеграл:\tabularnewline
$\int \frac{\dx}{(x - 1)^3 \sqrt{x^2 - 2x - 1}}$\tabularnewline
\noalign{\vskip4mm}

Вычислите площадь фигуры, ограниченной кривыми:\tabularnewline
$r = 2\sqrt{\phi\arccos(\phi^2 - 1)}, \text{ }, \phi = 1 \text{ и } \phi = \sqrt{\frac{3}{2}}$\tabularnewline
\noalign{\vskip4mm}

Вычислите длину кривой:\tabularnewline
$x = (1 - \cos 2t), y = \sin t - \frac{\sin 3t}{3}, t \in [0; \frac{\pi}{2}]$\tabularnewline
\noalign{\vskip4mm}

\end{tabular}\tabularnewline
\noalign{\vskip4mm}
\begin{tabular}{l}
Вариант Командир крысиной гвардии \tabularnewline
Вычислите интеграл:\tabularnewline
$\int \frac{\dx}{3 \ch x + 5 \sh x + 3}$\tabularnewline
\noalign{\vskip4mm}

Вычислите интеграл:\tabularnewline
$\int \frac{(x + 1)dx}{(2x^2 + 1) \sqrt{x^2 + 1}}$\tabularnewline
\noalign{\vskip4mm}

Вычислите площадь фигуры, ограниченной кривыми:\tabularnewline
$x^2 +y^2 = 2 \text{ и } y^2 = 2x - 1$\tabularnewline
\noalign{\vskip4mm}

Вычислите объем тела вращения данной кривой вокруг оси Ox:\tabularnewline
$x = \cos^3 t, y = \sin^3 t, t\in[0; 2\pi]$\tabularnewline
\noalign{\vskip4mm}

\end{tabular}& %
\begin{tabular}{l}
Вариант Боец крысиной гвардии \tabularnewline
Вычислите интеграл:\tabularnewline
$\int \frac{\dx}{3 \ch x + 5 \sh x + 3}$\tabularnewline
\noalign{\vskip4mm}

Вычислите интеграл:\tabularnewline
$\int \frac{x^4 - 2x^2 + 2}{(x^2 - 2x + 2)^2}\dx$\tabularnewline
\noalign{\vskip4mm}

Вычислите площадь фигуры, ограниченной кривыми:\tabularnewline
$r = 2\sqrt{\phi\arccos(\phi^2 - 1)}, \text{ }, \phi = 1 \text{ и } \phi = \sqrt{\frac{3}{2}}$\tabularnewline
\noalign{\vskip4mm}

Вычислите длину кривой:\tabularnewline
$x = (1 - \cos 2t), y = \sin t - \frac{\sin 3t}{3}, t \in [0; \frac{\pi}{2}]$\tabularnewline
\noalign{\vskip4mm}

\end{tabular}\tabularnewline
\noalign{\vskip4mm}
\end{tabular}

\begin{tabular}{cc}
\begin{tabular}{l}
Вариант Демон из Плавильни \tabularnewline
Вычислите интеграл:\tabularnewline
$\int \frac{\dx}{3 \ch x + 5 \sh x + 3}$\tabularnewline
\noalign{\vskip4mm}

Вычислите интеграл:\tabularnewline
$\int \frac{x^4 - 2x^2 + 2}{(x^2 - 2x + 2)^2}\dx$\tabularnewline
\noalign{\vskip4mm}

Вычислите площадь фигуры, ограниченной кривыми:\tabularnewline
$r = 2\sqrt{\phi\arccos(\phi^2 - 1)}, \text{ }, \phi = 1 \text{ и } \phi = \sqrt{\frac{3}{2}}$\tabularnewline
\noalign{\vskip4mm}

Вычислите длину кривой:\tabularnewline
$r = \th \frac{\phi}{2}, \phi \in [0; \phi_0]$\tabularnewline
\noalign{\vskip4mm}

\end{tabular}& %
\begin{tabular}{l}
Вариант Древний Дракон \tabularnewline
Вычислите интеграл:\tabularnewline
$\int \left(\frac{\ln x}{x} \right)^3 \dx$\tabularnewline
\noalign{\vskip4mm}

Вычислите интеграл:\tabularnewline
$\int \frac{(x + 1)dx}{(2x^2 + 1) \sqrt{x^2 + 1}}$\tabularnewline
\noalign{\vskip4mm}

Вычислите площадь фигуры, ограниченной кривыми:\tabularnewline
$x = \cos^3 t, y = \sin^3 t$\tabularnewline
\noalign{\vskip4mm}

Вычислите длину петли кривой:\tabularnewline
$r = \frac{1}{\sin^2 (\phi / 3)}, \phi \in [0; 3\pi]$\tabularnewline
\noalign{\vskip4mm}

\end{tabular}\tabularnewline
\noalign{\vskip4mm}
\begin{tabular}{l}
Вариант Вендрик \tabularnewline
Вычислите интеграл:\tabularnewline
$\int \frac{\cos^3 x}{\sqrt[5]{\sin x}}\dx$\tabularnewline
\noalign{\vskip4mm}

Вычислите интеграл:\tabularnewline
$\int \frac{\dx}{(x - 1)^3 \sqrt{x^2 - 2x - 1}}$\tabularnewline
\noalign{\vskip4mm}

Вычислите площадь фигуры, ограниченной кривыми:\tabularnewline
$r = 2\sqrt{\phi\arccos(\phi^2 - 1)}, \text{ }, \phi = 1 \text{ и } \phi = \sqrt{\frac{3}{2}}$\tabularnewline
\noalign{\vskip4mm}

Вычислите длину петли кривой:\tabularnewline
$r = \frac{1}{\sin^2 (\phi / 3)}, \phi \in [0; 3\pi]$\tabularnewline
\noalign{\vskip4mm}

\end{tabular}& %
\begin{tabular}{l}
Вариант Алдия, ученый Первородного Греха \tabularnewline
Вычислите интеграл:\tabularnewline
$\int \frac{\dx}{3 \ch x + 5 \sh x + 3}$\tabularnewline
\noalign{\vskip4mm}

Вычислите интеграл:\tabularnewline
$\int \frac{\dx}{(x - 1)^3 \sqrt{x^2 - 2x - 1}}$\tabularnewline
\noalign{\vskip4mm}

Вычислите площадь фигуры, ограниченной кривыми:\tabularnewline
$x = \cos^3 t, y = \sin^3 t$\tabularnewline
\noalign{\vskip4mm}

Вычислите объем тела вращения плоской фигуры, ограниченной данными кривыми, вокруг оси Oy:\tabularnewline
$y = e^x + 6, y = e^{2x}, x = 0$ \tabularnewline
\noalign{\vskip4mm}

\end{tabular}\tabularnewline
\noalign{\vskip4mm}
\begin{tabular}{l}
Вариант Варг, Церах и Разорительница Гробниц \tabularnewline
Вычислите интеграл:\tabularnewline
$\int \cos \ln x \dx$\tabularnewline
\noalign{\vskip4mm}

Вычислите интеграл:\tabularnewline
$\int \frac{\dx}{(x - 1)^3 \sqrt{x^2 - 2x - 1}}$\tabularnewline
\noalign{\vskip4mm}

Вычислите площадь фигуры, ограниченной кривыми:\tabularnewline
$x^2 +y^2 = 2 \text{ и } y^2 = 2x - 1$\tabularnewline
\noalign{\vskip4mm}

Вычислите длину петли кривой:\tabularnewline
$r = \frac{1}{\sin^2 (\phi / 3)}, \phi \in [0; 3\pi]$\tabularnewline
\noalign{\vskip4mm}

\end{tabular}& %
\begin{tabular}{l}
Вариант Скверная королева Элана \tabularnewline
Вычислите интеграл:\tabularnewline
$\int \cos \ln x \dx$\tabularnewline
\noalign{\vskip4mm}

Вычислите интеграл:\tabularnewline
$\int \frac{\dx}{(x - 1)^3 \sqrt{x^2 - 2x - 1}}$\tabularnewline
\noalign{\vskip4mm}

Вычислите площадь фигуры, ограниченной кривыми:\tabularnewline
$x^2 +y^2 = 2 \text{ и } y^2 = 2x - 1$\tabularnewline
\noalign{\vskip4mm}

Вычислите объем тела вращения данной кривой вокруг оси Ox:\tabularnewline
$x = \cos^3 t, y = \sin^3 t, t\in[0; 2\pi]$\tabularnewline
\noalign{\vskip4mm}

\end{tabular}\tabularnewline
\noalign{\vskip4mm}
\end{tabular}

\begin{tabular}{cc}
\begin{tabular}{l}
Вариант Син Дремлющий дракон \tabularnewline
Вычислите интеграл:\tabularnewline
$\int \frac{\dx}{3 \ch x + 5 \sh x + 3}$\tabularnewline
\noalign{\vskip4mm}

Вычислите интеграл:\tabularnewline
$\int \frac{\dx}{(x - 1)^3 \sqrt{x^2 - 2x - 1}}$\tabularnewline
\noalign{\vskip4mm}

Вычислите площадь фигуры, ограниченной кривыми:\tabularnewline
$x^2 +y^2 = 2 \text{ и } y^2 = 2x - 1$\tabularnewline
\noalign{\vskip4mm}

Вычислите длину кривой:\tabularnewline
$r = \th \frac{\phi}{2}, \phi \in [0; \phi_0]$\tabularnewline
\noalign{\vskip4mm}

\end{tabular}& %
\begin{tabular}{l}
Вариант Дымный рыцарь \tabularnewline
Выразите через $Li(x)$:\tabularnewline
$\int \frac{x^{100} \dx}{\ln x}$\tabularnewline
\noalign{\vskip4mm}

Вычислите интеграл:\tabularnewline
$\int \frac{(x + 1)dx}{(2x^2 + 1) \sqrt{x^2 + 1}}$\tabularnewline
\noalign{\vskip4mm}

Вычислите площадь фигуры, ограниченной кривыми:\tabularnewline
$r = 2\sqrt{\phi\arccos(\phi^2 - 1)}, \text{ }, \phi = 1 \text{ и } \phi = \sqrt{\frac{3}{2}}$\tabularnewline
\noalign{\vskip4mm}

Вычислите длину петли кривой:\tabularnewline
$r = \frac{1}{\sin^2 (\phi / 3)}, \phi \in [0; 3\pi]$\tabularnewline
\noalign{\vskip4mm}

\end{tabular}\tabularnewline
\noalign{\vskip4mm}
\begin{tabular}{l}
Вариант Сэр Алонн \tabularnewline
Вычислите интеграл:\tabularnewline
$\int \frac{\cos^3 x}{\sqrt[5]{\sin x}}\dx$\tabularnewline
\noalign{\vskip4mm}

Вычислите интеграл:\tabularnewline
$\int \frac{(x + 1)dx}{(2x^2 + 1) \sqrt{x^2 + 1}}$\tabularnewline
\noalign{\vskip4mm}

Вычислите площадь фигуры, ограниченной кривыми:\tabularnewline
$x = \cos^3 t, y = \sin^3 t$\tabularnewline
\noalign{\vskip4mm}

Вычислите длину кривой:\tabularnewline
$x = (1 - \cos 2t), y = \sin t - \frac{\sin 3t}{3}, t \in [0; \frac{\pi}{2}]$\tabularnewline
\noalign{\vskip4mm}

\end{tabular}& %
\begin{tabular}{l}
Вариант Старый Демон из Плавильни \tabularnewline
Вычислите интеграл:\tabularnewline
$\int \frac{\dx}{3 \ch x + 5 \sh x + 3}$\tabularnewline
\noalign{\vskip4mm}

Вычислите интеграл:\tabularnewline
$\int \frac{\dx}{x^3 \sqrt[3]{2 - x^3}}$\tabularnewline
\noalign{\vskip4mm}

Вычислите площадь фигуры, ограниченной кривыми:\tabularnewline
$x^2 +y^2 = 2 \text{ и } y^2 = 2x - 1$\tabularnewline
\noalign{\vskip4mm}

Вычислите объем тела вращения плоской фигуры, ограниченной данными кривыми, вокруг оси Oy:\tabularnewline
$y = e^x + 6, y = e^{2x}, x = 0$ \tabularnewline
\noalign{\vskip4mm}

\end{tabular}\tabularnewline
\noalign{\vskip4mm}
\begin{tabular}{l}
Вариант Аава, питомец короля \tabularnewline
Вычислите интеграл:\tabularnewline
$\int \cos x \ln (1 + \sin^2 x) \dx$\tabularnewline
\noalign{\vskip4mm}

Вычислите интеграл:\tabularnewline
$\int \frac{(x + 1)dx}{(2x^2 + 1) \sqrt{x^2 + 1}}$\tabularnewline
\noalign{\vskip4mm}

Вычислите площадь фигуры, ограниченной кривыми:\tabularnewline
$r = 2\sqrt{\phi\arccos(\phi^2 - 1)}, \text{ }, \phi = 1 \text{ и } \phi = \sqrt{\frac{3}{2}}$\tabularnewline
\noalign{\vskip4mm}

Вычислите длину кривой:\tabularnewline
$x = (1 - \cos 2t), y = \sin t - \frac{\sin 3t}{3}, t \in [0; \frac{\pi}{2}]$\tabularnewline
\noalign{\vskip4mm}

\end{tabular}& %
\begin{tabular}{l}
Вариант Луд и Заллен \tabularnewline
Вычислите интеграл:\tabularnewline
$\int \cos x \ln (1 + \sin^2 x) \dx$\tabularnewline
\noalign{\vskip4mm}

Вычислите интеграл:\tabularnewline
$\int \frac{(x + 1)dx}{(2x^2 + 1) \sqrt{x^2 + 1}}$\tabularnewline
\noalign{\vskip4mm}

Вычислите площадь фигуры, ограниченной кривыми:\tabularnewline
$x^2 +y^2 = 2 \text{ и } y^2 = 2x - 1$\tabularnewline
\noalign{\vskip4mm}

Вычислите длину кривой:\tabularnewline
$r = \th \frac{\phi}{2}, \phi \in [0; \phi_0]$\tabularnewline
\noalign{\vskip4mm}

\end{tabular}\tabularnewline
\noalign{\vskip4mm}
\end{tabular}

\begin{tabular}{cc}
\begin{tabular}{l}
Вариант Сгоревший Король Слоновой Кости \tabularnewline
Вычислите интеграл:\tabularnewline
$\int \left(\frac{\ln x}{x} \right)^3 \dx$\tabularnewline
\noalign{\vskip4mm}

Вычислите интеграл:\tabularnewline
$\int \frac{(x + 1)dx}{(2x^2 + 1) \sqrt{x^2 + 1}}$\tabularnewline
\noalign{\vskip4mm}

Вычислите площадь фигуры, ограниченной кривыми:\tabularnewline
$x = \cos^3 t, y = \sin^3 t$\tabularnewline
\noalign{\vskip4mm}

Вычислите длину петли кривой:\tabularnewline
$r = \frac{1}{\sin^2 (\phi / 3)}, \phi \in [0; 3\pi]$\tabularnewline
\noalign{\vskip4mm}

\end{tabular}& %
\begin{tabular}{l}
Вариант Судия Гундир \tabularnewline
Вычислите интеграл:\tabularnewline
$\int \cos x \ln (1 + \sin^2 x) \dx$\tabularnewline
\noalign{\vskip4mm}

Вычислите интеграл:\tabularnewline
$\int \frac{x^4 - 2x^2 + 2}{(x^2 - 2x + 2)^2}\dx$\tabularnewline
\noalign{\vskip4mm}

Вычислите площадь фигуры, ограниченной кривыми:\tabularnewline
$x^2 +y^2 = 2 \text{ и } y^2 = 2x - 1$\tabularnewline
\noalign{\vskip4mm}

Вычислите объем тела вращения данной кривой вокруг оси Ox:\tabularnewline
$x = \cos^3 t, y = \sin^3 t, t\in[0; 2\pi]$\tabularnewline
\noalign{\vskip4mm}

\end{tabular}\tabularnewline
\noalign{\vskip4mm}
\begin{tabular}{l}
Вариант Вордт из Холодной долины \tabularnewline
Выразите через $Li(x)$:\tabularnewline
$\int \frac{x^{100} \dx}{\ln x}$\tabularnewline
\noalign{\vskip4mm}

Вычислите интеграл:\tabularnewline
$\int \frac{(x + 1)dx}{(2x^2 + 1) \sqrt{x^2 + 1}}$\tabularnewline
\noalign{\vskip4mm}

Вычислите площадь фигуры, ограниченной кривыми:\tabularnewline
$x^2 +y^2 = 2 \text{ и } y^2 = 2x - 1$\tabularnewline
\noalign{\vskip4mm}

Вычислите длину петли кривой:\tabularnewline
$r = \frac{1}{\sin^2 (\phi / 3)}, \phi \in [0; 3\pi]$\tabularnewline
\noalign{\vskip4mm}

\end{tabular}& %
\begin{tabular}{l}
Вариант Знаток кристальных чар \tabularnewline
Выразите через $Li(x)$:\tabularnewline
$\int \frac{x^{100} \dx}{\ln x}$\tabularnewline
\noalign{\vskip4mm}

Вычислите интеграл:\tabularnewline
$\int \frac{\dx}{x^3 \sqrt[3]{2 - x^3}}$\tabularnewline
\noalign{\vskip4mm}

Вычислите площадь фигуры, ограниченной кривыми:\tabularnewline
$x = \cos^3 t, y = \sin^3 t$\tabularnewline
\noalign{\vskip4mm}

Вычислите объем тела вращения плоской фигуры, ограниченной данными кривыми, вокруг оси Oy:\tabularnewline
$y = e^x + 6, y = e^{2x}, x = 0$ \tabularnewline
\noalign{\vskip4mm}

\end{tabular}\tabularnewline
\noalign{\vskip4mm}
\begin{tabular}{l}
Вариант Дьяконы глубин \tabularnewline
Вычислите интеграл:\tabularnewline
$\int \left(\frac{\ln x}{x} \right)^3 \dx$\tabularnewline
\noalign{\vskip4mm}

Вычислите интеграл:\tabularnewline
$\int \frac{\dx}{(x - 1)^3 \sqrt{x^2 - 2x - 1}}$\tabularnewline
\noalign{\vskip4mm}

Вычислите площадь фигуры, ограниченной кривыми:\tabularnewline
$x^2 +y^2 = 2 \text{ и } y^2 = 2x - 1$\tabularnewline
\noalign{\vskip4mm}

Вычислите длину кривой:\tabularnewline
$r = \th \frac{\phi}{2}, \phi \in [0; \phi_0]$\tabularnewline
\noalign{\vskip4mm}

\end{tabular}& %
\begin{tabular}{l}
Вариант Хранители Бездны \tabularnewline
Вычислите интеграл:\tabularnewline
$\int \cos x \ln (1 + \sin^2 x) \dx$\tabularnewline
\noalign{\vskip4mm}

Вычислите интеграл:\tabularnewline
$\int \frac{(x + 1)dx}{(2x^2 + 1) \sqrt{x^2 + 1}}$\tabularnewline
\noalign{\vskip4mm}

Вычислите площадь фигуры, ограниченной кривыми:\tabularnewline
$x^2 +y^2 = 2 \text{ и } y^2 = 2x - 1$\tabularnewline
\noalign{\vskip4mm}

Вычислите объем тела вращения данной кривой вокруг оси Ox:\tabularnewline
$x = \cos^3 t, y = \sin^3 t, t\in[0; 2\pi]$\tabularnewline
\noalign{\vskip4mm}

\end{tabular}\tabularnewline
\noalign{\vskip4mm}
\end{tabular}

\begin{tabular}{cc}
\begin{tabular}{l}
Вариант Верховный повелитель Вольнир \tabularnewline
Вычислите интеграл:\tabularnewline
$\int \left(\frac{\ln x}{x} \right)^3 \dx$\tabularnewline
\noalign{\vskip4mm}

Вычислите интеграл:\tabularnewline
$\int \frac{(x + 1)dx}{(2x^2 + 1) \sqrt{x^2 + 1}}$\tabularnewline
\noalign{\vskip4mm}

Вычислите площадь фигуры, ограниченной кривыми:\tabularnewline
$x^2 +y^2 = 2 \text{ и } y^2 = 2x - 1$\tabularnewline
\noalign{\vskip4mm}

Вычислите объем тела вращения плоской фигуры, ограниченной данными кривыми, вокруг оси Oy:\tabularnewline
$y = e^x + 6, y = e^{2x}, x = 0$ \tabularnewline
\noalign{\vskip4mm}

\end{tabular}& %
\begin{tabular}{l}
Вариант Понтифик Саливан \tabularnewline
Вычислите интеграл:\tabularnewline
$\int \frac{\cos^3 x}{\sqrt[5]{\sin x}}\dx$\tabularnewline
\noalign{\vskip4mm}

Вычислите интеграл:\tabularnewline
$\int \frac{\dx}{(x - 1)^3 \sqrt{x^2 - 2x - 1}}$\tabularnewline
\noalign{\vskip4mm}

Вычислите площадь фигуры, ограниченной кривыми:\tabularnewline
$r = 2\sqrt{\phi\arccos(\phi^2 - 1)}, \text{ }, \phi = 1 \text{ и } \phi = \sqrt{\frac{3}{2}}$\tabularnewline
\noalign{\vskip4mm}

Вычислите длину кривой:\tabularnewline
$r = \th \frac{\phi}{2}, \phi \in [0; \phi_0]$\tabularnewline
\noalign{\vskip4mm}

\end{tabular}\tabularnewline
\noalign{\vskip4mm}
\begin{tabular}{l}
Вариант Олдрик, пожиратель богов \tabularnewline
Вычислите интеграл:\tabularnewline
$\int \cos x \ln (1 + \sin^2 x) \dx$\tabularnewline
\noalign{\vskip4mm}

Вычислите интеграл:\tabularnewline
$\int \frac{(x + 1)dx}{(2x^2 + 1) \sqrt{x^2 + 1}}$\tabularnewline
\noalign{\vskip4mm}

Вычислите площадь фигуры, ограниченной кривыми:\tabularnewline
$x^2 +y^2 = 2 \text{ и } y^2 = 2x - 1$\tabularnewline
\noalign{\vskip4mm}

Вычислите объем тела вращения данной кривой вокруг оси Ox:\tabularnewline
$x = \cos^3 t, y = \sin^3 t, t\in[0; 2\pi]$\tabularnewline
\noalign{\vskip4mm}

\end{tabular}& %
\begin{tabular}{l}
Вариант Гигант Йорм \tabularnewline
Вычислите интеграл:\tabularnewline
$\int \frac{\cos^3 x}{\sqrt[5]{\sin x}}\dx$\tabularnewline
\noalign{\vskip4mm}

Вычислите интеграл:\tabularnewline
$\int \frac{(x + 1)dx}{(2x^2 + 1) \sqrt{x^2 + 1}}$\tabularnewline
\noalign{\vskip4mm}

Вычислите площадь фигуры, ограниченной кривыми:\tabularnewline
$r = 2\sqrt{\phi\arccos(\phi^2 - 1)}, \text{ }, \phi = 1 \text{ и } \phi = \sqrt{\frac{3}{2}}$\tabularnewline
\noalign{\vskip4mm}

Вычислите объем тела вращения данной кривой вокруг оси Ox:\tabularnewline
$x = \cos^3 t, y = \sin^3 t, t\in[0; 2\pi]$\tabularnewline
\noalign{\vskip4mm}

\end{tabular}\tabularnewline
\noalign{\vskip4mm}
\begin{tabular}{l}
Вариант Танцовщица Холодной долины \tabularnewline
Вычислите интеграл:\tabularnewline
$\int \frac{\dx}{3 \ch x + 5 \sh x + 3}$\tabularnewline
\noalign{\vskip4mm}

Вычислите интеграл:\tabularnewline
$\int \frac{x^4 - 2x^2 + 2}{(x^2 - 2x + 2)^2}\dx$\tabularnewline
\noalign{\vskip4mm}

Вычислите площадь фигуры, ограниченной кривыми:\tabularnewline
$r = 2\sqrt{\phi\arccos(\phi^2 - 1)}, \text{ }, \phi = 1 \text{ и } \phi = \sqrt{\frac{3}{2}}$\tabularnewline
\noalign{\vskip4mm}

Вычислите объем тела вращения плоской фигуры, ограниченной данными кривыми, вокруг оси Oy:\tabularnewline
$y = e^x + 6, y = e^{2x}, x = 0$ \tabularnewline
\noalign{\vskip4mm}

\end{tabular}& %
\begin{tabular}{l}
Вариант Доспехи драконоборца \tabularnewline
Вычислите интеграл:\tabularnewline
$\int \left(\frac{\ln x}{x} \right)^3 \dx$\tabularnewline
\noalign{\vskip4mm}

Вычислите интеграл:\tabularnewline
$\int \frac{\dx}{x^3 \sqrt[3]{2 - x^3}}$\tabularnewline
\noalign{\vskip4mm}

Вычислите площадь фигуры, ограниченной кривыми:\tabularnewline
$r = 2\sqrt{\phi\arccos(\phi^2 - 1)}, \text{ }, \phi = 1 \text{ и } \phi = \sqrt{\frac{3}{2}}$\tabularnewline
\noalign{\vskip4mm}

Вычислите длину петли кривой:\tabularnewline
$r = \frac{1}{\sin^2 (\phi / 3)}, \phi \in [0; 3\pi]$\tabularnewline
\noalign{\vskip4mm}

\end{tabular}\tabularnewline
\noalign{\vskip4mm}
\end{tabular}

\begin{tabular}{cc}
\begin{tabular}{l}
Вариант Лотрик, младший принц и Лориан, старший принц \tabularnewline
Вычислите интеграл:\tabularnewline
$\int \frac{\cos^3 x}{\sqrt[5]{\sin x}}\dx$\tabularnewline
\noalign{\vskip4mm}

Вычислите интеграл:\tabularnewline
$\int \frac{\dx}{(x - 1)^3 \sqrt{x^2 - 2x - 1}}$\tabularnewline
\noalign{\vskip4mm}

Вычислите площадь фигуры, ограниченной кривыми:\tabularnewline
$x = \cos^3 t, y = \sin^3 t$\tabularnewline
\noalign{\vskip4mm}

Вычислите длину кривой:\tabularnewline
$x = (1 - \cos 2t), y = \sin t - \frac{\sin 3t}{3}, t \in [0; \frac{\pi}{2}]$\tabularnewline
\noalign{\vskip4mm}

\end{tabular}& %
\begin{tabular}{l}
Вариант Душа пепла \tabularnewline
Выразите через $Li(x)$:\tabularnewline
$\int \frac{x^{100} \dx}{\ln x}$\tabularnewline
\noalign{\vskip4mm}

Вычислите интеграл:\tabularnewline
$\int \frac{\dx}{x^3 \sqrt[3]{2 - x^3}}$\tabularnewline
\noalign{\vskip4mm}

Вычислите площадь фигуры, ограниченной кривыми:\tabularnewline
$x^2 +y^2 = 2 \text{ и } y^2 = 2x - 1$\tabularnewline
\noalign{\vskip4mm}

Вычислите длину петли кривой:\tabularnewline
$r = \frac{1}{\sin^2 (\phi / 3)}, \phi \in [0; 3\pi]$\tabularnewline
\noalign{\vskip4mm}

\end{tabular}\tabularnewline
\noalign{\vskip4mm}
\end{tabular}



\end{document}