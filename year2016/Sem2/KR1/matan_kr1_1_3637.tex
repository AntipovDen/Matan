
\documentclass[russian]{article}
\usepackage[T2A,T1]{fontenc}
\usepackage[utf8]{inputenc}
\usepackage[a4paper]{geometry}
\geometry{verbose,tmargin=1cm,bmargin=0cm,lmargin=0cm,rmargin=0cm}
\usepackage{amsmath}

\makeatletter

\DeclareRobustCommand{\cyrtext}{%
  \fontencoding{T2A}\selectfont\def\encodingdefault{T2A}}
\DeclareRobustCommand{\textcyr}[1]{\leavevmode{\cyrtext #1}}
\AtBeginDocument{\DeclareFontEncoding{T2A}{}{}}


\makeatother

\usepackage[russian]{babel}
\begin{document}
 
\begin{tabular}{cc}
\begin{tabular}{l}
Вариант Валькирия \tabularnewline
Постройте график в декартовых координатах:\tabularnewline
$y = \tg\frac{\pi}{x^2}$\tabularnewline
\noalign{\vskip4mm}

Докажите предел по определению:\tabularnewline
$\lim\limits_{n \to +\infty} \frac{n(2^{-n} + 1)}{2n  + 1} = \frac{1}{2}$\tabularnewline
\noalign{\vskip4mm}

Докажите, что последовательность сходится:\tabularnewline
$x_n = \prod_{k = 1}^{n}(1 + 2^{-k})$\tabularnewline
\noalign{\vskip4mm}

Вычислите предел функции:\tabularnewline
$\lim\limits_{x \to 0} \frac{\sqrt{1 + \tg x} - \sqrt{1 + \sin x}}{x ^ 3}$\tabularnewline
\noalign{\vskip4mm}

\end{tabular}& %
\begin{tabular}{l}
Вариант Крейсер \tabularnewline
Постройте график в полярных координатах:\tabularnewline
$r = \frac{1}{\phi^2}$\tabularnewline
\noalign{\vskip4mm}

Докажите предел по определению:\tabularnewline
$\lim\limits_{n \to +\infty} \frac{n(2^{-n} + 1)}{2n  + 1} = \frac{1}{2}$\tabularnewline
\noalign{\vskip4mm}

Докажите, что последовательность сходится:\tabularnewline
$x_n = \prod_{k = 1}^{n}(1 + 2^{-k})$\tabularnewline
\noalign{\vskip4mm}

Вычислите предел функции:\tabularnewline
$\lim\limits_{x \to 0} \frac{\sqrt{1 + \tg x} - \sqrt{1 + \sin x}}{x ^ 3}$\tabularnewline
\noalign{\vskip4mm}

\end{tabular}\tabularnewline
\noalign{\vskip4mm}
\begin{tabular}{l}
Вариант Научное судно \tabularnewline
Постройте график в декартовых координатах:\tabularnewline
$y = \tg\frac{\pi}{x^2}$\tabularnewline
\noalign{\vskip4mm}

Докажите предел по определению:\tabularnewline
$\lim\limits_{n \to +\infty} \frac{\ln(n + 1)}{\ln(n^2 + 1)} = \frac{1}{2}$\tabularnewline
\noalign{\vskip4mm}

Докажите, что последовательность сходится:\tabularnewline
$x_n = \prod_{k = 1}^{n}(1 + 2^{-k})$\tabularnewline
\noalign{\vskip4mm}

Вычислите предел функции:\tabularnewline
$\lim\limits_{x \to 0} \frac{\sqrt{1 + \tg x} - \sqrt{1 + \sin x}}{x ^ 3}$\tabularnewline
\noalign{\vskip4mm}

\end{tabular}& %
\begin{tabular}{l}
Вариант Осадный танк \tabularnewline
Постройте график в полярных координатах:\tabularnewline
$r = \frac{1}{\phi^2}$\tabularnewline
\noalign{\vskip4mm}

Докажите предел по определению:\tabularnewline
$\lim\limits_{n \to +\infty} \frac{\ln(n + 1)}{\ln(n^2 + 1)} = \frac{1}{2}$\tabularnewline
\noalign{\vskip4mm}

Докажите, что последовательность сходится:\tabularnewline
$x_n = \prod_{k = 1}^{n}(1 + 2^{-k})$\tabularnewline
\noalign{\vskip4mm}

Вычислите предел функции:\tabularnewline
$\lim\limits_{x \to 0} \frac{\sqrt{1 + \tg x} - \sqrt{1 + \sin x}}{x ^ 3}$\tabularnewline
\noalign{\vskip4mm}

\end{tabular}\tabularnewline
\noalign{\vskip4mm}
\begin{tabular}{l}
Вариант Призрак \tabularnewline
Постройте график в декартовых координатах:\tabularnewline
$y = \tg\frac{\pi}{x^2}$\tabularnewline
\noalign{\vskip4mm}

Докажите предел по определению:\tabularnewline
$\lim\limits_{n \to +\infty} \frac{n(2^{-n} + 1)}{2n  + 1} = \frac{1}{2}$\tabularnewline
\noalign{\vskip4mm}

Докажите, что последовательность расходится:\tabularnewline
$x_n = \sum_{k = 2}^n \frac{1}{\sqrt{k} \ln k}$\tabularnewline
\noalign{\vskip4mm}

Вычислите предел функции:\tabularnewline
$\lim\limits_{x \to 0} \frac{\sqrt{1 + \tg x} - \sqrt{1 + \sin x}}{x ^ 3}$\tabularnewline
\noalign{\vskip4mm}

\end{tabular}& %
\begin{tabular}{l}
Вариант Стелс \tabularnewline
Постройте график в полярных координатах:\tabularnewline
$r = \frac{1}{\phi^2}$\tabularnewline
\noalign{\vskip4mm}

Докажите предел по определению:\tabularnewline
$\lim\limits_{n \to +\infty} \frac{n(2^{-n} + 1)}{2n  + 1} = \frac{1}{2}$\tabularnewline
\noalign{\vskip4mm}

Докажите, что последовательность расходится:\tabularnewline
$x_n = \sum_{k = 2}^n \frac{1}{\sqrt{k} \ln k}$\tabularnewline
\noalign{\vskip4mm}

Вычислите предел функции:\tabularnewline
$\lim\limits_{x \to 0} \frac{\sqrt{1 + \tg x} - \sqrt{1 + \sin x}}{x ^ 3}$\tabularnewline
\noalign{\vskip4mm}

\end{tabular}\tabularnewline
\noalign{\vskip4mm}
\begin{tabular}{l}
Вариант Люркер \tabularnewline
Постройте график в декартовых координатах:\tabularnewline
$y = \tg\frac{\pi}{x^2}$\tabularnewline
\noalign{\vskip4mm}

Докажите предел по определению:\tabularnewline
$\lim\limits_{n \to +\infty} \frac{\ln(n + 1)}{\ln(n^2 + 1)} = \frac{1}{2}$\tabularnewline
\noalign{\vskip4mm}

Докажите, что последовательность расходится:\tabularnewline
$x_n = \sum_{k = 2}^n \frac{1}{\sqrt{k} \ln k}$\tabularnewline
\noalign{\vskip4mm}

Вычислите предел функции:\tabularnewline
$\lim\limits_{x \to 0} \frac{\sqrt{1 + \tg x} - \sqrt{1 + \sin x}}{x ^ 3}$\tabularnewline
\noalign{\vskip4mm}

\end{tabular}& %
\begin{tabular}{l}
Вариант Дэфайлер \tabularnewline
Постройте график в полярных координатах:\tabularnewline
$r = \frac{1}{\phi^2}$\tabularnewline
\noalign{\vskip4mm}

Докажите предел по определению:\tabularnewline
$\lim\limits_{n \to +\infty} \frac{\ln(n + 1)}{\ln(n^2 + 1)} = \frac{1}{2}$\tabularnewline
\noalign{\vskip4mm}

Докажите, что последовательность расходится:\tabularnewline
$x_n = \sum_{k = 2}^n \frac{1}{\sqrt{k} \ln k}$\tabularnewline
\noalign{\vskip4mm}

Вычислите предел функции:\tabularnewline
$\lim\limits_{x \to 0} \frac{\sqrt{1 + \tg x} - \sqrt{1 + \sin x}}{x ^ 3}$\tabularnewline
\noalign{\vskip4mm}

\end{tabular}\tabularnewline
\noalign{\vskip4mm}
\end{tabular}

\begin{tabular}{cc}
\begin{tabular}{l}
Вариант Ультралиск \tabularnewline
Постройте график в декартовых координатах:\tabularnewline
$y = \tg\frac{\pi}{x^2}$\tabularnewline
\noalign{\vskip4mm}

Докажите предел по определению:\tabularnewline
$\lim\limits_{n \to +\infty} \frac{n(2^{-n} + 1)}{2n  + 1} = \frac{1}{2}$\tabularnewline
\noalign{\vskip4mm}

Докажите, что последовательность сходится:\tabularnewline
$x_n = \prod_{k = 1}^{n}(1 + 2^{-k})$\tabularnewline
\noalign{\vskip4mm}

Вычислите предел функции:\tabularnewline
$\lim\limits_{x \to 0} x^{\frac{2x}{2x + 1}}$\tabularnewline
\noalign{\vskip4mm}

\end{tabular}& %
\begin{tabular}{l}
Вариант Управитель \tabularnewline
Постройте график в полярных координатах:\tabularnewline
$r = \frac{1}{\phi^2}$\tabularnewline
\noalign{\vskip4mm}

Докажите предел по определению:\tabularnewline
$\lim\limits_{n \to +\infty} \frac{n(2^{-n} + 1)}{2n  + 1} = \frac{1}{2}$\tabularnewline
\noalign{\vskip4mm}

Докажите, что последовательность сходится:\tabularnewline
$x_n = \prod_{k = 1}^{n}(1 + 2^{-k})$\tabularnewline
\noalign{\vskip4mm}

Вычислите предел функции:\tabularnewline
$\lim\limits_{x \to 0} x^{\frac{2x}{2x + 1}}$\tabularnewline
\noalign{\vskip4mm}

\end{tabular}\tabularnewline
\noalign{\vskip4mm}
\begin{tabular}{l}
Вариант Королева \tabularnewline
Постройте график в декартовых координатах:\tabularnewline
$y = \tg\frac{\pi}{x^2}$\tabularnewline
\noalign{\vskip4mm}

Докажите предел по определению:\tabularnewline
$\lim\limits_{n \to +\infty} \frac{\ln(n + 1)}{\ln(n^2 + 1)} = \frac{1}{2}$\tabularnewline
\noalign{\vskip4mm}

Докажите, что последовательность сходится:\tabularnewline
$x_n = \prod_{k = 1}^{n}(1 + 2^{-k})$\tabularnewline
\noalign{\vskip4mm}

Вычислите предел функции:\tabularnewline
$\lim\limits_{x \to 0} x^{\frac{2x}{2x + 1}}$\tabularnewline
\noalign{\vskip4mm}

\end{tabular}& %
\begin{tabular}{l}
Вариант Темный Темплар \tabularnewline
Постройте график в полярных координатах:\tabularnewline
$r = \frac{1}{\phi^2}$\tabularnewline
\noalign{\vskip4mm}

Докажите предел по определению:\tabularnewline
$\lim\limits_{n \to +\infty} \frac{\ln(n + 1)}{\ln(n^2 + 1)} = \frac{1}{2}$\tabularnewline
\noalign{\vskip4mm}

Докажите, что последовательность сходится:\tabularnewline
$x_n = \prod_{k = 1}^{n}(1 + 2^{-k})$\tabularnewline
\noalign{\vskip4mm}

Вычислите предел функции:\tabularnewline
$\lim\limits_{x \to 0} x^{\frac{2x}{2x + 1}}$\tabularnewline
\noalign{\vskip4mm}

\end{tabular}\tabularnewline
\noalign{\vskip4mm}
\begin{tabular}{l}
Вариант Архон \tabularnewline
Постройте график в декартовых координатах:\tabularnewline
$y = \tg\frac{\pi}{x^2}$\tabularnewline
\noalign{\vskip4mm}

Докажите предел по определению:\tabularnewline
$\lim\limits_{n \to +\infty} \frac{n(2^{-n} + 1)}{2n  + 1} = \frac{1}{2}$\tabularnewline
\noalign{\vskip4mm}

Докажите, что последовательность расходится:\tabularnewline
$x_n = \sum_{k = 2}^n \frac{1}{\sqrt{k} \ln k}$\tabularnewline
\noalign{\vskip4mm}

Вычислите предел функции:\tabularnewline
$\lim\limits_{x \to 0} x^{\frac{2x}{2x + 1}}$\tabularnewline
\noalign{\vskip4mm}

\end{tabular}& %
\begin{tabular}{l}
Вариант Ривер \tabularnewline
Постройте график в полярных координатах:\tabularnewline
$r = \frac{1}{\phi^2}$\tabularnewline
\noalign{\vskip4mm}

Докажите предел по определению:\tabularnewline
$\lim\limits_{n \to +\infty} \frac{n(2^{-n} + 1)}{2n  + 1} = \frac{1}{2}$\tabularnewline
\noalign{\vskip4mm}

Докажите, что последовательность расходится:\tabularnewline
$x_n = \sum_{k = 2}^n \frac{1}{\sqrt{k} \ln k}$\tabularnewline
\noalign{\vskip4mm}

Вычислите предел функции:\tabularnewline
$\lim\limits_{x \to 0} x^{\frac{2x}{2x + 1}}$\tabularnewline
\noalign{\vskip4mm}

\end{tabular}\tabularnewline
\noalign{\vskip4mm}
\begin{tabular}{l}
Вариант Арбитр \tabularnewline
Постройте график в декартовых координатах:\tabularnewline
$y = \tg\frac{\pi}{x^2}$\tabularnewline
\noalign{\vskip4mm}

Докажите предел по определению:\tabularnewline
$\lim\limits_{n \to +\infty} \frac{\ln(n + 1)}{\ln(n^2 + 1)} = \frac{1}{2}$\tabularnewline
\noalign{\vskip4mm}

Докажите, что последовательность расходится:\tabularnewline
$x_n = \sum_{k = 2}^n \frac{1}{\sqrt{k} \ln k}$\tabularnewline
\noalign{\vskip4mm}

Вычислите предел функции:\tabularnewline
$\lim\limits_{x \to 0} x^{\frac{2x}{2x + 1}}$\tabularnewline
\noalign{\vskip4mm}

\end{tabular}& %
\begin{tabular}{l}
Вариант Носитель \tabularnewline
Постройте график в полярных координатах:\tabularnewline
$r = \frac{1}{\phi^2}$\tabularnewline
\noalign{\vskip4mm}

Докажите предел по определению:\tabularnewline
$\lim\limits_{n \to +\infty} \frac{\ln(n + 1)}{\ln(n^2 + 1)} = \frac{1}{2}$\tabularnewline
\noalign{\vskip4mm}

Докажите, что последовательность расходится:\tabularnewline
$x_n = \sum_{k = 2}^n \frac{1}{\sqrt{k} \ln k}$\tabularnewline
\noalign{\vskip4mm}

Вычислите предел функции:\tabularnewline
$\lim\limits_{x \to 0} x^{\frac{2x}{2x + 1}}$\tabularnewline
\noalign{\vskip4mm}

\end{tabular}\tabularnewline
\noalign{\vskip4mm}
\end{tabular}

 
 
\end{document}