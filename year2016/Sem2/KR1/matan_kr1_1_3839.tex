
\documentclass[russian]{article}
\usepackage[T2A,T1]{fontenc}
\usepackage[utf8]{inputenc}
\usepackage[a4paper]{geometry}
\geometry{verbose,tmargin=1cm,bmargin=0cm,lmargin=0cm,rmargin=0cm}
\usepackage{amsmath}

\makeatletter

\DeclareRobustCommand{\cyrtext}{%
  \fontencoding{T2A}\selectfont\def\encodingdefault{T2A}}
\DeclareRobustCommand{\textcyr}[1]{\leavevmode{\cyrtext #1}}
\AtBeginDocument{\DeclareFontEncoding{T2A}{}{}}


\makeatother

\usepackage[russian]{babel}
\begin{document}
 
 
\begin{tabular}{cc}
\begin{tabular}{l}
Вариант КСМ \tabularnewline
Постройте график в декартовых координатах:\tabularnewline
$y = e^{1/(x^2 - 1)}$\tabularnewline
\noalign{\vskip4mm}

Докажите предел по определению:\tabularnewline
$\lim\limits_{n \to +\infty} \frac{n + 1}{\sqrt{n^2 + 2n}} = 1$\tabularnewline
\noalign{\vskip4mm}

Докажите, что последовательность сходится:\tabularnewline
$x_n = \sqrt{n!}$\tabularnewline
\noalign{\vskip4mm}

Вычислите предел функции:\tabularnewline
$\lim\limits_{x \to \pi} \frac{\sin x}{\pi^3 - x^3}$\tabularnewline
\noalign{\vskip4mm}

\end{tabular}& %
\begin{tabular}{l}
Вариант Медик \tabularnewline
Постройте график в полярных координатах:\tabularnewline
$r^2 \sin(2\phi + \frac{\pi}{2}) = \frac{1}{2}$\tabularnewline
\noalign{\vskip4mm}

Докажите предел по определению:\tabularnewline
$\lim\limits_{n \to +\infty} \frac{n + 1}{\sqrt{n^2 + 2n}} = 1$\tabularnewline
\noalign{\vskip4mm}

Докажите, что последовательность сходится:\tabularnewline
$x_n = \sqrt{n!}$\tabularnewline
\noalign{\vskip4mm}

Вычислите предел функции:\tabularnewline
$\lim\limits_{x \to \pi} \frac{\sin x}{\pi^3 - x^3}$\tabularnewline
\noalign{\vskip4mm}

\end{tabular}\tabularnewline
\noalign{\vskip4mm}
\begin{tabular}{l}
Вариант Морпех \tabularnewline
Постройте график в декартовых координатах:\tabularnewline
$y = e^{1/(x^2 - 1)}$\tabularnewline
\noalign{\vskip4mm}

Докажите предел по определению:\tabularnewline
$\lim\limits_{n \to +\infty} \frac{\sqrt[3]{n^2 + n}}{n + 2} = 0$\tabularnewline
\noalign{\vskip4mm}

Докажите, что последовательность сходится:\tabularnewline
$x_n = \sqrt{n!}$\tabularnewline
\noalign{\vskip4mm}

Вычислите предел функции:\tabularnewline
$\lim\limits_{x \to \pi} \frac{\sin x}{\pi^3 - x^3}$\tabularnewline
\noalign{\vskip4mm}

\end{tabular}& %
\begin{tabular}{l}
Вариант Огнемётчик \tabularnewline
Постройте график в полярных координатах:\tabularnewline
$r^2 \sin(2\phi + \frac{\pi}{2}) = \frac{1}{2}$\tabularnewline
\noalign{\vskip4mm}

Докажите предел по определению:\tabularnewline
$\lim\limits_{n \to +\infty} \frac{\sqrt[3]{n^2 + n}}{n + 2} = 0$\tabularnewline
\noalign{\vskip4mm}

Докажите, что последовательность сходится:\tabularnewline
$x_n = \sqrt{n!}$\tabularnewline
\noalign{\vskip4mm}

Вычислите предел функции:\tabularnewline
$\lim\limits_{x \to \pi} \frac{\sin x}{\pi^3 - x^3}$\tabularnewline
\noalign{\vskip4mm}

\end{tabular}\tabularnewline
\noalign{\vskip4mm}
\begin{tabular}{l}
Вариант Ястреб \tabularnewline
Постройте график в декартовых координатах:\tabularnewline
$y = e^{1/(x^2 - 1)}$\tabularnewline
\noalign{\vskip4mm}

Докажите предел по определению:\tabularnewline
$\lim\limits_{n \to +\infty} \frac{n + 1}{\sqrt{n^2 + 2n}} = 1$\tabularnewline
\noalign{\vskip4mm}

Докажите, что последовательность расходится:\tabularnewline
$x_n = \frac{(2n)!!}{(2n + 1)!!}$\tabularnewline
\noalign{\vskip4mm}

Вычислите предел функции:\tabularnewline
$\lim\limits_{x \to \pi} \frac{\sin x}{\pi^3 - x^3}$\tabularnewline
\noalign{\vskip4mm}

\end{tabular}& %
\begin{tabular}{l}
Вариант Дрон \tabularnewline
Постройте график в полярных координатах:\tabularnewline
$r^2 \sin(2\phi + \frac{\pi}{2}) = \frac{1}{2}$\tabularnewline
\noalign{\vskip4mm}

Докажите предел по определению:\tabularnewline
$\lim\limits_{n \to +\infty} \frac{n + 1}{\sqrt{n^2 + 2n}} = 1$\tabularnewline
\noalign{\vskip4mm}

Докажите, что последовательность расходится:\tabularnewline
$x_n = \frac{(2n)!!}{(2n + 1)!!}$\tabularnewline
\noalign{\vskip4mm}

Вычислите предел функции:\tabularnewline
$\lim\limits_{x \to \pi} \frac{\sin x}{\pi^3 - x^3}$\tabularnewline
\noalign{\vskip4mm}

\end{tabular}\tabularnewline
\noalign{\vskip4mm}
\begin{tabular}{l}
Вариант Зерлинг \tabularnewline
Постройте график в декартовых координатах:\tabularnewline
$y = e^{1/(x^2 - 1)}$\tabularnewline
\noalign{\vskip4mm}

Докажите предел по определению:\tabularnewline
$\lim\limits_{n \to +\infty} \frac{\sqrt[3]{n^2 + n}}{n + 2} = 0$\tabularnewline
\noalign{\vskip4mm}

Докажите, что последовательность расходится:\tabularnewline
$x_n = \frac{(2n)!!}{(2n + 1)!!}$\tabularnewline
\noalign{\vskip4mm}

Вычислите предел функции:\tabularnewline
$\lim\limits_{x \to \pi} \frac{\sin x}{\pi^3 - x^3}$\tabularnewline
\noalign{\vskip4mm}

\end{tabular}& %
\begin{tabular}{l}
Вариант Гидралиск \tabularnewline
Постройте график в полярных координатах:\tabularnewline
$r^2 \sin(2\phi + \frac{\pi}{2}) = \frac{1}{2}$\tabularnewline
\noalign{\vskip4mm}

Докажите предел по определению:\tabularnewline
$\lim\limits_{n \to +\infty} \frac{\sqrt[3]{n^2 + n}}{n + 2} = 0$\tabularnewline
\noalign{\vskip4mm}

Докажите, что последовательность расходится:\tabularnewline
$x_n = \frac{(2n)!!}{(2n + 1)!!}$\tabularnewline
\noalign{\vskip4mm}

Вычислите предел функции:\tabularnewline
$\lim\limits_{x \to \pi} \frac{\sin x}{\pi^3 - x^3}$\tabularnewline
\noalign{\vskip4mm}

\end{tabular}\tabularnewline
\noalign{\vskip4mm}
\end{tabular}

\begin{tabular}{cc}
\begin{tabular}{l}
Вариант Выводок \tabularnewline
Постройте график в декартовых координатах:\tabularnewline
$y = e^{1/(x^2 - 1)}$\tabularnewline
\noalign{\vskip4mm}

Докажите предел по определению:\tabularnewline
$\lim\limits_{n \to +\infty} \frac{n + 1}{\sqrt{n^2 + 2n}} = 1$\tabularnewline
\noalign{\vskip4mm}

Докажите, что последовательность сходится:\tabularnewline
$x_n = \sqrt{n!}$\tabularnewline
\noalign{\vskip4mm}

Вычислите предел функции:\tabularnewline
$\lim\limits_{x \to 0} x \ctg 5x$\tabularnewline
\noalign{\vskip4mm}

\end{tabular}& %
\begin{tabular}{l}
Вариант Зараженный землянин \tabularnewline
Постройте график в полярных координатах:\tabularnewline
$r^2 \sin(2\phi + \frac{\pi}{2}) = \frac{1}{2}$\tabularnewline
\noalign{\vskip4mm}

Докажите предел по определению:\tabularnewline
$\lim\limits_{n \to +\infty} \frac{n + 1}{\sqrt{n^2 + 2n}} = 1$\tabularnewline
\noalign{\vskip4mm}

Докажите, что последовательность сходится:\tabularnewline
$x_n = \sqrt{n!}$\tabularnewline
\noalign{\vskip4mm}

Вычислите предел функции:\tabularnewline
$\lim\limits_{x \to 0} x \ctg 5x$\tabularnewline
\noalign{\vskip4mm}

\end{tabular}\tabularnewline
\noalign{\vskip4mm}
\begin{tabular}{l}
Вариант Бич \tabularnewline
Постройте график в декартовых координатах:\tabularnewline
$y = e^{1/(x^2 - 1)}$\tabularnewline
\noalign{\vskip4mm}

Докажите предел по определению:\tabularnewline
$\lim\limits_{n \to +\infty} \frac{\sqrt[3]{n^2 + n}}{n + 2} = 0$\tabularnewline
\noalign{\vskip4mm}

Докажите, что последовательность сходится:\tabularnewline
$x_n = \sqrt{n!}$\tabularnewline
\noalign{\vskip4mm}

Вычислите предел функции:\tabularnewline
$\lim\limits_{x \to 0} x \ctg 5x$\tabularnewline
\noalign{\vskip4mm}

\end{tabular}& %
\begin{tabular}{l}
Вариант Проба \tabularnewline
Постройте график в полярных координатах:\tabularnewline
$r^2 \sin(2\phi + \frac{\pi}{2}) = \frac{1}{2}$\tabularnewline
\noalign{\vskip4mm}

Докажите предел по определению:\tabularnewline
$\lim\limits_{n \to +\infty} \frac{\sqrt[3]{n^2 + n}}{n + 2} = 0$\tabularnewline
\noalign{\vskip4mm}

Докажите, что последовательность сходится:\tabularnewline
$x_n = \sqrt{n!}$\tabularnewline
\noalign{\vskip4mm}

Вычислите предел функции:\tabularnewline
$\lim\limits_{x \to 0} x \ctg 5x$\tabularnewline
\noalign{\vskip4mm}

\end{tabular}\tabularnewline
\noalign{\vskip4mm}
\begin{tabular}{l}
Вариант Зилот \tabularnewline
Постройте график в декартовых координатах:\tabularnewline
$y = e^{1/(x^2 - 1)}$\tabularnewline
\noalign{\vskip4mm}

Докажите предел по определению:\tabularnewline
$\lim\limits_{n \to +\infty} \frac{n + 1}{\sqrt{n^2 + 2n}} = 1$\tabularnewline
\noalign{\vskip4mm}

Докажите, что последовательность расходится:\tabularnewline
$x_n = \frac{(2n)!!}{(2n + 1)!!}$\tabularnewline
\noalign{\vskip4mm}

Вычислите предел функции:\tabularnewline
$\lim\limits_{x \to 0} x \ctg 5x$\tabularnewline
\noalign{\vskip4mm}

\end{tabular}& %
\begin{tabular}{l}
Вариант Драгун \tabularnewline
Постройте график в полярных координатах:\tabularnewline
$r^2 \sin(2\phi + \frac{\pi}{2}) = \frac{1}{2}$\tabularnewline
\noalign{\vskip4mm}

Докажите предел по определению:\tabularnewline
$\lim\limits_{n \to +\infty} \frac{n + 1}{\sqrt{n^2 + 2n}} = 1$\tabularnewline
\noalign{\vskip4mm}

Докажите, что последовательность расходится:\tabularnewline
$x_n = \frac{(2n)!!}{(2n + 1)!!}$\tabularnewline
\noalign{\vskip4mm}

Вычислите предел функции:\tabularnewline
$\lim\limits_{x \to 0} x \ctg 5x$\tabularnewline
\noalign{\vskip4mm}

\end{tabular}\tabularnewline
\noalign{\vskip4mm}
\begin{tabular}{l}
Вариант Темплар \tabularnewline
Постройте график в декартовых координатах:\tabularnewline
$y = e^{1/(x^2 - 1)}$\tabularnewline
\noalign{\vskip4mm}

Докажите предел по определению:\tabularnewline
$\lim\limits_{n \to +\infty} \frac{\sqrt[3]{n^2 + n}}{n + 2} = 0$\tabularnewline
\noalign{\vskip4mm}

Докажите, что последовательность расходится:\tabularnewline
$x_n = \frac{(2n)!!}{(2n + 1)!!}$\tabularnewline
\noalign{\vskip4mm}

Вычислите предел функции:\tabularnewline
$\lim\limits_{x \to 0} x \ctg 5x$\tabularnewline
\noalign{\vskip4mm}

\end{tabular}& %
\begin{tabular}{l}
Вариант Наблюдатель \tabularnewline
Постройте график в полярных координатах:\tabularnewline
$r^2 \sin(2\phi + \frac{\pi}{2}) = \frac{1}{2}$\tabularnewline
\noalign{\vskip4mm}

Докажите предел по определению:\tabularnewline
$\lim\limits_{n \to +\infty} \frac{\sqrt[3]{n^2 + n}}{n + 2} = 0$\tabularnewline
\noalign{\vskip4mm}

Докажите, что последовательность расходится:\tabularnewline
$x_n = \frac{(2n)!!}{(2n + 1)!!}$\tabularnewline
\noalign{\vskip4mm}

Вычислите предел функции:\tabularnewline
$\lim\limits_{x \to 0} x \ctg 5x$\tabularnewline
\noalign{\vskip4mm}

\end{tabular}\tabularnewline
\noalign{\vskip4mm}
\end{tabular}



 
\end{document}