\documentclass[russian]{article}
\usepackage[T2A,T1]{fontenc}
\usepackage[utf8]{inputenc}
\usepackage[a4paper]{geometry}
\geometry{verbose,tmargin=1cm,bmargin=1cm,lmargin=1cm,rmargin=1cm}
\usepackage{amsmath}
\makeatletter

%%%%%%%%%%%%%%%%%%%%%%%%%%%%%% LyX specific LaTeX commands.
\DeclareRobustCommand{\cyrtext}{%
  \fontencoding{T2A}\selectfont\def\encodingdefault{T2A}}
\DeclareRobustCommand{\textcyr}[1]{\leavevmode{\cyrtext #1}}
\AtBeginDocument{\DeclareFontEncoding{T2A}{}{}}

\newcommand{\dx}{\text{d}x}
\makeatother

\usepackage[russian]{babel}
\begin{document}

\textit{1 задание: простые интегралы и интегралы транцентдентных функций}

Посчитайте неопределенный интеграл:
$$\int \arcsin\sqrt{\frac{x}{x + 1}} \dx$$

Посчитайте неопределенный интеграл:
$$\int \sqrt{\tg^3 x} \dx$$

\textit{2 задание: интегралы рациональных и иррациональных функций}

Посчитайте неопределенный интеграл:
$$\int \frac{4x\dx}{(x + 1)(x^2 + 1)}$$

Посчитайте неопределенный интеграл:
$$\int \frac{x + \sqrt{1 + x + x^2}}{1 + x + \sqrt{1 + x + x^2}} \dx$$

\textit{3 задание: площади плоских фигур}

Посчитайте площадь фигуры, ограниченной следующими кривыми:
$$x = a(t - \sin t); y = a(1 - \cos t); t \in [0; 2\pi]$$
$$y = 0$$

Посчитайте площадь фигуры, ограниченной кривой, заданной в полярных координатах:
$$ r^2 = 2a^2 \cos 2 \phi$$

\textit{4 задание: длина кривой или объем}

Посчитайте длину участка графика функции:
$$ y = \sqrt{x^2 - 32} + 8 \ln (x + \sqrt{x^2 - 32}), x \in [6;9]$$

Посчитайте объем фигуры, получающейся в результате вращения плоской фигуры вокруг оси OY. Плоская фигура ограничена следующими кривыми:
$$y = tg^2 x$$
$$y = 0$$
$$x = \sqrt{\pi/3}$$
\end{document}