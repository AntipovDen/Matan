\documentclass[russian,landscape]{article}
\usepackage[T2A,T1]{fontenc}
\usepackage[utf8]{inputenc}
\usepackage[a4paper]{geometry}
\geometry{verbose,tmargin=1cm,bmargin=0cm,lmargin=0cm,rmargin=0cm}
\usepackage{amsmath}
\usepackage{amsfonts}

\makeatletter

\DeclareRobustCommand{\cyrtext}{%
  \fontencoding{T2A}\selectfont\def\encodingdefault{T2A}}
\DeclareRobustCommand{\textcyr}[1]{\leavevmode{\cyrtext #1}}
\AtBeginDocument{\DeclareFontEncoding{T2A}{}{}}

\newcommand{\dx}{\text{d}x}
\makeatother

\usepackage[russian]{babel}
\begin{document}

\begin{tabular}{cc}
\begin{tabular}{l}
Вариант Asylum Demon \tabularnewline
Вычислите интеграл:\tabularnewline
$\int\frac{\arcsin(x/2)}{\sqrt{2 - x}} \dx$\tabularnewline
\noalign{\vskip4mm}

Вычислите интеграл:\tabularnewline
$\int \sqrt[3]{1 + \sqrt[4]{x}} \dx$\tabularnewline
\noalign{\vskip4mm}

Вычислите площадь фигуры, ограниченной кривыми:\tabularnewline
$r = \frac{1}{\sqrt{1 + \cos^2 \phi}}, \text{ } \phi = 0 \text{ и } \phi = \frac{\pi}{4}$\tabularnewline
\noalign{\vskip4mm}

Вычислите длину кривой:\tabularnewline
$y =\sqrt{x^2 -32} + 8 \ln(x + \sqrt{x^2 - 32}), x \in [6;9]$\tabularnewline
\noalign{\vskip4mm}

\end{tabular}& %
\begin{tabular}{l}
Вариант Bed of Chaos \tabularnewline
Вычислите интеграл:\tabularnewline
$\int\frac{\arcsin(x/2)}{\sqrt{2 - x}} \dx$\tabularnewline
\noalign{\vskip4mm}

Вычислите интеграл:\tabularnewline
$\int \frac{x(x^2 + 1) \dx}{(x + 1)(x^2 + 2x + 2)}$\tabularnewline
\noalign{\vskip4mm}

Вычислите площадь фигуры, ограниченной кривыми:\tabularnewline
$y = \sin^3 x + \cos^3 x, x \in [-\frac{\pi}{4}; \frac{\pi}{4}] \text{ и } y = 0$\tabularnewline
\noalign{\vskip4mm}

Вычислите длину кривой:\tabularnewline
$y =\sqrt{x^2 -32} + 8 \ln(x + \sqrt{x^2 - 32}), x \in [6;9]$\tabularnewline
\noalign{\vskip4mm}

\end{tabular}\tabularnewline
\noalign{\vskip4mm}
\begin{tabular}{l}
Вариант Bell Gargoyles \tabularnewline
Вычислите интеграл:\tabularnewline
$\int \frac{1 + \tg x}{\sin 2x} \dx$\tabularnewline
\noalign{\vskip4mm}

Вычислите интеграл:\tabularnewline
$\int \sqrt[5]{\frac{x}{x + 1}} \frac{\dx}{x^3}$\tabularnewline
\noalign{\vskip4mm}

Вычислите площадь фигуры, ограниченной кривыми:\tabularnewline
$r = \frac{1}{\sqrt{1 + \cos^2 \phi}}, \text{ } \phi = 0 \text{ и } \phi = \frac{\pi}{4}$\tabularnewline
\noalign{\vskip4mm}

Вычислите длину петли кривой:\tabularnewline
$x = t^2, y = t(\frac{1}{3} - t^2)$\tabularnewline
\noalign{\vskip4mm}

\end{tabular}& %
\begin{tabular}{l}
Вариант Black Dragon Kalameet \tabularnewline
Вычислите интеграл:\tabularnewline
$\int \frac{1 + \tg x}{\sin 2x} \dx$\tabularnewline
\noalign{\vskip4mm}

Вычислите интеграл:\tabularnewline
$\int \sqrt[3]{1 + \sqrt[4]{x}} \dx$\tabularnewline
\noalign{\vskip4mm}

Вычислите площадь фигуры, ограниченной кривыми:\tabularnewline
$r = \frac{1}{\sqrt{1 + \cos^2 \phi}}, \text{ } \phi = 0 \text{ и } \phi = \frac{\pi}{4}$\tabularnewline
\noalign{\vskip4mm}

Вычислите объем тела вращения данной кривой вокруг оси Ox:\tabularnewline
$y = \sqrt[4]{1 + e^{2x}}, x \in [\ln \sqrt{3}; \ln \sqrt{8}]$\tabularnewline
\noalign{\vskip4mm}

\end{tabular}\tabularnewline
\noalign{\vskip4mm}
\begin{tabular}{l}
Вариант Capra Demon \tabularnewline
Вычислите интеграл:\tabularnewline
$\int \frac{\sh^3 x}{\sqrt[3]{\ch^2x}} \dx$\tabularnewline
\noalign{\vskip4mm}

Вычислите интеграл:\tabularnewline
$\int \sqrt[3]{1 + \sqrt[4]{x}} \dx$\tabularnewline
\noalign{\vskip4mm}

Вычислите площадь фигуры, ограниченной кривыми:\tabularnewline
$y = \sin^3 x + \cos^3 x, x \in [-\frac{\pi}{4}; \frac{\pi}{4}] \text{ и } y = 0$\tabularnewline
\noalign{\vskip4mm}

Вычислите длину петли кривой:\tabularnewline
$x = t^2, y = t(\frac{1}{3} - t^2)$\tabularnewline
\noalign{\vskip4mm}

\end{tabular}& %
\begin{tabular}{l}
Вариант Ceaseless Discharge \tabularnewline
Вычислите интеграл:\tabularnewline
$\int \frac{1 + \tg x}{\sin 2x} \dx$\tabularnewline
\noalign{\vskip4mm}

Вычислите интеграл:\tabularnewline
$\int \sqrt[5]{\frac{x}{x + 1}} \frac{\dx}{x^3}$\tabularnewline
\noalign{\vskip4mm}

Вычислите площадь фигуры, ограниченной кривыми:\tabularnewline
$r = \frac{1}{\sqrt{1 + \cos^2 \phi}}, \text{ } \phi = 0 \text{ и } \phi = \frac{\pi}{4}$\tabularnewline
\noalign{\vskip4mm}

Вычислите длину кривой:\tabularnewline
$y =\sqrt{x^2 -32} + 8 \ln(x + \sqrt{x^2 - 32}), x \in [6;9]$\tabularnewline
\noalign{\vskip4mm}

\end{tabular}\tabularnewline
\noalign{\vskip4mm}
\end{tabular}

\begin{tabular}{cc}
\begin{tabular}{l}
Вариант Centipede Demon \tabularnewline
Вычислите интеграл:\tabularnewline
$\int 3^x \cos x \dx$\tabularnewline
\noalign{\vskip4mm}

Вычислите интеграл:\tabularnewline
$\int \frac{\dx}{(x + 2x + 2)^{5/2}}$\tabularnewline
\noalign{\vskip4mm}

Вычислите площадь фигуры, ограниченной кривыми:\tabularnewline
$r = \frac{1}{\sqrt{1 + \cos^2 \phi}}, \text{ } \phi = 0 \text{ и } \phi = \frac{\pi}{4}$\tabularnewline
\noalign{\vskip4mm}

Вычислите длину кривой:\tabularnewline
$y =\sqrt{x^2 -32} + 8 \ln(x + \sqrt{x^2 - 32}), x \in [6;9]$\tabularnewline
\noalign{\vskip4mm}

\end{tabular}& %
\begin{tabular}{l}
Вариант Chaos Witch Quelaag \tabularnewline
Вычислите интеграл:\tabularnewline
$\int \frac{\dx}{2\cos^2 x+ \sin x \cos x + \sin^2 x}$\tabularnewline
\noalign{\vskip4mm}

Вычислите интеграл:\tabularnewline
$\int \sqrt[3]{1 + \sqrt[4]{x}} \dx$\tabularnewline
\noalign{\vskip4mm}

Вычислите площадь фигуры, ограниченной кривыми:\tabularnewline
$r = \frac{1}{\sqrt{1 + \cos^2 \phi}}, \text{ } \phi = 0 \text{ и } \phi = \frac{\pi}{4}$\tabularnewline
\noalign{\vskip4mm}

Вычислите длину кривой:\tabularnewline
$y =\sqrt{x^2 -32} + 8 \ln(x + \sqrt{x^2 - 32}), x \in [6;9]$\tabularnewline
\noalign{\vskip4mm}

\end{tabular}\tabularnewline
\noalign{\vskip4mm}
\begin{tabular}{l}
Вариант Crossbreed Priscilla \tabularnewline
Вычислите интеграл:\tabularnewline
$\int \frac{\dx}{2\cos^2 x+ \sin x \cos x + \sin^2 x}$\tabularnewline
\noalign{\vskip4mm}

Вычислите интеграл:\tabularnewline
$\int \frac{x(x^2 + 1) \dx}{(x + 1)(x^2 + 2x + 2)}$\tabularnewline
\noalign{\vskip4mm}

Вычислите площадь фигуры, ограниченной кривой:\tabularnewline
$y = a\sin t, x = b\cos t; a, b \in \mathbb{R}$\tabularnewline
\noalign{\vskip4mm}

Вычислите объем тела вращения данной кривой вокруг оси Ox:\tabularnewline
$y = \sqrt[4]{1 + e^{2x}}, x \in [\ln \sqrt{3}; \ln \sqrt{8}]$\tabularnewline
\noalign{\vskip4mm}

\end{tabular}& %
\begin{tabular}{l}
Вариант Dark Sun Gwyndolin \tabularnewline
Вычислите интеграл:\tabularnewline
$\int\frac{\arcsin(x/2)}{\sqrt{2 - x}} \dx$\tabularnewline
\noalign{\vskip4mm}

Вычислите интеграл:\tabularnewline
$\int \frac{x(x^2 + 1) \dx}{(x + 1)(x^2 + 2x + 2)}$\tabularnewline
\noalign{\vskip4mm}

Вычислите площадь фигуры, ограниченной кривой:\tabularnewline
$y = a\sin t, x = b\cos t; a, b \in \mathbb{R}$\tabularnewline
\noalign{\vskip4mm}

Вычислите длину петли кривой:\tabularnewline
$x = t^2, y = t(\frac{1}{3} - t^2)$\tabularnewline
\noalign{\vskip4mm}

\end{tabular}\tabularnewline
\noalign{\vskip4mm}
\begin{tabular}{l}
Вариант Demon Firesage \tabularnewline
Вычислите интеграл:\tabularnewline
$\int\frac{\arcsin(x/2)}{\sqrt{2 - x}} \dx$\tabularnewline
\noalign{\vskip4mm}

Вычислите интеграл:\tabularnewline
$\int \frac{\dx}{(x + 2x + 2)^{5/2}}$\tabularnewline
\noalign{\vskip4mm}

Вычислите площадь фигуры, ограниченной кривыми:\tabularnewline
$y = \sin^3 x + \cos^3 x, x \in [-\frac{\pi}{4}; \frac{\pi}{4}] \text{ и } y = 0$\tabularnewline
\noalign{\vskip4mm}

Вычислите длину кривой:\tabularnewline
$y =\sqrt{x^2 -32} + 8 \ln(x + \sqrt{x^2 - 32}), x \in [6;9]$\tabularnewline
\noalign{\vskip4mm}

\end{tabular}& %
\begin{tabular}{l}
Вариант Dragon Slayer Ornstein \& Executioner Smough \tabularnewline
Вычислите интеграл:\tabularnewline
$\int \frac{1 + \tg x}{\sin 2x} \dx$\tabularnewline
\noalign{\vskip4mm}

Вычислите интеграл:\tabularnewline
$\int \frac{x(x^2 + 1) \dx}{(x + 1)(x^2 + 2x + 2)}$\tabularnewline
\noalign{\vskip4mm}

Вычислите площадь фигуры, ограниченной кривыми:\tabularnewline
$y = \sin^3 x + \cos^3 x, x \in [-\frac{\pi}{4}; \frac{\pi}{4}] \text{ и } y = 0$\tabularnewline
\noalign{\vskip4mm}

Вычислите длину кривой:\tabularnewline
$r = ae^{k\phi}, \phi \in [\phi_1;\phi_2], a > 0, k \in \mathbb{R}$\tabularnewline
\noalign{\vskip4mm}

\end{tabular}\tabularnewline
\noalign{\vskip4mm}
\end{tabular}

\begin{tabular}{cc}
\begin{tabular}{l}
Вариант Four Kings \tabularnewline
Вычислите интеграл:\tabularnewline
$\int \frac{\sh^3 x}{\sqrt[3]{\ch^2x}} \dx$\tabularnewline
\noalign{\vskip4mm}

Вычислите интеграл:\tabularnewline
$\int \frac{\dx}{(x + 2x + 2)^{5/2}}$\tabularnewline
\noalign{\vskip4mm}

Вычислите площадь фигуры, ограниченной кривой:\tabularnewline
$y = a\sin t, x = b\cos t; a, b \in \mathbb{R}$\tabularnewline
\noalign{\vskip4mm}

Вычислите длину кривой:\tabularnewline
$r = ae^{k\phi}, \phi \in [\phi_1;\phi_2], a > 0, k \in \mathbb{R}$\tabularnewline
\noalign{\vskip4mm}

\end{tabular}& %
\begin{tabular}{l}
Вариант Gaping Dragon \tabularnewline
Вычислите интеграл:\tabularnewline
$\int 3^x \cos x \dx$\tabularnewline
\noalign{\vskip4mm}

Вычислите интеграл:\tabularnewline
$\int \sqrt[5]{\frac{x}{x + 1}} \frac{\dx}{x^3}$\tabularnewline
\noalign{\vskip4mm}

Вычислите площадь фигуры, ограниченной кривой:\tabularnewline
$y = a\sin t, x = b\cos t; a, b \in \mathbb{R}$\tabularnewline
\noalign{\vskip4mm}

Вычислите длину кривой:\tabularnewline
$y =\sqrt{x^2 -32} + 8 \ln(x + \sqrt{x^2 - 32}), x \in [6;9]$\tabularnewline
\noalign{\vskip4mm}

\end{tabular}\tabularnewline
\noalign{\vskip4mm}
\begin{tabular}{l}
Вариант Gravelord Nito \tabularnewline
Вычислите интеграл:\tabularnewline
$\int 3^x \cos x \dx$\tabularnewline
\noalign{\vskip4mm}

Вычислите интеграл:\tabularnewline
$\int \frac{x(x^2 + 1) \dx}{(x + 1)(x^2 + 2x + 2)}$\tabularnewline
\noalign{\vskip4mm}

Вычислите площадь фигуры, ограниченной кривыми:\tabularnewline
$r = \frac{1}{\sqrt{1 + \cos^2 \phi}}, \text{ } \phi = 0 \text{ и } \phi = \frac{\pi}{4}$\tabularnewline
\noalign{\vskip4mm}

Вычислите длину кривой:\tabularnewline
$y =\sqrt{x^2 -32} + 8 \ln(x + \sqrt{x^2 - 32}), x \in [6;9]$\tabularnewline
\noalign{\vskip4mm}

\end{tabular}& %
\begin{tabular}{l}
Вариант Gwyn, Lord of Cinder \tabularnewline
Вычислите интеграл:\tabularnewline
$\int\frac{\arcsin(x/2)}{\sqrt{2 - x}} \dx$\tabularnewline
\noalign{\vskip4mm}

Вычислите интеграл:\tabularnewline
$\int \frac{x(x^2 + 1) \dx}{(x + 1)(x^2 + 2x + 2)}$\tabularnewline
\noalign{\vskip4mm}

Вычислите площадь фигуры, ограниченной кривыми:\tabularnewline
$y = \sin^3 x + \cos^3 x, x \in [-\frac{\pi}{4}; \frac{\pi}{4}] \text{ и } y = 0$\tabularnewline
\noalign{\vskip4mm}

Вычислите длину петли кривой:\tabularnewline
$x = t^2, y = t(\frac{1}{3} - t^2)$\tabularnewline
\noalign{\vskip4mm}

\end{tabular}\tabularnewline
\noalign{\vskip4mm}
\begin{tabular}{l}
Вариант Iron Golem \tabularnewline
Вычислите интеграл:\tabularnewline
$\int \frac{1 + \tg x}{\sin 2x} \dx$\tabularnewline
\noalign{\vskip4mm}

Вычислите интеграл:\tabularnewline
$\int \frac{x(x^2 + 1) \dx}{(x + 1)(x^2 + 2x + 2)}$\tabularnewline
\noalign{\vskip4mm}

Вычислите площадь фигуры, ограниченной кривыми:\tabularnewline
$y = \sin^3 x + \cos^3 x, x \in [-\frac{\pi}{4}; \frac{\pi}{4}] \text{ и } y = 0$\tabularnewline
\noalign{\vskip4mm}

Вычислите длину кривой:\tabularnewline
$y =\sqrt{x^2 -32} + 8 \ln(x + \sqrt{x^2 - 32}), x \in [6;9]$\tabularnewline
\noalign{\vskip4mm}

\end{tabular}& %
\begin{tabular}{l}
Вариант Knight Artorias \tabularnewline
Вычислите интеграл:\tabularnewline
$\int\frac{\arcsin(x/2)}{\sqrt{2 - x}} \dx$\tabularnewline
\noalign{\vskip4mm}

Вычислите интеграл:\tabularnewline
$\int \sqrt[3]{1 + \sqrt[4]{x}} \dx$\tabularnewline
\noalign{\vskip4mm}

Вычислите площадь фигуры, ограниченной кривыми:\tabularnewline
$r = \frac{1}{\sqrt{1 + \cos^2 \phi}}, \text{ } \phi = 0 \text{ и } \phi = \frac{\pi}{4}$\tabularnewline
\noalign{\vskip4mm}

Вычислите длину кривой:\tabularnewline
$r = ae^{k\phi}, \phi \in [\phi_1;\phi_2], a > 0, k \in \mathbb{R}$\tabularnewline
\noalign{\vskip4mm}

\end{tabular}\tabularnewline
\noalign{\vskip4mm}
\end{tabular}

\begin{tabular}{cc}
\begin{tabular}{l}
Вариант Manus, Father of the Abyss \tabularnewline
Вычислите интеграл:\tabularnewline
$\int \frac{\sh^3 x}{\sqrt[3]{\ch^2x}} \dx$\tabularnewline
\noalign{\vskip4mm}

Вычислите интеграл:\tabularnewline
$\int \sqrt[5]{\frac{x}{x + 1}} \frac{\dx}{x^3}$\tabularnewline
\noalign{\vskip4mm}

Вычислите площадь фигуры, ограниченной кривыми:\tabularnewline
$y = \sin^3 x + \cos^3 x, x \in [-\frac{\pi}{4}; \frac{\pi}{4}] \text{ и } y = 0$\tabularnewline
\noalign{\vskip4mm}

Вычислите длину петли кривой:\tabularnewline
$x = t^2, y = t(\frac{1}{3} - t^2)$\tabularnewline
\noalign{\vskip4mm}

\end{tabular}& %
\begin{tabular}{l}
Вариант Moonlight Butterfly \tabularnewline
Вычислите интеграл:\tabularnewline
$\int\frac{\arcsin(x/2)}{\sqrt{2 - x}} \dx$\tabularnewline
\noalign{\vskip4mm}

Вычислите интеграл:\tabularnewline
$\int \frac{\dx}{(x + 2x + 2)^{5/2}}$\tabularnewline
\noalign{\vskip4mm}

Вычислите площадь фигуры, ограниченной кривыми:\tabularnewline
$y = \sin^3 x + \cos^3 x, x \in [-\frac{\pi}{4}; \frac{\pi}{4}] \text{ и } y = 0$\tabularnewline
\noalign{\vskip4mm}

Вычислите объем тела вращения данной кривой вокруг оси Ox:\tabularnewline
$y = \sqrt[4]{1 + e^{2x}}, x \in [\ln \sqrt{3}; \ln \sqrt{8}]$\tabularnewline
\noalign{\vskip4mm}

\end{tabular}\tabularnewline
\noalign{\vskip4mm}
\begin{tabular}{l}
Вариант Pinwheel \tabularnewline
Вычислите интеграл:\tabularnewline
$\int\frac{\arcsin(x/2)}{\sqrt{2 - x}} \dx$\tabularnewline
\noalign{\vskip4mm}

Вычислите интеграл:\tabularnewline
$\int \frac{\dx}{(x + 2x + 2)^{5/2}}$\tabularnewline
\noalign{\vskip4mm}

Вычислите площадь фигуры, ограниченной кривыми:\tabularnewline
$r = \frac{1}{\sqrt{1 + \cos^2 \phi}}, \text{ } \phi = 0 \text{ и } \phi = \frac{\pi}{4}$\tabularnewline
\noalign{\vskip4mm}

Вычислите длину кривой:\tabularnewline
$r = ae^{k\phi}, \phi \in [\phi_1;\phi_2], a > 0, k \in \mathbb{R}$\tabularnewline
\noalign{\vskip4mm}

\end{tabular}& %
\begin{tabular}{l}
Вариант Sanctuary Guardian \tabularnewline
Вычислите интеграл:\tabularnewline
$\int \frac{\dx}{2\cos^2 x+ \sin x \cos x + \sin^2 x}$\tabularnewline
\noalign{\vskip4mm}

Вычислите интеграл:\tabularnewline
$\int \sqrt[5]{\frac{x}{x + 1}} \frac{\dx}{x^3}$\tabularnewline
\noalign{\vskip4mm}

Вычислите площадь фигуры, ограниченной кривыми:\tabularnewline
$y = \sin^3 x + \cos^3 x, x \in [-\frac{\pi}{4}; \frac{\pi}{4}] \text{ и } y = 0$\tabularnewline
\noalign{\vskip4mm}

Вычислите объем тела вращения данной кривой вокруг оси Ox:\tabularnewline
$y = \sqrt[4]{1 + e^{2x}}, x \in [\ln \sqrt{3}; \ln \sqrt{8}]$\tabularnewline
\noalign{\vskip4mm}

\end{tabular}\tabularnewline
\noalign{\vskip4mm}
\begin{tabular}{l}
Вариант Seath the Scaleless \tabularnewline
Вычислите интеграл:\tabularnewline
$\int \frac{1 + \tg x}{\sin 2x} \dx$\tabularnewline
\noalign{\vskip4mm}

Вычислите интеграл:\tabularnewline
$\int \sqrt[3]{1 + \sqrt[4]{x}} \dx$\tabularnewline
\noalign{\vskip4mm}

Вычислите площадь фигуры, ограниченной кривыми:\tabularnewline
$r = \frac{1}{\sqrt{1 + \cos^2 \phi}}, \text{ } \phi = 0 \text{ и } \phi = \frac{\pi}{4}$\tabularnewline
\noalign{\vskip4mm}

Вычислите объем тела вращения данной кривой вокруг оси Ox:\tabularnewline
$y = \sqrt[4]{1 + e^{2x}}, x \in [\ln \sqrt{3}; \ln \sqrt{8}]$\tabularnewline
\noalign{\vskip4mm}

\end{tabular}& %
\begin{tabular}{l}
Вариант Sif, the Great Grey Wolf \tabularnewline
Вычислите интеграл:\tabularnewline
$\int \frac{1 + \tg x}{\sin 2x} \dx$\tabularnewline
\noalign{\vskip4mm}

Вычислите интеграл:\tabularnewline
$\int \frac{x(x^2 + 1) \dx}{(x + 1)(x^2 + 2x + 2)}$\tabularnewline
\noalign{\vskip4mm}

Вычислите площадь фигуры, ограниченной кривыми:\tabularnewline
$r = \frac{1}{\sqrt{1 + \cos^2 \phi}}, \text{ } \phi = 0 \text{ и } \phi = \frac{\pi}{4}$\tabularnewline
\noalign{\vskip4mm}

Вычислите длину кривой:\tabularnewline
$r = ae^{k\phi}, \phi \in [\phi_1;\phi_2], a > 0, k \in \mathbb{R}$\tabularnewline
\noalign{\vskip4mm}

\end{tabular}\tabularnewline
\noalign{\vskip4mm}
\end{tabular}

\begin{tabular}{cc}
\begin{tabular}{l}
Вариант Stray Demon \tabularnewline
Вычислите интеграл:\tabularnewline
$\int \frac{\dx}{2\cos^2 x+ \sin x \cos x + \sin^2 x}$\tabularnewline
\noalign{\vskip4mm}

Вычислите интеграл:\tabularnewline
$\int \sqrt[3]{1 + \sqrt[4]{x}} \dx$\tabularnewline
\noalign{\vskip4mm}

Вычислите площадь фигуры, ограниченной кривыми:\tabularnewline
$y = \sin^3 x + \cos^3 x, x \in [-\frac{\pi}{4}; \frac{\pi}{4}] \text{ и } y = 0$\tabularnewline
\noalign{\vskip4mm}

Вычислите длину петли кривой:\tabularnewline
$x = t^2, y = t(\frac{1}{3} - t^2)$\tabularnewline
\noalign{\vskip4mm}

\end{tabular}& %
\begin{tabular}{l}
Вариант Taurus Demon \tabularnewline
Вычислите интеграл:\tabularnewline
$\int 3^x \cos x \dx$\tabularnewline
\noalign{\vskip4mm}

Вычислите интеграл:\tabularnewline
$\int \frac{x(x^2 + 1) \dx}{(x + 1)(x^2 + 2x + 2)}$\tabularnewline
\noalign{\vskip4mm}

Вычислите площадь фигуры, ограниченной кривыми:\tabularnewline
$y = \sin^3 x + \cos^3 x, x \in [-\frac{\pi}{4}; \frac{\pi}{4}] \text{ и } y = 0$\tabularnewline
\noalign{\vskip4mm}

Вычислите длину кривой:\tabularnewline
$r = ae^{k\phi}, \phi \in [\phi_1;\phi_2], a > 0, k \in \mathbb{R}$\tabularnewline
\noalign{\vskip4mm}

\end{tabular}\tabularnewline
\noalign{\vskip4mm}
\begin{tabular}{l}
Вариант Последний Гигант \tabularnewline
Вычислите интеграл:\tabularnewline
$\int \frac{\sh^3 x}{\sqrt[3]{\ch^2x}} \dx$\tabularnewline
\noalign{\vskip4mm}

Вычислите интеграл:\tabularnewline
$\int \sqrt[5]{\frac{x}{x + 1}} \frac{\dx}{x^3}$\tabularnewline
\noalign{\vskip4mm}

Вычислите площадь фигуры, ограниченной кривыми:\tabularnewline
$y = \sin^3 x + \cos^3 x, x \in [-\frac{\pi}{4}; \frac{\pi}{4}] \text{ и } y = 0$\tabularnewline
\noalign{\vskip4mm}

Вычислите длину кривой:\tabularnewline
$r = ae^{k\phi}, \phi \in [\phi_1;\phi_2], a > 0, k \in \mathbb{R}$\tabularnewline
\noalign{\vskip4mm}

\end{tabular}& %
\begin{tabular}{l}
Вариант Драконий всадник \tabularnewline
Вычислите интеграл:\tabularnewline
$\int\frac{\arcsin(x/2)}{\sqrt{2 - x}} \dx$\tabularnewline
\noalign{\vskip4mm}

Вычислите интеграл:\tabularnewline
$\int \frac{x(x^2 + 1) \dx}{(x + 1)(x^2 + 2x + 2)}$\tabularnewline
\noalign{\vskip4mm}

Вычислите площадь фигуры, ограниченной кривыми:\tabularnewline
$y = \sin^3 x + \cos^3 x, x \in [-\frac{\pi}{4}; \frac{\pi}{4}] \text{ и } y = 0$\tabularnewline
\noalign{\vskip4mm}

Вычислите длину кривой:\tabularnewline
$y =\sqrt{x^2 -32} + 8 \ln(x + \sqrt{x^2 - 32}), x \in [6;9]$\tabularnewline
\noalign{\vskip4mm}

\end{tabular}\tabularnewline
\noalign{\vskip4mm}
\begin{tabular}{l}
Вариант Стражи Руин \tabularnewline
Вычислите интеграл:\tabularnewline
$\int 3^x \cos x \dx$\tabularnewline
\noalign{\vskip4mm}

Вычислите интеграл:\tabularnewline
$\int \frac{\dx}{(x + 2x + 2)^{5/2}}$\tabularnewline
\noalign{\vskip4mm}

Вычислите площадь фигуры, ограниченной кривой:\tabularnewline
$y = a\sin t, x = b\cos t; a, b \in \mathbb{R}$\tabularnewline
\noalign{\vskip4mm}

Вычислите объем тела вращения данной кривой вокруг оси Ox:\tabularnewline
$y = \sqrt[4]{1 + e^{2x}}, x \in [\ln \sqrt{3}; \ln \sqrt{8}]$\tabularnewline
\noalign{\vskip4mm}

\end{tabular}& %
\begin{tabular}{l}
Вариант Забытая Грешница \tabularnewline
Вычислите интеграл:\tabularnewline
$\int 3^x \cos x \dx$\tabularnewline
\noalign{\vskip4mm}

Вычислите интеграл:\tabularnewline
$\int \sqrt[3]{1 + \sqrt[4]{x}} \dx$\tabularnewline
\noalign{\vskip4mm}

Вычислите площадь фигуры, ограниченной кривыми:\tabularnewline
$r = \frac{1}{\sqrt{1 + \cos^2 \phi}}, \text{ } \phi = 0 \text{ и } \phi = \frac{\pi}{4}$\tabularnewline
\noalign{\vskip4mm}

Вычислите длину петли кривой:\tabularnewline
$x = t^2, y = t(\frac{1}{3} - t^2)$\tabularnewline
\noalign{\vskip4mm}

\end{tabular}\tabularnewline
\noalign{\vskip4mm}
\end{tabular}

\begin{tabular}{cc}
\begin{tabular}{l}
Вариант Скорпион Нажка \tabularnewline
Вычислите интеграл:\tabularnewline
$\int 3^x \cos x \dx$\tabularnewline
\noalign{\vskip4mm}

Вычислите интеграл:\tabularnewline
$\int \sqrt[5]{\frac{x}{x + 1}} \frac{\dx}{x^3}$\tabularnewline
\noalign{\vskip4mm}

Вычислите площадь фигуры, ограниченной кривыми:\tabularnewline
$r = \frac{1}{\sqrt{1 + \cos^2 \phi}}, \text{ } \phi = 0 \text{ и } \phi = \frac{\pi}{4}$\tabularnewline
\noalign{\vskip4mm}

Вычислите длину петли кривой:\tabularnewline
$x = t^2, y = t(\frac{1}{3} - t^2)$\tabularnewline
\noalign{\vskip4mm}

\end{tabular}& %
\begin{tabular}{l}
Вариант Странствующий маг и прихожане \tabularnewline
Вычислите интеграл:\tabularnewline
$\int 3^x \cos x \dx$\tabularnewline
\noalign{\vskip4mm}

Вычислите интеграл:\tabularnewline
$\int \sqrt[5]{\frac{x}{x + 1}} \frac{\dx}{x^3}$\tabularnewline
\noalign{\vskip4mm}

Вычислите площадь фигуры, ограниченной кривыми:\tabularnewline
$y = \sin^3 x + \cos^3 x, x \in [-\frac{\pi}{4}; \frac{\pi}{4}] \text{ и } y = 0$\tabularnewline
\noalign{\vskip4mm}

Вычислите длину кривой:\tabularnewline
$r = ae^{k\phi}, \phi \in [\phi_1;\phi_2], a > 0, k \in \mathbb{R}$\tabularnewline
\noalign{\vskip4mm}

\end{tabular}\tabularnewline
\noalign{\vskip4mm}
\begin{tabular}{l}
Вариант Фрея, Возлюбленная Герцога \tabularnewline
Вычислите интеграл:\tabularnewline
$\int \frac{\sh^3 x}{\sqrt[3]{\ch^2x}} \dx$\tabularnewline
\noalign{\vskip4mm}

Вычислите интеграл:\tabularnewline
$\int \sqrt[5]{\frac{x}{x + 1}} \frac{\dx}{x^3}$\tabularnewline
\noalign{\vskip4mm}

Вычислите площадь фигуры, ограниченной кривой:\tabularnewline
$y = a\sin t, x = b\cos t; a, b \in \mathbb{R}$\tabularnewline
\noalign{\vskip4mm}

Вычислите длину петли кривой:\tabularnewline
$x = t^2, y = t(\frac{1}{3} - t^2)$\tabularnewline
\noalign{\vskip4mm}

\end{tabular}& %
\begin{tabular}{l}
Вариант Повелители скелетов \tabularnewline
Вычислите интеграл:\tabularnewline
$\int\frac{\arcsin(x/2)}{\sqrt{2 - x}} \dx$\tabularnewline
\noalign{\vskip4mm}

Вычислите интеграл:\tabularnewline
$\int \frac{\dx}{(x + 2x + 2)^{5/2}}$\tabularnewline
\noalign{\vskip4mm}

Вычислите площадь фигуры, ограниченной кривой:\tabularnewline
$y = a\sin t, x = b\cos t; a, b \in \mathbb{R}$\tabularnewline
\noalign{\vskip4mm}

Вычислите объем тела вращения данной кривой вокруг оси Ox:\tabularnewline
$y = \sqrt[4]{1 + e^{2x}}, x \in [\ln \sqrt{3}; \ln \sqrt{8}]$\tabularnewline
\noalign{\vskip4mm}

\end{tabular}\tabularnewline
\noalign{\vskip4mm}
\begin{tabular}{l}
Вариант Алчный Демон \tabularnewline
Вычислите интеграл:\tabularnewline
$\int \frac{\dx}{2\cos^2 x+ \sin x \cos x + \sin^2 x}$\tabularnewline
\noalign{\vskip4mm}

Вычислите интеграл:\tabularnewline
$\int \sqrt[5]{\frac{x}{x + 1}} \frac{\dx}{x^3}$\tabularnewline
\noalign{\vskip4mm}

Вычислите площадь фигуры, ограниченной кривыми:\tabularnewline
$r = \frac{1}{\sqrt{1 + \cos^2 \phi}}, \text{ } \phi = 0 \text{ и } \phi = \frac{\pi}{4}$\tabularnewline
\noalign{\vskip4mm}

Вычислите длину петли кривой:\tabularnewline
$x = t^2, y = t(\frac{1}{3} - t^2)$\tabularnewline
\noalign{\vskip4mm}

\end{tabular}& %
\begin{tabular}{l}
Вариант Мита Губительная королева \tabularnewline
Вычислите интеграл:\tabularnewline
$\int \frac{\sh^3 x}{\sqrt[3]{\ch^2x}} \dx$\tabularnewline
\noalign{\vskip4mm}

Вычислите интеграл:\tabularnewline
$\int \frac{\dx}{(x + 2x + 2)^{5/2}}$\tabularnewline
\noalign{\vskip4mm}

Вычислите площадь фигуры, ограниченной кривой:\tabularnewline
$y = a\sin t, x = b\cos t; a, b \in \mathbb{R}$\tabularnewline
\noalign{\vskip4mm}

Вычислите длину петли кривой:\tabularnewline
$x = t^2, y = t(\frac{1}{3} - t^2)$\tabularnewline
\noalign{\vskip4mm}

\end{tabular}\tabularnewline
\noalign{\vskip4mm}
\end{tabular}

\begin{tabular}{cc}
\begin{tabular}{l}
Вариант Старый Железный Король \tabularnewline
Вычислите интеграл:\tabularnewline
$\int 3^x \cos x \dx$\tabularnewline
\noalign{\vskip4mm}

Вычислите интеграл:\tabularnewline
$\int \sqrt[3]{1 + \sqrt[4]{x}} \dx$\tabularnewline
\noalign{\vskip4mm}

Вычислите площадь фигуры, ограниченной кривыми:\tabularnewline
$r = \frac{1}{\sqrt{1 + \cos^2 \phi}}, \text{ } \phi = 0 \text{ и } \phi = \frac{\pi}{4}$\tabularnewline
\noalign{\vskip4mm}

Вычислите длину кривой:\tabularnewline
$y =\sqrt{x^2 -32} + 8 \ln(x + \sqrt{x^2 - 32}), x \in [6;9]$\tabularnewline
\noalign{\vskip4mm}

\end{tabular}& %
\begin{tabular}{l}
Вариант Гниющий \tabularnewline
Вычислите интеграл:\tabularnewline
$\int \frac{1 + \tg x}{\sin 2x} \dx$\tabularnewline
\noalign{\vskip4mm}

Вычислите интеграл:\tabularnewline
$\int \frac{\dx}{(x + 2x + 2)^{5/2}}$\tabularnewline
\noalign{\vskip4mm}

Вычислите площадь фигуры, ограниченной кривой:\tabularnewline
$y = a\sin t, x = b\cos t; a, b \in \mathbb{R}$\tabularnewline
\noalign{\vskip4mm}

Вычислите длину кривой:\tabularnewline
$r = ae^{k\phi}, \phi \in [\phi_1;\phi_2], a > 0, k \in \mathbb{R}$\tabularnewline
\noalign{\vskip4mm}

\end{tabular}\tabularnewline
\noalign{\vskip4mm}
\begin{tabular}{l}
Вариант Дракон-Страж \tabularnewline
Вычислите интеграл:\tabularnewline
$\int \frac{\sh^3 x}{\sqrt[3]{\ch^2x}} \dx$\tabularnewline
\noalign{\vskip4mm}

Вычислите интеграл:\tabularnewline
$\int \sqrt[3]{1 + \sqrt[4]{x}} \dx$\tabularnewline
\noalign{\vskip4mm}

Вычислите площадь фигуры, ограниченной кривыми:\tabularnewline
$r = \frac{1}{\sqrt{1 + \cos^2 \phi}}, \text{ } \phi = 0 \text{ и } \phi = \frac{\pi}{4}$\tabularnewline
\noalign{\vskip4mm}

Вычислите объем тела вращения данной кривой вокруг оси Ox:\tabularnewline
$y = \sqrt[4]{1 + e^{2x}}, x \in [\ln \sqrt{3}; \ln \sqrt{8}]$\tabularnewline
\noalign{\vskip4mm}

\end{tabular}& %
\begin{tabular}{l}
Вариант Зеркальный рыцарь \tabularnewline
Вычислите интеграл:\tabularnewline
$\int \frac{1 + \tg x}{\sin 2x} \dx$\tabularnewline
\noalign{\vskip4mm}

Вычислите интеграл:\tabularnewline
$\int \sqrt[5]{\frac{x}{x + 1}} \frac{\dx}{x^3}$\tabularnewline
\noalign{\vskip4mm}

Вычислите площадь фигуры, ограниченной кривыми:\tabularnewline
$y = \sin^3 x + \cos^3 x, x \in [-\frac{\pi}{4}; \frac{\pi}{4}] \text{ и } y = 0$\tabularnewline
\noalign{\vskip4mm}

Вычислите длину петли кривой:\tabularnewline
$x = t^2, y = t(\frac{1}{3} - t^2)$\tabularnewline
\noalign{\vskip4mm}

\end{tabular}\tabularnewline
\noalign{\vskip4mm}
\begin{tabular}{l}
Вариант Демон песни \tabularnewline
Вычислите интеграл:\tabularnewline
$\int\frac{\arcsin(x/2)}{\sqrt{2 - x}} \dx$\tabularnewline
\noalign{\vskip4mm}

Вычислите интеграл:\tabularnewline
$\int \frac{x(x^2 + 1) \dx}{(x + 1)(x^2 + 2x + 2)}$\tabularnewline
\noalign{\vskip4mm}

Вычислите площадь фигуры, ограниченной кривой:\tabularnewline
$y = a\sin t, x = b\cos t; a, b \in \mathbb{R}$\tabularnewline
\noalign{\vskip4mm}

Вычислите объем тела вращения данной кривой вокруг оси Ox:\tabularnewline
$y = \sqrt[4]{1 + e^{2x}}, x \in [\ln \sqrt{3}; \ln \sqrt{8}]$\tabularnewline
\noalign{\vskip4mm}

\end{tabular}& %
\begin{tabular}{l}
Вариант Вельстадт Королевский защитник \tabularnewline
Вычислите интеграл:\tabularnewline
$\int \frac{\dx}{2\cos^2 x+ \sin x \cos x + \sin^2 x}$\tabularnewline
\noalign{\vskip4mm}

Вычислите интеграл:\tabularnewline
$\int \sqrt[5]{\frac{x}{x + 1}} \frac{\dx}{x^3}$\tabularnewline
\noalign{\vskip4mm}

Вычислите площадь фигуры, ограниченной кривой:\tabularnewline
$y = a\sin t, x = b\cos t; a, b \in \mathbb{R}$\tabularnewline
\noalign{\vskip4mm}

Вычислите длину петли кривой:\tabularnewline
$x = t^2, y = t(\frac{1}{3} - t^2)$\tabularnewline
\noalign{\vskip4mm}

\end{tabular}\tabularnewline
\noalign{\vskip4mm}
\end{tabular}


\end{document}