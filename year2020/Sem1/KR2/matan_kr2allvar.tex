
\documentclass[russian]{article}
\usepackage[T2A,T1]{fontenc}
\usepackage[utf8]{inputenc}
\usepackage[a4paper]{geometry}
\geometry{verbose,tmargin=1cm,bmargin=2cm,lmargin=2cm,rmargin=2cm}
\usepackage{amsmath}
\usepackage{tikz}

\makeatletter

\DeclareRobustCommand{\cyrtext}{%
 \fontencoding{T2A}\selectfont\def\encodingdefault{T2A}}
\DeclareRobustCommand{\textcyr}[1]{\leavevmode{\cyrtext #1}}
\AtBeginDocument{\DeclareFontEncoding{T2A}{}{}}


\DeclareMathOperator{\ch}{ch}
\DeclareMathOperator{\tg}{tg}
\DeclareMathOperator{\ctg}{ctg}
\DeclareMathOperator{\arcctg}{arcctg}
% \DeclareMathOperator{\arcsin}{arcsin}
\DeclareMathOperator{\arctg}{arctg}
\DeclareMathOperator{\sign}{sign}

\makeatother

\usepackage[russian]{babel}
\begin{document}

Вычислите производную функции
$$\frac{3 - \sin x}{2}\sqrt{\cos^2 x - 2 \sin x} + 2\arcsin \frac{1 + \sin x}{\sqrt{2}} + x^\frac{7}{\ln x}$$

Вычислите производную $n$-ого порядка
$$\ln \left((x - 1)^{2x}\right)$$

Вычислите производную $n$-ого порядка
$$\frac{1}{\sqrt{1 - 2x}}$$


Разложите по формуле Тейлора до $o((x + 1)^{2n})$
$$\frac{(x + 1)^3}{\sqrt{x^2 + 2x + 2}}$$

Разложите по формуле Тейлора до $o((x - \frac{\pi}{2})^{2n + 1})$
$$(x^2 - \pi x) \cos\left(x + \frac{\pi}{2}\right)$$

Разложите по формуле Тейлора до $o(x^{4n})$
$$\frac{1}{\sqrt{x^2 + 2} + \sqrt{2 - x^2}}$$


Вычислите предел
$$\lim_{x \to 0}\frac{\sin(xe^x) + \sin (xe^{-x}) - 2x - \frac{2x^3}{3}}{x^5}$$

Вычислите предел
$$\lim_{x \to 0} \left(\frac{1}{e}(1 + x)^{\frac{1}{x}} + \frac{2x}{4 + 5x}\right)^{\ctg^2 x} $$

Вычислите предел
$$\lim_{x \to 0} \left(\frac{2x}{x - 2} + \ln(e + xe^{x + 1})\right)^{\frac{1}{x^3}}$$


Из трех досок одинаковой ширины нужно сколотить желоб. При каком угле наклона боковых стенок площадь поперечного сеченияжелоба будет наибольшей?

\begin{tikzpicture}
\draw [ultra thick, fill=lightgray] (-1.5, 2) -- (0, 0) -- (2.5, 0) -- (4, 2);
\node at (1.25, 1) {$S \to \max$};
\draw [dashed, thick] (0, 0) -- (0, 2);
\draw [dashed, thick] (2.5, 0) -- (2.5, 2);
\draw (0, 0.5) arc (90:130:0.5);
\draw (2.5, 0.5) arc (90:50:0.5);
\node [left] at (0, 0.8) {$\alpha$};
\node [right] at (2.5, 0.8) {$\alpha$};
\node at (-0.9, 0.9) {$\ell$};
\node at (3.4, 0.9) {$\ell$};
\node [below] at (1.25, 0) {$\ell$};
\node at (5, 1) {$\alpha = ?$};
\end{tikzpicture}

Чтобы уменьшить трение жидкости о стенки канала, площадь, смачиваемая водой, должна быть возможно меньшей. Показать, что лучшей формой открытого прямоугольного канала с заданной площадью поперечного сечения является такая, при которой ширина канала в два раза больше его высоты.

\begin{tikzpicture}
 \draw [ultra thick, fill=lightgray] (0, 2) -- (0, 0) -- (4, 0) -- (4, 2);
 \node at (2, 1) {$S$};
 \node [left] at (0, 1) {$h$};
 \node [right] at (4, 1) {$h$};
 \node [below] at (2, 0) {$\ell$};
 \node at (6, 1) {$2h + \ell \to \min$};
\end{tikzpicture} 

Из круглого бревна вытесывается балка с прямоугольным поперечным сечением. Считая, что прочность балки пропорциональна $ah^2$, где $а$ -- основание, $h$ -- высота прямоугольника, найти такое отношение $h/a$, при котором балка будет иметь наибольшую прочность.

\begin{tikzpicture}
  \draw [ultra thick, fill=lightgray] (2, 1.5) -- (-2, 1.5) -- (-2, -1.5) -- (2, -1.5) -- (2, 1.5);
  \draw [thick] (0, 0) circle (2.5);
  \node [below] at (0, 1.5) {$a$};
  \node [left] at (2, 0) {$h$};
  \node [right] at (3, 1) {$ah^2 \to \max$};
\end{tikzpicture} 
 

\end{document}

