
\documentclass[russian]{article}
\usepackage[T2A,T1]{fontenc}
\usepackage[utf8]{inputenc}
\usepackage[a4paper]{geometry}
\geometry{verbose,tmargin=1cm,bmargin=0cm,lmargin=0cm,rmargin=0cm}
\usepackage{amsmath}

\makeatletter

\DeclareRobustCommand{\cyrtext}{%
  \fontencoding{T2A}\selectfont\def\encodingdefault{T2A}}
\DeclareRobustCommand{\textcyr}[1]{\leavevmode{\cyrtext #1}}
\AtBeginDocument{\DeclareFontEncoding{T2A}{}{}}


\makeatother

\usepackage[russian]{babel}
\begin{document}

\begin{tabular}{p{\textwidth}}
  \begin{tabular}{l}
  Вариант БиМО \tabularnewline
  Постройте график в полярных координатах:\tabularnewline
  $r\cos(\phi - \frac{\pi}{4}) = -1$\tabularnewline
  \noalign{\vskip4mm}
  
  Докажите предел по определению:\tabularnewline
  $\lim\limits_{n \to +\infty} \frac{2^{n + 2} + 3^{n + 3}}{2^n + 3^n} = 27$\tabularnewline
  \noalign{\vskip4mm}
  
  Докажите, что последовательность расходится:\tabularnewline
  $x_n = \sum_{k = 2}^n \frac{1}{\sqrt{k} \ln k}$\tabularnewline
  \noalign{\vskip4mm}
  
  Вычислите предел функции:\tabularnewline
  $\lim\limits_{x \to 0} \frac{\ln(1 + 3x + x^2) + \ln(1 - 3x + x^2)}{x^2}$\tabularnewline
  \noalign{\vskip4mm}
  
  \end{tabular}\tabularnewline
  \hline
  \noalign{\vskip4mm}
  \begin{tabular}{l}
  Вариант Билли  \tabularnewline
  Постройте график в декартовых координатах:\tabularnewline
  $y = \sin(\cos x)$\tabularnewline
  \noalign{\vskip4mm}
  
  Докажите предел по определению:\tabularnewline
  $\lim\limits_{n\to+\infty}\frac{\sqrt{3n + 1} + \sqrt{3n - 1}}{\sqrt{3n - 2}}=2$\tabularnewline
  \noalign{\vskip4mm}
  
  Докажите, что последовательность расходится:\tabularnewline
  $x_n = \sqrt{n!}$\tabularnewline
  \noalign{\vskip4mm}
  
  Вычислите предел функции:\tabularnewline
  $\lim\limits_{x\to 0} \frac{10^x - 2^x}{\tan x}$\tabularnewline
  \noalign{\vskip4mm}
  
  \end{tabular}\tabularnewline
  \hline
  \noalign{\vskip4mm}
  \begin{tabular}{l}
  Вариант Гантер  \tabularnewline
  Постройте график в полярных координатах:\tabularnewline
  $r = 8\sin(\phi - \frac{\pi}{3})$\tabularnewline
  \noalign{\vskip4mm}
  
  Докажите предел по определению:\tabularnewline
  $\lim\limits_{n\to+\infty}\frac{\sqrt{3n + 1} + \sqrt{3n - 1}}{\sqrt{3n - 2}}=2$\tabularnewline
  \noalign{\vskip4mm}
  
  Докажите, что последовательность сходится:\tabularnewline
  $x_n = \sum_{k = 1}^n \frac{(-1)^k}{k}$\tabularnewline
  \noalign{\vskip4mm}
  
  Вычислите предел функции:\tabularnewline
  $\lim\limits_{x \to \pi} \frac{\sin x}{\pi^3 - x^3}$\tabularnewline
  \noalign{\vskip4mm}
  
  \end{tabular}\tabularnewline
  \hline
  \noalign{\vskip4mm}
  \begin{tabular}{l}
  Вариант Гусь-выбирусь \tabularnewline
  Постройте график в декартовых координатах:\tabularnewline
  $y = \sin(\cos x)$\tabularnewline
  \noalign{\vskip4mm}
  
  Докажите предел по определению:\tabularnewline
  $\lim\limits_{n \to +\infty} \frac{\ln(n + 1)}{\ln(n^2 + 1)} = \frac{1}{2}$\tabularnewline
  \noalign{\vskip4mm}
  
  Докажите, что последовательность сходится:\tabularnewline
  $x_n = \sum_{k = 1}^n \frac{(-1)^k}{k}$\tabularnewline
  \noalign{\vskip4mm}
  
  Вычислите предел функции:\tabularnewline
  $\lim\limits_{x\to 0} (\cos{x})^\frac{1}{x^2}$\tabularnewline
  \noalign{\vskip4mm}
  
  \end{tabular}\tabularnewline
  \hline
  \noalign{\vskip4mm}
  \end{tabular}
  
  \begin{tabular}{p{\textwidth}}
  \begin{tabular}{l}
  Вариант Деревяшка  \tabularnewline
  Постройте график в декартовых координатах:\tabularnewline
  $y = 2^\frac{x + 1}{x}$\tabularnewline
  \noalign{\vskip4mm}
  
  Докажите предел по определению:\tabularnewline
  $\lim\limits_{n \to +\infty} \frac{\ln(n + 1)}{\ln(n^2 + 1)} = \frac{1}{2}$\tabularnewline
  \noalign{\vskip4mm}
  
  Докажите, что последовательность сходится:\tabularnewline
  $x_n=\sum_{k = 0}^{n}\frac{k^2 + 1}{k!}$\tabularnewline
  \noalign{\vskip4mm}
  
  Вычислите предел функции:\tabularnewline
  $\lim\limits_{x \to \infty} (\sqrt{x^4 + x^2\sqrt{x^4 + 1}} - \sqrt{2x^4})$\tabularnewline
  \noalign{\vskip4mm}
  
  \end{tabular}\tabularnewline
  \hline
  \noalign{\vskip4mm}
  \begin{tabular}{l}
  Вариант Коричный пряник  \tabularnewline
  Постройте график в полярных координатах:\tabularnewline
  $r = 8\sin(\phi - \frac{\pi}{3})$\tabularnewline
  \noalign{\vskip4mm}
  
  Докажите предел по определению:\tabularnewline
  $\lim\limits_{n\to+\infty}\frac{\sqrt{\sqrt{n} + n^2}}{n + \sin{n}}=1$\tabularnewline
  \noalign{\vskip4mm}
  
  Докажите, что последовательность сходится:\tabularnewline
  $x_n=\sum_{k = 0}^{n}\frac{k^2 + 1}{k!}$\tabularnewline
  \noalign{\vskip4mm}
  
  Вычислите предел функции:\tabularnewline
  $\lim\limits_{x\to 0} \frac{10^x - 2^x}{\tan x}$\tabularnewline
  \noalign{\vskip4mm}
  
  \end{tabular}\tabularnewline
  \hline
  \noalign{\vskip4mm}
  \begin{tabular}{l}
  Вариант Король Ооо \tabularnewline
  Постройте график в полярных координатах:\tabularnewline
  $r = 8\sin(\phi - \frac{\pi}{3})$\tabularnewline
  \noalign{\vskip4mm}
  
  Докажите предел по определению:\tabularnewline
  $\lim\limits_{n\to+\infty}\frac{\sqrt{\sqrt{n} + n^2}}{n + \sin{n}}=1$\tabularnewline
  \noalign{\vskip4mm}
  
  Докажите, что последовательность расходится:\tabularnewline
  $x_n = \sqrt[n]{((-1)^n - 1)^n + 1}$\tabularnewline
  \noalign{\vskip4mm}
  
  Вычислите предел функции:\tabularnewline
  $\lim\limits_{x \to 0} \frac{\ln\cos 5x}{\ln\cos 4x}$\tabularnewline
  \noalign{\vskip4mm}
  
  \end{tabular}\tabularnewline
  \hline
  \noalign{\vskip4mm}
  \begin{tabular}{l}
  Вариант Космическая сова \tabularnewline
  Постройте график в полярных координатах:\tabularnewline
  $r = e^\phi$\tabularnewline
  \noalign{\vskip4mm}
  
  Докажите предел по определению:\tabularnewline
  $\lim\limits_{n\to+\infty}\sqrt{\frac{4n + 1}{n - 1}}=2$\tabularnewline
  \noalign{\vskip4mm}
  
  Докажите, что последовательность расходится:\tabularnewline
  $x_n = \sum_{k = 1}^n \frac{k^2}{(k^2 + 1)(k - 1)}$\tabularnewline
  \noalign{\vskip4mm}
  
  Вычислите предел функции:\tabularnewline
  $\lim\limits_{x \to \infty} (\sqrt{x^4 + x^2\sqrt{x^4 + 1}} - \sqrt{2x^4})$\tabularnewline
  \noalign{\vskip4mm}
  
  \end{tabular}\tabularnewline
  \hline
  \noalign{\vskip4mm}
  \end{tabular}
  
  \begin{tabular}{p{\textwidth}}
  \begin{tabular}{l}
  Вариант Леди Ливнерог  \tabularnewline
  Постройте график в декартовых координатах:\tabularnewline
  $y = \frac{2^x}{2^{x + 1} - 1}$\tabularnewline
  \noalign{\vskip4mm}
  
  Докажите предел по определению:\tabularnewline
  $\lim\limits_{n \to +\infty} \left(8 - \frac{1}{n^2}\right)^{-\frac{1}{3}} = \frac{1}{2}$\tabularnewline
  \noalign{\vskip4mm}
  
  Докажите, что последовательность расходится:\tabularnewline
  $x_n = \sum_{k = 2}^n \frac{1}{\sqrt{k} \ln k}$\tabularnewline
  \noalign{\vskip4mm}
  
  Вычислите предел функции:\tabularnewline
  $\lim\limits_{x\to 0} (e^x + x)^\frac{1}{x}$\tabularnewline
  \noalign{\vskip4mm}
  
  \end{tabular}\tabularnewline
  \hline
  \noalign{\vskip4mm}
  \begin{tabular}{l}
  Вариант Ледяной Король \tabularnewline
  Постройте график в декартовых координатах:\tabularnewline
  $y = \frac{2^x}{2^{x + 1} - 1}$\tabularnewline
  \noalign{\vskip4mm}
  
  Докажите предел по определению:\tabularnewline
  $\lim\limits_{n \to +\infty} \frac{\sqrt[3]{n^3 + 1}}{n + 1} = 1$\tabularnewline
  \noalign{\vskip4mm}
  
  Докажите, что последовательность расходится:\tabularnewline
  $x_n = \sqrt[n]{((-1)^n - 1)^n + 1}$\tabularnewline
  \noalign{\vskip4mm}
  
  Вычислите предел функции:\tabularnewline
  $\lim\limits_{x \to 0} \frac{\ln\cos 5x}{\ln\cos 4x}$\tabularnewline
  \noalign{\vskip4mm}
  
  \end{tabular}\tabularnewline
  \hline
  \noalign{\vskip4mm}
  \begin{tabular}{l}
  Вариант Лич \tabularnewline
  Постройте график в полярных координатах:\tabularnewline
  $r\cos(\phi - \frac{\pi}{4}) = -1$\tabularnewline
  \noalign{\vskip4mm}
  
  Докажите предел по определению:\tabularnewline
  $\lim\limits_{n \to +\infty} (\sqrt{n^2 + n} - n) = \frac{1}{2}$\tabularnewline
  \noalign{\vskip4mm}
  
  Докажите, что последовательность расходится:\tabularnewline
  $x_n = \sum_{k = 2}^n \frac{1}{\sqrt{k} \ln k}$\tabularnewline
  \noalign{\vskip4mm}
  
  Вычислите предел функции:\tabularnewline
  $\lim\limits_{x \to \infty} (\sqrt{x^4 + x^2\sqrt{x^4 + 1}} - \sqrt{2x^4})$\tabularnewline
  \noalign{\vskip4mm}
  
  \end{tabular}\tabularnewline
  \hline
  \noalign{\vskip4mm}
  \begin{tabular}{l}
  Вариант Марселин \tabularnewline
  Постройте график в декартовых координатах:\tabularnewline
  $y = 2^\frac{x + 1}{x}$\tabularnewline
  \noalign{\vskip4mm}
  
  Докажите предел по определению:\tabularnewline
  $\lim\limits_{n\to+\infty}\frac{\sqrt{3n + 1} + \sqrt{3n - 1}}{\sqrt{3n - 2}}=2$\tabularnewline
  \noalign{\vskip4mm}
  
  Докажите, что последовательность расходится:\tabularnewline
  $x_n = \sum_{k = 2}^n \frac{1}{\sqrt{k} \ln k}$\tabularnewline
  \noalign{\vskip4mm}
  
  Вычислите предел функции:\tabularnewline
  $\lim\limits_{x\to 0} (e^x + x)^\frac{1}{x}$\tabularnewline
  \noalign{\vskip4mm}
  
  \end{tabular}\tabularnewline
  \hline
  \noalign{\vskip4mm}
  \end{tabular}
  
  \begin{tabular}{p{\textwidth}}
  \begin{tabular}{l}
  Вариант Мятный лакей  \tabularnewline
  Постройте график в декартовых координатах:\tabularnewline
  $y = 2^\frac{x + 1}{x}$\tabularnewline
  \noalign{\vskip4mm}
  
  Докажите предел по определению:\tabularnewline
  $\lim\limits_{n \to +\infty} \frac{2^{n + 2} + 3^{n + 3}}{2^n + 3^n} = 27$\tabularnewline
  \noalign{\vskip4mm}
  
  Докажите, что последовательность расходится:\tabularnewline
  $x_n = \sum_{k = 2}^n \frac{1}{\sqrt{k} \ln k}$\tabularnewline
  \noalign{\vskip4mm}
  
  Вычислите предел функции:\tabularnewline
  $\lim\limits_{x \to \pi} \frac{\sin x}{\pi^3 - x^3}$\tabularnewline
  \noalign{\vskip4mm}
  
  \end{tabular}\tabularnewline
  \hline
  \noalign{\vskip4mm}
  \begin{tabular}{l}
  Вариант Огненная принцесса  \tabularnewline
  Постройте график в полярных координатах:\tabularnewline
  $r\cos(\phi - \frac{\pi}{4}) = -1$\tabularnewline
  \noalign{\vskip4mm}
  
  Докажите предел по определению:\tabularnewline
  $\lim\limits_{n\to+\infty}\frac{\sqrt{3n + 1} + \sqrt{3n - 1}}{\sqrt{3n - 2}}=2$\tabularnewline
  \noalign{\vskip4mm}
  
  Докажите, что последовательность расходится:\tabularnewline
  $x_n = \sqrt{n!}$\tabularnewline
  \noalign{\vskip4mm}
  
  Вычислите предел функции:\tabularnewline
  $\lim\limits_{x\to 0} (e^x + x)^\frac{1}{x}$\tabularnewline
  \noalign{\vskip4mm}
  
  \end{tabular}\tabularnewline
  \hline
  \noalign{\vskip4mm}
  \begin{tabular}{l}
  Вариант Призмо \tabularnewline
  Постройте график в декартовых координатах:\tabularnewline
  $y = \sqrt{1 + x^2}$\tabularnewline
  \noalign{\vskip4mm}
  
  Докажите предел по определению:\tabularnewline
  $\lim\limits_{n \to +\infty} \left(8 - \frac{1}{n^2}\right)^{-\frac{1}{3}} = \frac{1}{2}$\tabularnewline
  \noalign{\vskip4mm}
  
  Докажите, что последовательность расходится:\tabularnewline
  $x_n = \sqrt[n]{((-1)^n - 1)^n + 1}$\tabularnewline
  \noalign{\vskip4mm}
  
  Вычислите предел функции:\tabularnewline
  $\lim\limits_{x \to 0} \frac{e^{\sin 5x} - e^{\sin x}}{\ln(1 + 2x)}$\tabularnewline
  \noalign{\vskip4mm}
  
  \end{tabular}\tabularnewline
  \hline
  \noalign{\vskip4mm}
  \begin{tabular}{l}
  Вариант Принцесса Жевачка \tabularnewline
  Постройте график в полярных координатах:\tabularnewline
  $r = e^\phi$\tabularnewline
  \noalign{\vskip4mm}
  
  Докажите предел по определению:\tabularnewline
  $\lim\limits_{n \to +\infty} \frac{\sqrt[3]{n^3 + 1}}{n + 1} = 1$\tabularnewline
  \noalign{\vskip4mm}
  
  Докажите, что последовательность расходится:\tabularnewline
  $x_n = \sqrt[n]{((-1)^n - 1)^n + 1}$\tabularnewline
  \noalign{\vskip4mm}
  
  Вычислите предел функции:\tabularnewline
  $\lim\limits_{x\to 0} (\cos{x})^\frac{1}{x^2}$\tabularnewline
  \noalign{\vskip4mm}
  
  \end{tabular}\tabularnewline
  \hline
  \noalign{\vskip4mm}
  \end{tabular}
  
  \begin{tabular}{p{\textwidth}}
  \begin{tabular}{l}
  Вариант Принцесса Пупырчатого королевства \tabularnewline
  Постройте график в декартовых координатах:\tabularnewline
  $y = 2^\frac{x + 1}{x}$\tabularnewline
  \noalign{\vskip4mm}
  
  Докажите предел по определению:\tabularnewline
  $\lim\limits_{n\to+\infty}\frac{\sqrt{\sqrt{n} + n^2}}{n + \sin{n}}=1$\tabularnewline
  \noalign{\vskip4mm}
  
  Докажите, что последовательность сходится:\tabularnewline
  $x_n = \sum_{k = 1}^n \frac{(-1)^k}{k}$\tabularnewline
  \noalign{\vskip4mm}
  
  Вычислите предел функции:\tabularnewline
  $\lim\limits_{x \to \infty} (\sqrt{x^4 + x^2\sqrt{x^4 + 1}} - \sqrt{2x^4})$\tabularnewline
  \noalign{\vskip4mm}
  
  \end{tabular}\tabularnewline
  \hline
  \noalign{\vskip4mm}
  \begin{tabular}{l}
  Вариант Принцесса Слизь \tabularnewline
  Постройте график в полярных координатах:\tabularnewline
  $r\cos(\phi - \frac{\pi}{4}) = -1$\tabularnewline
  \noalign{\vskip4mm}
  
  Докажите предел по определению:\tabularnewline
  $\lim\limits_{n \to +\infty} \frac{\ln(n + 1)}{\ln(n^2 + 1)} = \frac{1}{2}$\tabularnewline
  \noalign{\vskip4mm}
  
  Докажите, что последовательность расходится:\tabularnewline
  $x_n = \sqrt{n!}$\tabularnewline
  \noalign{\vskip4mm}
  
  Вычислите предел функции:\tabularnewline
  $\lim\limits_{x\to 0} (e^x + x)^\frac{1}{x}$\tabularnewline
  \noalign{\vskip4mm}
  
  \end{tabular}\tabularnewline
  \hline
  \noalign{\vskip4mm}
  \begin{tabular}{l}
  Вариант Принцесса Хот-дог \tabularnewline
  Постройте график в декартовых координатах:\tabularnewline
  $y = 2^\frac{x + 1}{x}$\tabularnewline
  \noalign{\vskip4mm}
  
  Докажите предел по определению:\tabularnewline
  $\lim\limits_{n\to+\infty}\sqrt{\frac{4n + 1}{n - 1}}=2$\tabularnewline
  \noalign{\vskip4mm}
  
  Докажите, что последовательность сходится:\tabularnewline
  $x_n = 0.77..7 \text{($n$ семерок)}$\tabularnewline
  \noalign{\vskip4mm}
  
  Вычислите предел функции:\tabularnewline
  $\lim\limits_{x \to 0} \frac{\ln(1 + 3x + x^2) + \ln(1 - 3x + x^2)}{x^2}$\tabularnewline
  \noalign{\vskip4mm}
  
  \end{tabular}\tabularnewline
  \hline
  \noalign{\vskip4mm}
  \begin{tabular}{l}
  Вариант Принцесса Черепаха \tabularnewline
  Постройте график в полярных координатах:\tabularnewline
  $r\cos(\phi - \frac{\pi}{4}) = -1$\tabularnewline
  \noalign{\vskip4mm}
  
  Докажите предел по определению:\tabularnewline
  $\lim\limits_{n \to +\infty} \left(8 - \frac{1}{n^2}\right)^{-\frac{1}{3}} = \frac{1}{2}$\tabularnewline
  \noalign{\vskip4mm}
  
  Докажите, что последовательность расходится:\tabularnewline
  $x_n = \frac{n \cos\pi n - 1}{2n}$\tabularnewline
  \noalign{\vskip4mm}
  
  Вычислите предел функции:\tabularnewline
  $\lim\limits_{x\to 0} (e^x + x)^\frac{1}{x}$\tabularnewline
  \noalign{\vskip4mm}
  
  \end{tabular}\tabularnewline
  \hline
  \noalign{\vskip4mm}
  \end{tabular}
  
  \begin{tabular}{p{\textwidth}}
  \begin{tabular}{l}
  Вариант Принцесса Ягода  \tabularnewline
  Постройте график в полярных координатах:\tabularnewline
  $r\cos(\phi - \frac{\pi}{4}) = -1$\tabularnewline
  \noalign{\vskip4mm}
  
  Докажите предел по определению:\tabularnewline
  $\lim\limits_{n\to+\infty}\sqrt{\frac{4n + 1}{n - 1}}=2$\tabularnewline
  \noalign{\vskip4mm}
  
  Докажите, что последовательность сходится:\tabularnewline
  $x_n = 0.77..7 \text{($n$ семерок)}$\tabularnewline
  \noalign{\vskip4mm}
  
  Вычислите предел функции:\tabularnewline
  $\lim\limits_{x\to 0} \frac{10^x - 2^x}{\tan x}$\tabularnewline
  \noalign{\vskip4mm}
  
  \end{tabular}\tabularnewline
  \hline
  \noalign{\vskip4mm}
  \begin{tabular}{l}
  Вариант Рикардио \tabularnewline
  Постройте график в полярных координатах:\tabularnewline
  $r = 8\sin(\phi - \frac{\pi}{3})$\tabularnewline
  \noalign{\vskip4mm}
  
  Докажите предел по определению:\tabularnewline
  $\lim\limits_{n \to +\infty} \left(8 - \frac{1}{n^2}\right)^{-\frac{1}{3}} = \frac{1}{2}$\tabularnewline
  \noalign{\vskip4mm}
  
  Докажите, что последовательность сходится:\tabularnewline
  $x_n=\sum_{k = 0}^{n}\frac{k^2 + 1}{k!}$\tabularnewline
  \noalign{\vskip4mm}
  
  Вычислите предел функции:\tabularnewline
  $\lim\limits_{x \to 0} \frac{\ln(1 + 3x + x^2) + \ln(1 - 3x + x^2)}{x^2}$\tabularnewline
  \noalign{\vskip4mm}
  
  \end{tabular}\tabularnewline
  \hline
  \noalign{\vskip4mm}
  \begin{tabular}{l}
  Вариант Сильная Сьюзан \tabularnewline
  Постройте график в полярных координатах:\tabularnewline
  $r = 8\sin(\phi - \frac{\pi}{3})$\tabularnewline
  \noalign{\vskip4mm}
  
  Докажите предел по определению:\tabularnewline
  $\lim\limits_{n \to +\infty} \left(8 - \frac{1}{n^2}\right)^{-\frac{1}{3}} = \frac{1}{2}$\tabularnewline
  \noalign{\vskip4mm}
  
  Докажите, что последовательность расходится:\tabularnewline
  $x_n = \sum_{k = 2}^n \frac{1}{\sqrt{k} \ln k}$\tabularnewline
  \noalign{\vskip4mm}
  
  Вычислите предел функции:\tabularnewline
  $\lim\limits_{x \to 0} \frac{\ln\cos 5x}{\ln\cos 4x}$\tabularnewline
  \noalign{\vskip4mm}
  
  \end{tabular}\tabularnewline
  \hline
  \noalign{\vskip4mm}
  \begin{tabular}{l}
  Вариант Хансон Абадир \tabularnewline
  Постройте график в полярных координатах:\tabularnewline
  $r = e^\phi$\tabularnewline
  \noalign{\vskip4mm}
  
  Докажите предел по определению:\tabularnewline
  $\lim\limits_{n \to +\infty} \frac{\sqrt[3]{n^3 + 1}}{n + 1} = 1$\tabularnewline
  \noalign{\vskip4mm}
  
  Докажите, что последовательность расходится:\tabularnewline
  $x_n = \frac{n \cos\pi n - 1}{2n}$\tabularnewline
  \noalign{\vskip4mm}
  
  Вычислите предел функции:\tabularnewline
  $\lim\limits_{x\to 0} (\cos{x})^\frac{1}{x^2}$\tabularnewline
  \noalign{\vskip4mm}
  
  \end{tabular}\tabularnewline
  \hline
  \noalign{\vskip4mm}
  \end{tabular}

 
\end{document}
